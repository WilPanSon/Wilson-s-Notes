\documentclass[10pt]{article}
\usepackage[utf8]{inputenc}
\usepackage{amsmath,amsthm,amsfonts,amssymb,mathrsfs}
\usepackage{enumitem}
\usepackage{color, graphicx}


\newcommand{\dd}{\mathrm{d}}
\newcommand{\E}{\mathbb{E}}
\newcommand{\1}{\textbf{1}}
\newcommand{\R}{\mathbb{R}}
\newcommand{\C}{\mathbb{C}}
\newcommand{\Z}{\mathbb{Z}}
\newcommand{\Q}{\mathbb{Q}}
\newcommand{\p}[1]{\mathbb{P}\left( #1 \right)}
\newcommand{\e}{\varepsilon}
\newcommand{\cF}{\mathcal{F}}
\newcommand{\F}{\mathbb{F}}
\newcommand{\sL}{\mathscr{L}}
\newcommand{\sM}{\mathscr{M}}
\newcommand{\sT}{\mathscr{T}}
\newcommand{\scal}[2]{\left\langle #1, #2 \right\rangle}
\DeclareMathOperator{\vol}{Vol}
\DeclareMathOperator{\spa}{span}
\DeclareMathOperator{\Var}{Var}
\DeclareMathOperator{\rank}{rank}
\DeclareMathOperator{\im}{im}

\newcommand{\red}{\color{red}}

\usepackage[paper=a4paper, left=1.2in, right=1.2in, top=1.2in, bottom=1.2in]{geometry}
\usepackage{setspace}\onehalfspace


\begin{document}

%\thispagestyle{empty}

\vspace{-2em}

\noindent
\makebox[0pt][l]{\textbf{21-623 Complex Analysis}}%
\makebox[\textwidth][r]{\textbf{TT}}\\
\makebox[\textwidth][c]{\textbf{Homework 1 (due 21th Jan 2026)}}%

\marginpar{{\red \emph{Do N O T} \\ \emph{distribute}}}

\vspace{1em}




%\bigskip\bigskip





\begin{enumerate}[label=\textbf{\arabic*.}, leftmargin=*]


\item
Let $f\colon \C \to \C$, $f(z) = \sqrt{|\text{Re}(z)\text{Im}(z)|}$.

(i) Show that $f$ is \emph{not} holomorphic at $0$.
(ii) Show that $f$ satisfies the Cauchy-Riemann equations at $0$.

Why does this not contradict the theorem from class?


\item
Suppose a series of complex numbers $\sum_{n=0}^\infty a_n$ converges. Show that then the radius of convergence of the power series $\sum_{n=0}^\infty a_nz^n$ is not smaller than $1$. Moreover, show that for every region in the unit disc of the form $S_M = \{z \in \C, \ |z| < 1, \ |1-z| \leq M(1-|z|)\}$, $M > 0$, we have
\[
\lim_{S_M \ni z \to 1} \sum_{n=0}^\infty a_nz^n = \sum_{n=0}^\infty a_n
\]
(Abel's theorem).



\item
We have said that the convergence of power series on the boundary of their discs of convergence can be sometimes subtle. Here are some basic examples -- prove the following statements.

(a) The power series $\sum_{n=0}^\infty nz^n$ does not converge on any point of the unit circle.

(b) The power series $\sum_{n=1}^\infty \frac{1}{n^2}z^n$ converges at every point of the unit circle.

(c) The power series $\sum_{n=1}^\infty \frac{1}{n}z^n$ converges at every point of the unit circle except $z=1$.


\item
A subset $S$ of the set of positive integers is called an arithmetic progression of step $r$ if $S = \{a,a+r,a+2r,a+3r,\dots\}$ for some positive integers $a$ and $r$. Show that the set of positive integers cannot be partitioned into a finite number of subsets which are arithmetic progressions of distinct steps (excluding the trivial partition into just one set with $a = r = 1$).

\emph{Hints.} There is an intimate connection between integers and analysis: generating functions! First convince yourself that if such a partition was possible, we would have $\frac{z}{1-z} = \sum_{j=1}^k \frac{z^{a_j}}{1-z^{d_j}}$  with, say $d_1 < d_2 < \dots < d_k$ (for all $|z|<1$). Then consider this identity around the point $z_0 = e^{2\pi i/d_k}$.

\end{enumerate}

\end{document}
