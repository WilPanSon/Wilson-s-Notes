\documentclass[a4paper]{article}
\input{notestyle}

\newtheorem{thm}{Theorem}[section]
\newtheorem{lem}[thm]{Lemma}
\newtheorem{defn}[thm]{Definition}
\newtheorem{eg}[thm]{Example}
\newtheorem{ex}[thm]{Exercise}
\newtheorem{conj}[thm]{Conjecture}
\newtheorem{cor}[thm]{Corollary}
\newtheorem{claim}[thm]{Claim}
\newtheorem{rmk}[thm]{Remark}
\newtheorem{rec}[thm]{Recall}
\newtheorem{prop}[thm]{Proposition}

\title{Graduate Discrete Math (21-701) Notes}
\author{Wilson Pan}
\date{\today}

\begin{document}
\maketitle

\begin{abstract}
Lecture notes based on Graduate Discrete Math (21-701) 

\end{abstract}

\section{Graphs}
\begin{defn}
    Graph is a set of objects $(V,E)$ and $E\subseteq {V \choose 2}$
\end{defn}
\begin{defn}
    Walk is a sequence of vertices
\end{defn}
\begin{defn}
    A path is a walk without repeated vertices
\end{defn}
\begin{defn}
    A proper K-coloring of a graph is a function $c:V\to [k]$ such that $\forall u,v\in V, u \sim v\implies c(u)\neq c(v)$
\end{defn}
\begin{thm}
    A graph is 2 colorable if and only if there is no odd cycles in $G$
\end{thm}
\begin{proof}($\implies$) AFSOC there exist an odd cycle, $C$ in $G.$ Define the vertices of $C$ as $v_1,v_2,...,v_k$ where $k$ is odd. Define $c(v)=\begin{cases} 
    \text{red} & d(v,v_1) \text{ is even } \\ \text{blue} & d(v,v_1) \text{ is odd}
\end{cases}$\\
Then $c(v_1)$ and $c(v_k)$ are both red so a contradiction.
\\
($\impliedby$) We can assume each component is connected. Choose $v_0$ and define $c(v)=\begin{cases} 
    \text{red} & d(v,v_0) \text{ is even } \\ \text{blue} & d(v,v_0) \text{ is odd} 
\end{cases}$\\ 
If there exist vertices $u,v$ with $uv$ an edge such that $d(u,v_0)\equiv d(v,v_0) \text{ (mod 2)}$ then consider the cycle, C formed by shortest path from $v_0\to u$ and $v_0\to v$ with $uv.$ Then $|C|=d(u,v_0)+d(v,v_0)+1$ is odd and we're done.
\end{proof}

\section{Hypergraphs}
\begin{defn}
A collection $\mathcal{H}$ of subsets of a vertex set $V$.
\end{defn}
\begin{defn}
    $\mathcal{H}$ is k-uniform if $|f|=k, \forall f\in \mathcal{H}$
\end{defn}
\begin{defn}
    A proper k-coloring of $\mathcal{H}$ is an assignment $c:V\to [k]$ such that $\forall f\in \mathcal{H}, |c(f)|=k$
\end{defn}
\begin{defn}
    A rainbow coloring of $\mathcal{H}$ is an assignment $c:V\to [k]$ $\forall f\in \mathcal{H}, |c(f)|=|f|$
\end{defn}

\begin{eg}
    What is the least number of edges in a k-uniform graph that is not $2$-colorable?
\end{eg}
Let this number be $m(k)$ then $m(1)=0, m(2)=3,m(3)\geq 7$
\begin{thm}
    If $\mathcal{H}$ is a $3$-uniform hypergraph with less than $6$ edges then $\mathcal{H}$ is $2$-colorable
    \begin{proof}
        Using induction on $|V|$\\
        (Base Case) For $n=6$, consider all balanced $2$-colorings of $V$ there are ${6 \choose 3}=20.$ Each hyperedge is incompatible with 2 of those colorings (namely those were the edges are 3 blue or 3 red). Thus, at least $20-12>0$ of these colorings can be proper.\\
        (Induction Hypothesis) Suppose $n\geq 7$
        \begin{claim}
            There are 2 vertices $u$ and $v$ not in any common edge.
        \end{claim}
        \noindent Each edge connects ${3 \choose 2}=3$ pairs of vertices. There are ${7 \choose 2}=21$ pairs of vertices overall. So some pair of vertices is not connected as $21>18.$\\\\
        Define $\mathcal{H}'$ by merging $u,v$ into $w$
        \begin{claim}
            $\mathcal{H}'$ is $3$-uniform 
        \end{claim}
        \noindent Because no edge contains both $u$ and $v$ the merging doesn't create a $2$ set and every edge is still has size $3.$\\\\
        Additionally, $||\mathcal{H}'||\leq ||\mathcal{H}||\leq 6$ so by induction hypothesis $\mathcal{H}'$ is $2$-colorable. Giving both $u$ and $v$ the same color as $w$ and keeping the rest of the colors the same. \\

        \noindent If an edge of $e$ of $\mathcal{H}$ avoids $\{u,v\}$ then it is properly colored in $\mathcal{H}'.$ If $e$ contains $u$ or $v$ then after merging it corresponds to an edge of $\mathcal{H}'$ containing $w.$ If $e$ is monochromatic in $\mathcal{H}$ then it would be monochromatic in $\mathcal{H}'.$ This would be a contradiction so edge is monochromatic in $\mathcal{H}$ and thus a proper coloring. 
    \end{proof}
    \begin{rmk}
            Suppose it has $7$ edges and vertices. Consider the coloring $4$ red and $3$ blue. Then there are ${7 \choose 3}=35$ such colorings. If $\mathcal{H}$ is not $2$ colorable then there are ${3 \choose 3}+{4 \choose 3}=5$ excluded coloring for all distinct edges. There are $4$ forbidden configurations for any configurations that are not $2$ colorable $\mathcal{H}$ with $|\mathcal{H}|=7$ on $7$ vertices
    \end{rmk}
\section{Probabilistic Method}
\begin{thm}
    $m(k)\geq2^{k-1}$
    \begin{proof}
        Color vertices of $\mathcal{H}$ randomly red or blue. For each edge $f$, define $E_f$ to be the event that $f$ is monochromatic then $Pr[E_f]=\frac{1}{2^{k-1}}$
        $$Pr\left [\bigcup_{f\in \mathcal{H}}E_f \right ]\leq  \sum_{f\in \mathcal{H}}Pr\left [E_f \right ]=\frac{|\mathcal{H}|}{2^{k-1}}<1$$ 
        So there is non-zero probability that there exist a coloring with no monochromatic edges if $|\mathcal{H}|<2^{k-1}$
    \end{proof}
\end{thm}
\end{thm}
\begin{thm} 
    Erdős-Selfridge Theorem:
    Given hypergraph $\mathcal{H}$, consider a game between a maker and breaker. The maker's goal is to color some edge all blue and breaker's goal is to prevent all blue edges. \\\\
    If $\mathcal{H}$ is $k$-uniform and $|\mathcal{H}|<2^{k-1}$ then the breaker has a winning strategy even as player $2.$
    \begin{proof}
        Let $\phi(f)=\begin{cases}
            0 & \text{if blocked by breaker}\\
            \frac{2^{\text{\#blue}\in f}}{2^n} & \text{otherwise} 
        \end{cases}$\\
        be the "danger function". Define $$\phi(\mathcal{H})=\sum_{f\in \mathcal{H}}\phi(f)$$
        Observe that if an edge is all blue, then $\phi(\mathcal{H})\geq 1$\\
        At start of the game $\phi(\mathcal{H})=\frac{|\mathcal{H}|}{2^n}.$ The worst case for when maker moves is increasing by $\frac{|\mathcal{H}|}{2^n}$ when the chosen vertex is in all edges. Then when breaker moves, $$-\sum_{f \ni v_1}\phi (f).$$
        When maker goes after $$\sum_{f \ni v_2}\phi (f).$$
        Notice $$\sum_{f \ni v_1}\phi (f)>\sum_{f \ni v_2}\phi (f)$$ otherwise breaker played optimally. \\
        So as long as $\frac{|\mathcal{H}|}{2^{n-1}}<1$ then breaker wins.
    \end{proof}
\end{thm}
\begin{defn}
    Incidence matrix of a hypergraph $\mathcal{H}$ with $|V|=n$ and $|\mathcal{H}|=m$ is defined as $$I_{i,j}=\begin{cases}
        1 & \text{if } v_n\in f_m \\ 0 & \text{otherwise}
    \end{cases}$$
\end{defn}
\begin{thm}
    Hall's Theorem\\ If $G$ is a bipartite on $(A, B)$ there is a complete matching if and only if$$\forall S\subseteq A,|\Gamma(S)|\geq |S|$$ where $\Gamma(S)=\{u\in B|\exists v\in S, u \sim v\}.$
\end{thm}
\begin{thm}
    Consider complete graph $\mathcal{P}(X)$ where $|X|=n.$ \\$\mathcal{P}(X)$ has levels $${X \choose 0}, {X \choose 1},\cdots, {X \choose n}$$
    $\forall k<\frac{n}{2},$ there is an injection $$f_k:{X \choose k}\to {X \choose k+1}$$ such that $\forall S\in {X \choose k}, S\subseteq f_k(S)$
    \begin{proof}
        Consider bipartite graph $\left ({X \choose k}, {X \choose k+1} \right ),$ if $f\in {X \choose k}, g\in {X \choose k+1}$ then we define $f\sim g$ if $f\subseteq g.$ Then for some $S\subseteq {X \choose k}$ then $|\Gamma(S)|\geq \frac{|S|(n-k)}{k+1}.$ 
    \end{proof}
\end{thm}
\begin{defn}
    For a sperner system is a hypergraph $\mathcal{H}$ that satisfy if $$\forall f,g\in \mathcal{H}, f\not \subseteq g$$
\end{defn}
\begin{thm}
    If $\mathcal{H}$ is a sperner system of $n$-vertices then $$|\mathcal{H}|\leq {n \choose \floor{\frac{n}{2}}}$$
\end{thm}
\begin{thm}
    LYM Inequality on a sperner family $\mathcal{H}$, $$\sum_{f\in \mathcal{H}}\frac{1}{{n \choose |f|}}\leq 1$$
    \begin{proof}
        Suppose $F=\{\emptyset, \{1\},...,\{1,2,...,n\}\}$\\
        Note: Any sperner family can share at most one edge with $F.$\\
        Consider a random permutation $\sigma\in S_n$ and define $\mathcal{H}_\sigma=\{\sigma(f)|f\in \mathcal{H}\}.$\\
        For any $\sigma\in S_n, |\mathcal{H}_\sigma\cap F|\leq 1$\\
        Now choose any $\sigma,$ uniformly at random and define $\mathcal{X}=|\mathcal{H}_\sigma\cap F|$ is a random variable and $\mathcal{X}\leq 1.$\\
        Let $\mathcal{X}=I_f$ where $I_f=\begin{cases}
            1 & \sigma(f)\in F \\ 0 &\text{otherwise}
        \end{cases}$
        $$\mathbb{E}[\mathcal{X}]=\sum_{f\in \mathcal{H}}\mathbb{E}[I_f]=\sum_{f\in \mathcal{H}}\frac{1}{{n \choose |f|}}\leq 1$$
    \end{proof}
\end{thm}
\begin{defn}
    Define the "shadow" of $\mathcal{H}\subseteq { X \choose r}$ as $\partial \mathcal{H}\subseteq {X \choose r-1}$ $$\partial \mathcal{H}=\left \{S\subseteq {X \choose r-1}: \exists T\in \mathcal{H}, S\subseteq T\right \}$$
\end{defn}
\begin{thm}
    Let $n=|X|$ then $$\frac{|\mathcal{H}|}{{n \choose r}}\leq \frac{|\partial \mathcal{H}|}{{n \choose r-1}}$$ with equality only if $\mathcal{H}$ is empty on ${X \choose r}$
    \begin{proof}
        Suppose $\mathcal{\HG}$ is a sperner system, not all on one level.\\
        Write $\HG_i=\HG\cap {X \choose i}$ then $H=\HG_i\cup \HG_{i+1}\cup\cdots\cup \HG_j$ where $i<j$ and $\HG_i$ nonempty.\\
        We can instead of $\HG_j$ we can write $\partial \HG_j$ as $\partial \HG_j\subseteq \HG_j$\\
        Suppose $\HG$ maximizes the sum $\sum_{f\in \HG}\frac{1}{{n \choose |f|}}$ among all sperner graphs. \\
        Let $S\in \partial \HG$ and $T\subseteq \HG$. Define a bipartite graph from $S\to T$ and edges if $S\subseteq T.$ \\
        For $T\in \HG,deg(T)=r$ for $S\in \partial \HG,deg(S)=n-(r-1).$\\ 
        So $|\HG|\cdot r=b\leq |\partial \HG|\cdot(n-r+1)$.\\
        Then $|\HG|\cdot r {n \choose r}\leq |\partial \HG|\cdot(n-r+1) {n \choose r}\implies \frac{|\HG|}{{n \choose r}}\leq \frac{|\partial \HG}{{n \choose r-1}}$ 
    \end{proof}
\end{thm}
\begin{defn}
    An intersecting hypergraph has any $2$ hyperedges intersect.
\end{defn}
\begin{thm}
    For an intersecting hypergraph on $n$-vertices and $r$-uniform, \begin{enumerate}
        \item[] If $r=\frac{n}{2}$ then we can fix $1$ vertex and complete the remaining $\frac{n}{2}-1$ vertices. So ${n-1 \choose \frac{n}{2}-1}$
        \item[] If $r>\frac{n}{2}$ then $2^n$. 
        \item[] If $r<\frac{n}{2}$ then ${n-1 \choose r-1}.$
    \end{enumerate}
\end{thm}
We'll prove the last statement
\begin{proof}
    Assume $n=lk$ for some $l$ and for any $\sigma\in S_n$ define $\HG_\sigma=\{\sigma(f) | f\in \HG\}.$ Define $\mathcal{F}$ to be a $k$-uniform hypergraph with $l$ non-intersecting edges.\\ 
    If $\HG$ is intersecting then $|\HG_\sigma\cap \mathcal{F}|\leq 1.$ \\
    Let $\mathcal{X}=|\HG_\sigma\cap \mathcal{F}|.$ Then $$\mathbb{E}[\mathcal{X}]=\sum_f\mathbb{E}[I_f]=|\HG|\mathbb{P}[\sigma(f)\in \mathcal{F}]=|\HG|\frac{l}{{n \choose k}}=\frac{|\HG|}{{n-1 \choose k-1}}$$
    So $|\HG|\leq {n-1 \choose k-1}$\\\\
    Consider the case when $n$ is not divisible by $k.$ If $n\geq 2k$ then fix a cyclic ordering $\pi$ of the $n$ vertices. For that ordering consider the $n$ cyclic $k$-intervals for $i=1,2,...,n$  $$I_i(\pi)=\{\pi(i),\pi(i+1),...,\pi(i+k-1)\}$$
    indices taken modulo $n.$\\
    For a given $\pi$ define $$\mathcal{X}_\pi:=\#\{f\in \HG | f \text{ is one of the intervals } I_i(\pi)\}$$
    Any two sets counted in $\mathcal{X}_\pi$ must intersect since $\HG$ is intersecting. Among the $n$ cyclic $k$-intervals at most $k$ of them can be pairwise intersecting since we can fix one vertex however $k+1$ intervals will force two of them to be disjoint. So for every $\pi, \mathcal{X}_\pi\leq k.$\\
    So $$\mathbb{E}[\mathcal{X}_\pi]=|\HG|\frac{k!(n-k)!}{(n-1)!}=|\HG|\frac{n}{{n \choose k}}\leq k\implies |\HG|\leq {n-1 \choose k-1}$$
\end{proof}
\noindent We'll try constructing such a configuration.\\
If $|\mathcal{H}|={n-1 \choose k-1}$ then $|\mathcal{H}_\sigma\cap \mathcal{F}|=k$ for each $\sigma\in S_n$. Then there is an $i$ such that $\mathcal{I}=\begin{cases}
    \{i-k+1, i-k+2,...,i\} \\ \{i-k+2,i-k+3,...,i+1\} \\ \vdots \\ \{i,i+1,...,i+k+1\}
\end{cases}$ \\
Suppose $a_1,...,a_{k-1}\in [n]$ with no $a_j=i-k,i-k-1,...,i,...,i-k+1$\\
Consider a permutation $\sigma$ sending $a_1\to i+1,a_2\to i+2,...,a_{k-1}\to i+k-1$ and fixing $i-k,...,i.$\\
We know, $|\HG_\sigma\cap \mathcal{F}|$ includes all edges of $\mathcal{I}$\\
Now let $\sigma$ be any permutation such that $\HG\cap \mathcal{F}$ includes $i$ of $I.$ It suffices to show $\HG_\sigma\cap \mathcal{F}$ includes all of $\mathcal{I}$ for any transposition.\\
\textbf{Lemma:} Adjacent transposition generates $S_n.$\\
(Case 1) If $j,j+1\in \{i-k+1,1,...,i+k-1\},$ neither is $i$ so they're both on same side of $i.$\\
Letting $f_0=\{i-k+1,...,i\}$ and $f_1=\{i,...,i+k-1\}$ then $\tau(f_0)=f_0$ and $\tau(f_1)=f_1$\\
(Case 2) If $j=i+k-1$ and $j+1=i+k$ then $\tau(f_0)=f_0$ and $\tau(\{i-k,...,i-1\})=\{i-k,...,i-1\}.$
\begin{thm}
    Let $\alpha_1,...,\alpha_n\sim Ber(p),$ choosing numbers $\beta_1,...,\beta_n$ with $\sum\beta_i=1$ then $\mathbb{P}[\sum\beta_i\alpha_i\geq \frac{1}{2}]\geq p$
    \begin{proof} Define $\HG$ on $[n]$ by $f\in \HG$ if $\sum_{i\in f}a_i\geq \frac{1}{2}$
    For simplicity, assume no sum is $\frac{1}{2}.$
        Then $$\mathbb{P}\left [\sum\beta_i\alpha_i\geq \frac{1}{2}\right ]=\sum_{f\in \HG}p^{|f|}(1-p)^{n-|f|}$$
        Define $h_k=\left |\HG\cap {X \choose k}\right |$
        \begin{align*}
            \mathbb{P}\left [\sum\beta_i\alpha_i\geq \frac{1}{2}\right ]&=\sum_kh_kp^k(1-p)^{n-k}\\
            &=\sum_{k\leq \frac{n}{2}}h_kp^k(1-p)^{n-k}+h_{n-k}p^{n-k}(1-p)^k\\
            &=\sum_{k\leq \frac{n}{2}}(h_k+h_n)p^{n-k}(1-p)^{k}-h_{k}\left (p^{n-k}(1-p)^k-p^k(1-p)^{n-k}\right )\\
            &\text{Note: $h_k+h_{n-k}\geq {n \choose k}$ since it or its complement has to be in $\HG$}\\
            &\geq \sum_{k\leq \frac{n}{2}}{n \choose k}p^{n-k}(1-p)^{k}-h_{k}\left (p^{n-k}(1-p)^k-p^k(1-p)^{n-k}\right )\\
            &=\sum_{k\leq \frac{n}{2}}{n \choose k}p^{n-k}(1-p)^{k}+{n-1 \choose k-1}p^k(1-p)^{n-k}\\
            &=p\sum_{k\leq \frac{n}{2}}{n-1 \choose k-1}p^k(1-p)^{n-k}\\
            &=p\sum_{k\leq \frac{n}{2}}{n-1 \choose k-1}p^k(1-p)^{n-k}\\
            &=p\sum_{k\leq \frac{n}{2}}{n-1 \choose k-1}p^{k-1}(1-p)^{(n-k)-(k-1)}\\
            &=p
        \end{align*}
    \end{proof}
\end{thm}
\begin{thm}
    If there are $10$ points in the plane then they can be covered by $10$ non-intersecting unit circles.
    \begin{proof}
        Given any collection $X\subseteq \mathbb{R}^2,|x|=10$. Consider a random translation of the hexagonal circle pattern.\\
        Let $\mathcal{Z}=\#$ points in $X$ covered then $$\mathbb{E}[\mathcal{Z}]=\mathbb{E}[I_1]+\cdots+\mathbb{E}[I_{10}]=10\cdot \frac{\pi}{\frac{6}{\sqrt{3}}}\approx 9.07$$
        So there exist a translation such that $\mathcal{Z}=10$
    \end{proof}
\end{thm}
\begin{thm}
    Given a graph, $G$ on $n$ vertices and $\frac{nd}{2}$ edges, $d\geq 1.$ Then $\alpha(G)\geq \frac{n}{2d}.$
    \begin{proof}
        Let $S\subseteq V$ be a random subset defined by $\mathbb{P}[v\in S]=p,$ $p$ to be determined. Let $X=|S|$ and $Y=\mathbb{E}[G_{|S}].$ For each $e=\{i,j\}\in E$, let $Y_e$ be indicator random variable for the event $i,j\in S$ so that $$Y=\sum_{e\in E}Y_e$$ For any such $e$, $$\mathbb{E}[Y_e]=\mathbb{P}[i,j\in S]=p^2$$
        $$\mathbb{E}[Y]=\frac{nd}{2}p^2$$
        Clearly, $\mathbb{E}[X]=np$ so $\mathbb{E}[X-Y]=np-\frac{nd}{2}p^2.$ \\
        Setting $p=\frac{1}{d}$ then $\mathbb{E}[X-Y]=\frac{n}{2d}.$ \\
        So there exist a $S$ such that the number of vertices minus the number of edges is at least $\frac{n}{2d}.$\\
        Create $S^*$ from $S$ by deleting one vertex from each edge in $S$ and delete it and this leaves $S^*$ with at least $\frac{n}{2d}$ vertices. With all edges destroyed we leave $S^*$ an independent set.
    \end{proof}
\end{thm}
\begin{thm}
    Erdos Chromatic Number Girth Theorem\\
    $\forall k\in \mathbb{Z}^+, \exists $ graph of girth greater than or equal to $k$ and chromatic number $k.$
    \begin{proof}
        Idea: Choose random graph $G\sim G(n,p).$ To show a graph satisfies both properties we need the the number of short cycles (length less than $k$) to be 0 and there are no independent set of size no more than $\frac{n}{k}.$ \\
        For the first statement let $X=\#$cycles with length $\leq k$ then $$\mathbb{E}[X]=\sum_C\mathbb{E}[I_c]=\sum_{j=3}^{k}\sum_{|C|=j}\mathbb{E}[I_C]=\sum_{j=3}^k {n \choose j}\frac{(j-1)!}{2}p^j\leq \sum_{j=3}^kn^jp^j\leq (np)^{k+1}$$
        To have $\mathbb{E}[X]=O(1),$ we need $p=O\left (\frac{1}{n}\right )$ \\\\
        We want no independent set of size $a\approx \frac{n}{k}$ so
        \begin{align*}
            \mathbb{P}[\alpha(G_{n,p})&\geq a]\leq {n \choose a} (1-p)^{a \choose 2}\\
            &\leq n^a(1-p)^{a(a-1)/2}\\
            &\leq n^ae^{-pa(a-1)/2}\\
            &=\left (ne^{-p(a-1)/2} \right )^a\\
            &=\left (e^{\ln(n)-p(a-1)/2} \right )^a 
        \end{align*}
        Not possible since we need $p\geq 5\ln(n)/a$ but $p=O\left (\frac{1}{n} \right )$ from previous condition.\\\\
        To fix this issue consider an alteration.
        If $p=\frac{n^\epsilon}{n}$ and $0<\epsilon<\frac{1}{k}$ then we have $\mathbb{P}(\alpha(G_{n,p})\geq \frac{n}{2k})\to 0$ since $\frac{n^\epsilon}{n}>>\frac{5\ln(n)-2k}{n}.$\\
        To fix the short cycle issue,$$\mathbb{E}[X]=\sum_{j=3}^{k}(np)^j\leq (k-3)(np)^k\leq kn^{\epsilon k}$$
        By Markov's inequality $$\mathbb{P}\left (X\geq \frac{n}{2}\right )\leq \frac{kn^{\epsilon k}}{n/2}$$
        Choose $n$ large enough such that both probabilities are greater than $\frac{1}{2}.$ Then there exists a graph on $n$ vertices with no independent set of size $\frac{n}{2k}$ and less than $\frac{n}{2}$ short cycles. Delete an vertex from each short cycle to make a graph $G'$ with $\frac{n}{2}$ vertices, no short cycles and no independent set of size $\frac{n}{2k}.$ So $$\mathcal{X}(G')=\frac{n'}{\alpha(G')}\geq \frac{\frac{n}{2}}{\frac{n}{2k}}\geq k.$$
    \end{proof}
\end{thm}
\noindent When does $G_{n,p}$ have triangles? \\
If $X=\#$triangles then $$\mathbb{E}[X]=\sum_{\text{Triangle }T\in K_n}E[I_T]={n \choose 3}p^3\sim n^3p^3/6=O(1)$$ if $p=O \left ( \frac{1}{n} \right )$\\
Is $G_{n,p}$ connected for $p=\frac{c}{n}$?\\
Let $X=\#$spanning trees of $G_{n,p}$ then $$\mathbb{E}[X]=\sum_{T\in G_{n,p}}E(I_T)=n^{n-2}p^{n-1}=n^{n-2}\frac{c^{n-1}}{n^{n-1}}=\frac{c^{n-1}}{n}\to \infty.$$
Let $Y=\#$isolated vertices then $$\mathbb{E}[Y]=\sum_{v\in V}\mathbb{P}(v \text{ isolated})=n(1-p)^{n-1}\approx ne^{-p(n-1)}\approx ne^{-c}$$
For $p=O\left (\frac{1}{n} \right ),$ let $X=\#$triangles. Then $$\mathbb{E}[X]={n \choose 3}p^3\to 0.$$
So $\mathbb{P}(G_{n,p} \text{ having triangles})\to 0.$ \\\\
\begin{thm}
    Threshold of $\mathcal{H}$ in $G_{n,p}.$\\
    Consider $G_{n,p}$ with $p=p(n)$ and $\mathcal{H}$ is fixed graph with $k$ vertices and $l$ edges.\\
        Define $\epsilon = \epsilon(\mathcal{H})=\frac{l}{k}$ and $\epsilon'=\epsilon'(\mathcal{H})=\max_{J\subseteq \mathcal{H}}\epsilon(J).$
        If $p^\epsilon \cdot n\to 0$ then $\mathbb{E}[\#\mathcal{H} \text{ in } G_{n,p}]\to 0.$
        
    \begin{proof}
        $\mathbb{E}[\#\mathcal{H}\in G_{n,p}]\leq {n \choose k}hp^l\leq C(np^\epsilon)^k$ where $h=\frac{k!}{\text{Aut}(\mathcal{H})}$\\
        If $p^{\epsilon'}n\to 0,$ there is some argument for densest subgraph $J.$\end{proof}
        \noindent Now to show the other side, if $p^{\epsilon'}n\to \infty$ (if $p=\omega\left (\frac{1}{n^{1/\epsilon'}} \right )$ then $G_{n,p}$ has $\mathcal{H}$ as a subgraph with high probability. 
        \begin{proof}
            Let $X=\#\mathcal{H}$ subgraph in $G_{n,p}$ then 
            by Chebyshev, $$\mathbb{P}[X\leq 0]\leq \frac{\text{Var}[X]}{\mathbb{E}[X]^2}.$$
            We can compute $$\mathbb{E}[X^2]=\sum_{H_1,H_2\in \mathcal{H}}\mathbb{P}[H_1,H_2\subseteq G_{n,p}]=\sum_{t=0}^{k}\sum_{|H_1\cap H_2|=t}\mathbb{P}(H_1,H_2\subseteq G_{n,p})$$
            For $t=0$ we have $$\sum_{|H_1\cap H_2|=0}\mathbb{P}(H_1\subseteq G_{n,p})\mathbb{P}(H_2\subseteq G_{n,p})\leq \mathbb{E}[X]^2$$
            So $$\mathbb{E}[X^2]-\mathbb{E}[X]^2\leq \sum_{t=1}^{k}\sum_{|H_1\cap H_2|=t}\mathbb{P}(H_1,H_2\subseteq G)=\sum_{t=1}^{k}\sum_{|H_1\cap H_2|=t}{k \choose t}{n-k \choose k-t}hp^{e(H_1\cup H_2)}$$
            By PIE, $e(H_1\cup H_2)\geq 2l-\epsilon't$ since $e(H_1\cap H_2)\leq \epsilon't$ as $H_1\cap H_2$ is a subgraph of $H_1$
            \begin{align*}
                \sum_{t=1}^{k}\sum_{|H_1\cap H_2|=t}{k \choose t}{n-k \choose k-t}hp^{e(H_1\cup H_2)}&\leq \sum_{t=1}^{k}\sum_{|H_1\cap H_2|t}{k \choose t}{n-k \choose k-t}hp^{2l-\epsilon't}\\
                &=\sum_{t=1}^{k}\sum_{|H_1\cap H_2|t}{n \choose k}h\cdot h{n-k \choose k-t}p^{l-\epsilon't}\tag{$\sum_{H\in \mathcal{H}}p^l=\mathbb{E}[X]={n \choose k}\cdot h \cdot p^l$}\\ 
                &\leq\mathbb{E}[X] \sum_{t=1}^{k}h{k \choose t}{n-k \choose k-t}p^{l-\epsilon't}\\
                &\leq \mathbb{E}[X]\sum_{t=1}^k h\cdot C \cdot n^k \cdot \frac{1}{n^t}\cdot p^{l-\epsilon't}\\
                &\leq \mathbb{E}[X]\sum_{t=1}^k C' \cdot h \cdot {n \choose k}\cdot p^l \cdot \frac{1}{\left ( np^{\epsilon'}\right )^t }\\
                &=\mathbb{E}[X]^2\sum_{t=1}^kC'\left (\frac{1}{np^{\epsilon'}} \right )^t\to 0
            \end{align*}
        \end{proof}
\end{thm}
\begin{thm}
    Chernoff Bound \\
    Suppose you had independent random values $\zeta_1,...,\zeta_n$ with $\zeta_i \in \{-1, 1\}$ $\forall i $ and $\mathbb{P}(\zeta_i=1)=\mathbb{P}(\zeta=-1)=\frac{1}{2}$. \\Let $X=\sum_{i=1}^{n}\zeta_i$
    \begin{align*}
        \mathbb{P}(X>a)&=\mathbb{P}(e^{tX}> e^{ta})\\
        &\leq\frac{\mathbb{E}(e^{tX})}{e^{ta}}\\
        &=\frac{\mathbb{E}(e^{t\sum{\zeta_i}})}{e^{ta}}\\
        &= \frac{\prod_{i=1}^{n}\mathbb{E}[e^{t\zeta_i}]}{e^{ta}}\\
        &=\frac{\left (\frac{e^t+e^{-t}}{2} \right )^n}{e^{ta}}\\
        &\leq e^{nt^2/2-ta}\\
        &=e^{a^2/2n-a^2/n} \tag{for $t=\frac{a}{n}$}\\
        &=e^{-\frac{a^2}{2n}}
    \end{align*}
\end{thm}
 \noindent \textbf{Question: }How many vectors can I have in $\mathbb{R}^d$ all at a common pairwise angle.\\
 Equivalently, $\exists \alpha, v_i\cdot v_j=\begin{cases}
     \alpha & \text{if } i \neq j \\ 1 & \text{otherwise}
 \end{cases}$\\
\noindent Let $V=\begin{bmatrix}
     | & | &  &|\\
     v_1 & v_2 & \cdots & v_m \\
     | & | &  &|
 \end{bmatrix}$\\
 Then $V^TV=G=\begin{bmatrix}
     v_1\cdot v_1 & v_1 \cdot v_2 & \cdots & v_1\cdot v_m \\
     \vdots & v_2\cdot v_2 & \cdots \\
     \vdots & \vdots & \ddots
 \end{bmatrix}$\\\\
 Claim: $Rank(G)\leq d$. \\For each $\Vec{x}\in \text{Ker}(V)\implies V\Vec{x}=0\implies V^TV\Vec{x}=0\implies G\Vec{x}=0$ \\
We know $Rank(V)\leq d$ so $Rank(G)=Rank(V^TV)\leq d.$\\\\
We can write $G=\alpha J_m+(1-\alpha)I_m$ then $J_m$ has eigenvalues $m$ with multiplicity at least $1$ and eigenvectors with multiplicity $\geq m-1.$
$$\begin{bmatrix}1\\ -1 \\ 0 \\ \vdots\end{bmatrix}, \begin{bmatrix}1\\ 0 \\ -1 \\ \vdots\end{bmatrix}\cdots \begin{bmatrix}1\\ 0 \\ \vdots\\ -1\end{bmatrix}
$$
If $\Vec{x}$ is an eigenvector of $J_m$ with eigenvalue $\lambda$ then $G\Vec{x}=\alpha \lambda x+(1-\alpha)x=(\alpha\lambda +1-\alpha)x.$\\
$G$ has eigenvalues $1-\alpha$ with multiplicity at least $m-1.$\\
So $Rank(G)\geq n-1$ and consequently $d\geq n-1.$\\\\
\textbf{Question: } How many unit vectors $v_1,..,v_n$ can I have in $\mathbb{R}^d$ with all pairs approx equal angle.\\
Equivalently, $\exists \alpha$ such that $(*) \text{ }v_i\cdot v_j\in (\alpha - \epsilon, \alpha + \epsilon$ if $i\neq j$ and $1$ if $i=j.$
\begin{thm}
    $\forall \epsilon\in (0,1),$ there is a $c>0$ such that for all sufficiently large $d$ there is a collection $v_1,...,v_m\in \mathbb{R}^d$ of vectors satisfying $(*)$ with $m\geq 2^{cd}$
\end{thm}
\noindent Consider the easier problem when $\alpha=0.$ Choose $m$ vectors $v_i\in \{1,-1\}^d$ at random with $\mathbb{P}(v_i^j=+1)=\mathbb{P}(v_i^j=-1)=\frac{1}{2}.$
We know $v_i\cdot v_i=n$ then let $u_i=\frac{v_i}{\sqrt{n}}.$
Then $\mathbb{E}(u_i\cdot u_v)=0$ as $\zeta_{i,j,k}=\begin{cases}
    1 & \text{if } v_i^k\cdot v_j^k=1 \\ -1 & \text{if } v_i^k\cdot v_j^k=-1  
\end{cases}$
By Chernoff,
$$\mathbb{P}(v_i\cdot v_j \notin (-\epsilon n, \epsilon n))\leq 2e^{-\epsilon^2n/2}$$
$$\mathbb{P}(\exists i,j \text{ s.t. } u_i\cdot u_j\notin (-\epsilon n, \epsilon n ))\leq {m \choose 2}2e^{-\epsilon^2n/2}\leq m^2e^{-\epsilon^2n/2}=m^2\beta^{-n}$$
\begin{thm}
    If $\zeta_1,\zeta_2,...,\zeta_n$ are independent random variables with $\mathbb{E}(\zeta_i)=0,|\zeta_i|\leq 1,X=\sum\zeta_i$ then $$\mathbb{P}(X\geq a)\leq e^{-\frac{a^2}{2n}}$$
\end{thm}
\noindent \textbf{Question}: I flip biased coins with head prob $=\frac{1}{3}$ n times. Bound the probability $\#$ head $\geq \frac{n}{2}$. \\
Define $\zeta_i=\begin{cases}
    1 & \text{if head} \\ \frac{-p}{1-p} & \text{otherwise}
\end{cases}$ (Subtract expected value and dividing by $1-p$)
Then $\mathbb{E}[\zeta_i)=0,|\zeta_i|\leq 1$ for $p<\frac{1}{2}.$\\
Let $X=h+(n-h)\frac{-p}{1-p}$
\begin{thm} For random variables $X,Y$
\begin{align*}
    \mathbb{E}[XY]&=\sum_{x,y}xy\mathbb{P}(X=x,Y=y)\\
    &=\sum_{x,y}xy\mathbb{P}(Y=y)\mathbb{P}(X=x|Y=y) \\
    &=\sum_{y}y\mathbb{P}(Y=y)\sum_xx\mathbb{P}(X=x|Y=y)\\
    &=\mathbb{E}(Y\mathbb{E}(X|Y))
\end{align*}
\end{thm}
\begin{thm}
    Let $X$ be a random variable and $A$ an event ten $$\mathbb{E}(X | A)=\frac{1}{\mathbb{P}(A)}\sum_{w\in A}p(w)\mathcal{X}(w)$$
\end{thm}
\begin{thm}
    Let $\zeta_1,...,\zeta_n$ be random variable with $\mathbb{E}(\zeta_i)=0$ and $|\zeta_i|\leq 1.$
    $$\mathbb{E}(e^{t\sum\zeta_i})\leq \left (\frac{e^t+e^{-t}}{2} \right )^n$$
    \begin{proof}
        
    $$\mathbb{E}(e^{t\sum\zeta_i})= \mathbb{E}\left (\prod_{i=1}^n e^{t\zeta_i} \right )= \mathbb{E}\left (\prod_{i=1}^{n-1} e^{t\zeta_i}\mathbb{E}\left (e^{t\zeta_n } | \prod_{i=1}^{n-1}e^{t\zeta_i}\right )  \right )$$
    We know want to upper bound $\mathbb{E}\left (e^{t\zeta_n } | \prod_{i=1}^{n-1}e^{t\zeta_i}\right ).$\\
    By convexity, $e^{t\zeta_i}\leq h(\zeta_i)=\frac{1}{2}\left [(1-\zeta_i)e^{-t}+(1+\zeta_i)e^t\right ]\implies \mathbb{E}\left [e^{t\zeta_i} \right ]\leq \frac{e^t+e^{-t}}{2}$
    $$\mathbb{E}\left (e^{t\zeta_n } | \prod_{i=1}^{n-1}e^{t\zeta_i}\right )\leq \left ( \frac{e^t+e^{-t}}{2}\right ) \mathbb{E}\left (\prod_{i=1}^{n-1}e^{t\zeta_i} \right )\leq \left (\frac{e^t+e^{-t}}{2} \right )^n $$
    \end{proof}
\end{thm}
\begin{defn}
    Random variables $X_0,X_1,...$ is a martingale if it satisfies the following properties
    \begin{enumerate}
        \item $\mathbb{E}[|X_i|]<\infty$
        \item $\mathbb{E}[X_{i+1}|X_1,...,X_i]=X_i$
    \end{enumerate}
\end{defn}
\begin{thm}
    Azuma's Theorem: If $X_0,X_1,...$ is martingale and \\$|X_{i+1}-X_i|\leq 1$ then $$\mathbb{P}(X_n-X_0\geq \lambda \sqrt{n})\leq e^{-\lambda^2/2}$$
    \begin{proof}
        By Markov's Inequality we have $$\mathbb{P}(X_n-X_0\geq \lambda \sqrt{n})\leq \frac{\mathbb{E}\left [e^{t(X_n-X_0)}\right ]}{e^{t\lambda\sqrt{n}}}$$
        We can telescope the numerator as 
        \begin{align*}
            \mathbb{E}\left [e^{t(X_n-X_0)}\right ]&=\mathbb{E}\left [\prod_{k=1}^{n}e^{t(X_k-X_{k-1})} \right ]\\&=\mathbb{E}\left [\mathbb{E}\left [e^{t(X_n-X_{n-1})}|X_0,X_1,X_2,...,X_{n-1} \right ]\prod_{k=1}^{n-1}e^{t(X_k-X_{k-1})} \right ]
        \end{align*}
        By Theorem 3.23 we have $\mathbb{E}\left [e^{t(X_n-X_{n-1})}|X_1,X_2,...X_{n-1}\right ]\leq e^{t^2/2}$ so $$\mathbb{E}\left [\mathbb{E}\left [e^{t(X_n-X_{n-1})}|X_0,X_1,X_2,...,X_{n-1} \right ]\prod_{k=1}^{n}e^{t(X_k-X_{k-1})} \right ]\leq e^{nt^2/2}$$
        So $$\mathbb{P}(X_n-X_0\geq \lambda \sqrt{n})\leq e^{nt^2/2-t\lambda \sqrt{n}}.$$
        The RHS achieves it's max at $t=\frac{\lambda}{\sqrt{n}}$. 
        Thus $$\mathbb{P}(X_n-X_0\geq \lambda \sqrt{n})\leq e^{-\lambda^2/2}$$
    \end{proof}
\end{thm}
\begin{defn}
    Doob's Martingale: Let $X$ and $Y_1,Y_2,...$ be random variables with $\mathbb{E}[|Y_i|]<\infty.$\\Define $X_i:=\mathbb{E}(X|Y_1,...,Y_i)$ for $i\geq 1.$ with $X_0=\mathbb{E}[X]$
\end{defn}
\begin{thm}
    McDiarmid's Inequality: Let $f:\mathcal{Y}_1\times \mathcal{Y}_2\times \cdots \times \mathcal{Y}_n\to \mathbb{R}$ that is $1$-lipschitz. \\If $Y_1,Y_2,...,Y_n$ are independent random variables where $Y_i\in \mathcal{Y}_i.$ \\For $X:=f(Y_1,...,Y_n)$ satisfies $$\mathbb{P}(X-\mathbb{E}(X)\geq \lambda \sqrt{n})\leq e^{-2\lambda^2}.$$ 
    
\end{thm}
\begin{eg}
    For $m$ balls into $n$ bins, let $X=\#$ empty bins then $$\mathbb{E}[X]=n\left (1-\frac{1}{n}\right )^m\approx ne^{-\alpha}$$ where $m=\alpha n.$ If we let $Y_i=$ pos of ball $i$ then $f(Y_1,...,Y_m)=\#$ empty bins. So $\mathbb{P}(X\geq \mathbb{E}[X]+\lambda \sqrt{n})\leq e^{-2\lambda^2}$
\end{eg}
\begin{thm}
    $$\mathbb{P}(A | B \cap C)=\frac{\mathbb{P}(A\cap B | C)}{\mathbb{P}(B | C)}$$
\end{thm}
\begin{thm}
    Lovasz Local Lemma: Given a collection $\mathcal{B}=\{B_1,B_2,...,B_n\}$ of bad events. If $B_1,...,B_n$ have a dependency graph of max degree $d$ and $4pd\leq 1$ where $\mathbb{P}(B_i)\leq p$ then $$\mathbb{P}\left (\bigcap \overline{B_i} \right )>0.$$
    \begin{proof}
        We'll prove the statement by induction on $k=|S|$ for $S\subseteq [n].$ Assume $4pd\leq 1$ then we want to show $\forall i, \mathbb{P}\left (B_i | \bigcap _{j\in S}\overline{B_j} \right ) \leq 2p.$\\\\
        Let $S=T\cup U$ where $T=\{j\in S | j\sim i\}.$\\
        \begin{align*}
            \mathbb{P}\left (B_i | \bigcap_{j\in T}\overline{B_j} \cap \bigcap_{j\in U}\overline{B_j}\right )&=\frac{\mathbb{P}\left (B_i  \cap \bigcap_{j\in T}\overline{B_j} | \bigcap_{j\in U}\overline{B_j}\right )}{\mathbb{P}\left (\bigcap_{j\in T}\overline{B_j}|\cap_{j\in U}\overline{B_j}\right )}\\
            &\leq \frac{p}{1-\mathbb{P}\left (\bigcup_{j\in T}B_j|\cap_{j\in U}\overline{B_j}\right )}\\
            &\leq \frac{p}{1-\sum_{j \in T}\mathbb{P}(B_j | \bigcap_{k\in U}\overline{B_k})}\\
            &\leq \frac{p}{1-2dp} \tag{Induction Hypothesis}\\
            &\leq 2p \tag{Assumption}
        \end{align*}
        We're done with the induction. \\
        Then $$\mathbb{P}\left (\bigcap_{i=1}^n\overline{B_i} \right)=\mathbb{P}\left (\overline{B_n}|\bigcap _{i=1}^{n-1}\overline{B_i} \right )\cdot \mathbb{P}\left (\bigcap _{i=1}^{n-1}\overline{B_i} \right )\geq \frac{(1-2p)^n}{p}>0   $$
    \end{proof}
\end{thm}
\begin{thm}
    Local Lemma (General Version) Suppose $\mathcal{B}$ is a collection of "bad" events with some dependency graph. Suppose we can assign real number $0<X_A<1$ to each $A\in B$ such that $$\mathbb{P}(A)\leq X_A\prod_{B\sim A}(1-X_B).$$ Then $$\mathbb{P}\left (\bigcap_{B\in \mathcal{B}}\overline{B} \right )\geq \prod_{B\in \mathcal{B}}(1-X_B)> 0$$
    \begin{proof}
        We can prove it by induction on $|S|$ that if $B_1,...,B_t,B_{t+1},...,B_S\in \mathcal{B}$ such that $A\sim B_1,...,B_t$ and $A\not \sim B_{t+1},...,B_{S}$ to show $$\mathbb{P}\left (A | \bigcap_{i=1}^S \overline{B_i}\right )\leq X_A$$ 
        We have \begin{align*}
            \mathbb{P}\left (A | \bigcap_{i=1}^t \overline{B_i}\cap \bigcap_{i=t+1}^T \overline{B_i}  \right ) &=\frac{\mathbb{P}\left (A \cap \bigcap_{i=1}^T \overline{B_i} | \bigcap_{i=t+1}^S \overline{B_i} \right )}{\mathbb{P}(\bigcap_{i=1}^t \overline{B_i} | \bigcap_{i=t+1}^S \overline{B_i})}\\
            &\leq \frac{\mathbb{P}\left (A| \bigcap_{i=t+1}^S \overline{B_i} \right )}{\mathbb{P}(\bigcap_{i=1}^t \overline{B_i} | \bigcap_{i=t+1}^S \overline{B_i})}\\
            &\leq \frac{X_A\prod_{B\sim A}(1-X_B)}{\prod_{i=1}^t(1-X_B)}\\
            &\leq X_A
        \end{align*}
        Thus we're done with induction.
        Using the statement \begin{align*}
            \mathbb{P}\left (\bigcap_{i=1}^S \overline{B_i}\right )=\mathbb{P}\left ( \overline{B_1} |\overline{B_2}\right )\times \cdots \times \mathbb{P}\left ( \overline{B_n} | \bigcap_{i=1}^S \overline{B_i}\right ) \geq \prod_{i=1}^n (1-X_{B_i})>0   
        \end{align*}
    \end{proof}
 \end{thm}
 \begin{thm}
     Axel's Theorem: \\ $\forall \epsilon > N_\epsilon$ and an infinite binary sequence such that $\forall n>N_\epsilon,$ any 2 consecutive block of length $n$ differ in $\geq \left (\frac{1}{2}-\epsilon \right )n$ places. 
     \begin{proof}
         Let the bad events be $B_{i,n}$ where for each $i$, intervals $[i,...,i+n],[i+n+1,...,i+2n]$ differ by less than $\left (\frac{1}{2}-\epsilon \right )n$ \\
         Let $X=\#$ places where they differ then $\mathbb{E}[X]=\frac{n}{2}$.\\
         By Chernoff-Hoeffding's Lemma we have $$\mathbb{P}(X-\mathbb{E}[X]\geq -\epsilon n)=\mathbb{P}(X\geq n/2-\epsilon n)\leq e^{-2\epsilon^2 n}\leq e^{-\epsilon^2 n /10}$$
         Let $X_{B_{i,n}}=e^{-\epsilon^2 n /20}\cdot \frac{1}{n^3}$ then fix $B_{i_0, n_0}$ then we have 
         \begin{align*}
             X_{B_{i_0,n_0}}=e^{-\epsilon^2 n_0 /20}\cdot \frac{1}{n_0^3}
         \end{align*}
         \begin{align*}
             e^{-\epsilon^2 n_0 /20}\cdot \frac{1}{n_0^3}\prod_{n=N_\epsilon}^T\prod_{i=i_0-2n}^{i_0+2n}\left (1- e^{-\epsilon^2 n /20}\cdot \frac{1}{n^3}\right )^{2n+2n_0}\geq
         \end{align*}
         % e^{-\epsilon^2n_0/20}\cdot \frac{1}{n_0^3}\left (1-\sum_{n=N_\epsilon}^T(2n+2n_0+1)e^{-\epsilon^2n/20}\frac{1}{n^3} \right )^{2n} 
     \end{proof}
\end{thm}
\noindent From Homework \#3
\begin{thm}
    $\forall \epsilon > 0, \exists N_\epsilon,  \forall T, \exists $ a binary sequence of length $T$ such that \\
    $(*)$ $\forall n> N_\epsilon$ identical blocks of length $n$ are separated by distance $\geq (2-\epsilon)^n$
\end{thm}
\begin{thm}
    Konig's Infinity Lemma: Let $G$ be a connected, locally finite, infinite graph then $G$ contains an infinite path.
\end{thm}
\begin{thm}
    $\forall \epsilon>0, \exists N_\epsilon, \forall T, \exists$ an infinite binary sequence such that vertices of my tree are finite binary sequences with property $(*).$\\
    Let $\mathcal{T}$ be a complete tree of all binary sequences with all vertices and join $S\to S'$ if $S$ can be obtained from $S'$ by removing last digit of $S'.$\\
    We want to show $\mathcal{T}$ is locally finite and infinite. It is locally finite as each node has at most $2$ children. It is infinite because for any string that satisfy $(*),$ any of its prefix has to satisfy $(*).$ By Theorem 3.34, there must be an infinite path with property $(*)$
\end{thm}
\section{Topology}
\begin{defn}
    A topology is a set $X$ and a collection $\mathcal{O}$ of open sets satisfying
    \begin{enumerate}
        \item $\emptyset\in \mathcal{O}, X\in \mathcal{O}$
        \item $\mathcal{O}$ is closed under finite intersection
        \item $\mathcal{O}$ is closed under arbitrary union
    \end{enumerate}
    A collection of basic open sets are closed under finite intersections. 
\end{defn}
\begin{defn}
    $X$ is compact if every cover has a finite subcover
\end{defn}
\begin{defn}
    Product topogy is $$\prod_{\alpha\in A}\mathcal{O}_\alpha$$ where $\mathcal{O}_\alpha\subseteq X_\alpha$ is open and $\mathcal{O}_\alpha=X_\alpha$ except for finitely many.
\end{defn}
\begin{thm}
    If $X_\alpha$ where $\alpha\in I$ are compact topological spaces then $\prod_{\alpha\in I}X_\alpha$ is compact.
\end{thm}
\section{Ramsey Numbers}
\begin{defn}
    Ramsey number $R(k,l)=\min_{n\geq 1}\{K_n \text{ contains a red $K_k$ or blue $K_l$}\}$\\
    We can see $R(3,3)=6$
\end{defn}
\begin{thm}
    $R(k,l)\leq R(k-1,l)+R(k,l-1)$
    \begin{proof}
        Let $n\geq R(k-1,l)+R(k,l-1)$ and consider a red/blue coloring of $K_n.$ Fix $v_0$. Since $v_0$ has $\geq R(k-1,l)+R(k,l-1)-1$ edges,\\
        (Case 1) If $v_0$ has $\geq R(k-1,l)$ red edges then the induced subgraph of the neighbors, $G'$ must have red $K_{k-1}$ or blue $K_{l}.$ If red $K_{k-1}$ then $G'\cup v_0$ is a $K_k$, otherwise we have blue $K_{l}.$ \\
        (Case 2) If $v_0$ has $\geq R(k,l-1)$ blue neighbors then same argument as case 1.\\
        Thus we're done
    \end{proof}
\end{thm}
\begin{thm}
    $R(k,k)>(1-O(1))\frac{k}{e\sqrt{2}}2^{k/2}$
    \begin{proof}
        Flip fair coins to color a $K_n$ red or blue. Let $X=\#$ monotonic $K_k$ then $$\mathbb{E}[X]={n \choose k}2^{1-{k \choose 2}}\leq \left (\frac{en}{k} \right )^k\cdot 2\cdot 2^{-k(k-1)/2}$$
        If $\mathbb{E}[X]<1$ then $R(k,k)>n$.
        \begin{align*}
            2^{1/k}\left (\frac{en}{k}\right )2^{(k-1)/2}&<1
        \end{align*}
        Thus $$n<(1-O(1)\frac{k}{e\sqrt{2}}2^{k/2}$$
        Consequently, $R(k,k)>(1-O(1)\frac{k}{e\sqrt{2}}2^{k/2}$
    \end{proof}
\end{thm}
\begin{thm}
    Alterations: Color edges of $K_n$ randomly red or blue. Delete an vertex from each monochromatic $K_k$. Let $X=n-$ \# monochromatic cliques.\\
    \begin{align*}
        \mathbb{E}[X]&=n-{n \choose k}2^{1-{k \choose 2}}\\
        &\geq n-\left (\frac{en}{k} \right )^k\cdot 2\cdot 2^{-k(k+1)/2}\\
        &=n-2\left (\frac{en}{k}\cdot 2^{\frac{-k-1}{2}} \right )
    \end{align*}
    Let $n=\frac{k}{e}\cdot 2^{k/2}$ then $$\frac{k}{e}\cdot 2^{k/2}-2^{k/2}=(1-o(1))\cdot\frac{k}{e}\cdot 2^{k/2}$$
\end{thm}
\begin{thm}
    Using Lovasc Local Lemma: Given $k$, fix $n$, randomly red/blue color edges.
    \begin{proof}
        Bad events: $B_k$ for $k\in \mathcal{K}$ where $\mathcal{K}$ is the collection of $k$-clique. \\
        Then $\mathbb{P}(B_k)=2^{1-{n \choose k}}.$\\
        If $K_1,K_2$ share any edges, set $B_{K_1}\sim B_{K_2}$ in dependency graph. Then $$D\leq {k \choose 2}{n \choose k-2}$$
        Consequently 
        \begin{align*}
            epD&\leq e\cdot 2^{1-{n \choose k}}\left (2 {k \choose 2}{n \choose k-2}\right )<1 \\
            4e\left (\left (\frac{en}{k-2} \right )^{k-2}{k \choose 2}  \right ) &<2^{k \choose 2} \\
            \left (2e{k \choose 2} \right )^{\frac{1}{k-2}}\cdot {\frac{en}{k-2}}&<2^{{k \choose 2}-\frac{1}{k-2}}=2^{\frac{k+1}{2}}\\
            (1+o(1))\frac{en}{k-2}&<2^{\frac{k+1}{2}}
        \end{align*}
        So $$n<(1-o(1))\frac{k\sqrt{2}}{e}2^{k/2}$$
        Thus $R(k)>(1-o(1))\frac{k\sqrt{2}}{e}2^{k/2}$
    \end{proof}  
\end{thm}
\begin{defn}
    Define $K_k^j$ as the complete $j$ uniform hypergraph on $n$ vertices with $k$ vertices
\end{defn}
\begin{defn}
    Define $R_j(k)=$ minimum $n$ such that any red/blue coloring of ${[n] \choose j}$ has a monochromatic $K_k^j$ 
\end{defn}
\begin{thm}
    $R_r(k,l)\leq R_{r-1}(R_r(k-1,l),R_r(k,l-1))$
    \begin{proof}
        Let $N=R_{r-1}(R_r(k-1,l),R_r(k,l-1))+1$ and fix $v$. There are $N+1$ other vertices, $Y$. Each edge containing $v$ includes an $r-1$ edge in $Y.$ Let it inherit the color of the $r$ edges.\\
        (Case 1) We have $R_r(k-1,l)$ vertices in $Y$ such that all $r-1$ subsets are red. 
        (Case 1A)
    \end{proof}
\end{thm}
\noindent Let $C(k)=$ minimum $n$ such that $\forall X \subseteq \mathbb{R}^2$ such that $|X|=n$ and $X$ has a subset $S$ where $|S|=k$ and $S$ is in convex position.\\
Then $C(1)=1, C(2)=2, C(3)=3, C(4)=5,C(5)=9,...$
\begin{thm}
    $C(k)\leq R_4(5,k)$
\end{thm}
\noindent\textbf{Lemma:} If $S\subseteq \mathbb{R}^2$ is $k$-points in general position such that any $4$ of them are in convex position, then they all are. (Easy to see by triangulation)  \\\\
Given a set $F$ of four points color $F$ color $F$ red if not in convex position and blue otherwise.\\
Note: $K_k^n$ is a complete $k$-uniform hypergraph on $n$ vertices.\\
A red $K_4^5$ is impossible as $C(4)\leq 5$. Since we can find a blue $K_4^k$ and we win by lemma. \\
Color $3$-tuples according to whether "sorted slopes" are increasing or decreasing. If $n\geq R_3(k,k),$ I can find $k$ vertices all of whose $3$ tuples are caps or all $3$ tuples are cups. 
\begin{defn}
    Let $CC(k,l)=$ min $n$ such that any $n$ pts in general position, no two have same $x$ coordinate, have a $k$-cup or an $l$-cap. 
\end{defn}
\begin{thm}
    Erdos Szekeres: $CC(k,l)={k + l -4 \choose k-2}+1$
    \begin{proof}
        I have $k$-cup or a $l$-cup. We'll show it by induction on $k+l.$ Assume no $l$-cap. I do have a $(k-1)$-cup or $l$ cup by induction. If I delete the last point of each $(k-1)$-cup. Then only ${k+1-5 \choose k-3}$ points remain. So I deleted ${k_l-4 \choose k-2}+1-{k+l-5 \choose k-2}+1.$ 
    \end{proof}
\end{thm}
\begin{thm}
    For all positive integers $k$ and $r$, there exists $N$ such that any
$r$-coloring of the numbers $1, 2, . . . , N$ has a monochromatic $k$-term arithmetic progression.
\end{thm}
\begin{thm}
    If $\mathbb{N}$ is partitioned into 2 sets , one contains arbitrary long arithmetic progression.\\
    \textbf{Statement 1:} $\forall k, \exists N$ such that any 2 coloring of $[N]$ has a monochromatic $k$-term arithmetic progression. If such a statement is false for $k_0,$ then for all $n$ there is a coloring of $[n]$ with no $k_0$ arithmetic progression. With Konig's Lemma there exists a coloring of $\mathbb{N}$ with no $k_0$-term arithmetic progression.\\\\
    Statement 1 implies \\
    \textbf{Statement 2} $\forall r, \forall k, \exists W(k,r)$ such that any $r$-coloring of $[N]$ for $N\geq W(k,r)$ admits a $k$-term monochromatic A.P. 
 \end{thm}
 \noindent Some values of $W(k,r)$ are $W(k;1)=k$, $W(2,r)=r+1$, $W(2,2)=3$ and $W(3,2)=9.$
 \begin{thm}
     $W(3,2)\leq 325$\\
     Note: The technique used here can be used for the general case.
     \begin{proof}
         Consider $65$ blocks of $5$ spots each. Within the first 33 blocks, there must be 2 blocks of the same coloring. Let the blocks be $b_1,b_2\in [33]$. Of the first block consider the first 3 spots then if it's same color then we're done, WLOG for $a_1,a_2\in [3]$ with $a_1<a_2$ say $5b_1+a_1,5b_1+a_2$ be red. Let $a_3=2a_2-a_1\in [5].$ If $7b_1+a_3$ is red then we're done as $7b_1+a_1,7b_1+a_2,7b_1+a_3$ is a mono A.P. So say $7b_1+a_3$ is blue.\\
         Since $b_2$ is the same coloring then let $b_3=2b_2-b_1\in [65]$. If $7b_3+a_3$ is red then we have $7b_1+a_1,7b_2+a_2,7b_3+a_3$. Otherwise if blue we have $7b_1+a_3,7b_2+a_3,7b_3+a_3$.\\
         Thus we're done and $W(3,2)\geq 65\cdot 5=325$
     \end{proof}
 \end{thm}
\begin{defn}
    $WF(k,l,r)=$ minimum $N$ such that any $r$-coloring of $[N]$ admits $l$ color focused $k$-term A.P or a $k+1$ term $A.P.$
\end{defn}
\begin{thm}
    $$WF(2,2,r)\leq (2r^{2r+1}+1)(2r+1)$$
    $$WF(2,3,r)\leq (2r^{2r^{2r+1}+1}+1)(2r^{2r+1}+1)(2r+1)$$
\end{thm}
\begin{defn}
    Hales-Jewett: $\forall r, \forall n,\exists d$ such that in any $r$-coloring of $[n]^d$ hypercube, there is a monochromatic line.
\end{defn}
\begin{defn}
    A combinatorial line is a set of points represented by a string in $([n]\cup \{x\})^d\setminus [n]^d.$ The points of the line are obtained by substituting $x=1,2,...,n.$
\end{defn}
\begin{defn}
    A geometric line $([n]\cup \{x,\overline{x}\})^d\setminus [n]^d$ obtained by substituting in $x=1,...,n$ and $\overline{x}=n-x+1.$
\end{defn}
\noindent Given an A.P-free coloring of $[N]$ want to give a line free coloring of $[N]^d.$ Define $\phi:[n]^d\to (n-1)d$ by $\phi(a_0,a_1,...,a_{d-1})= a_0+a_1+\cdots +a_{d-1}.$ Then we have $$HJ(2,r)\leq d\iff 2^d<r\iff HJ(2,r)\leq \log_2r$$
$HJ^c(2,r)=r$ as if we take any of $(0,...,0),(1,0,...,0),...$ there are $d+1>r\implies$ a monochromatic combinatorial line. \\
For $HJ(3,2)$, take $p\in [3]^d$\\
\section*{Additive Combinatorics}
\begin{defn}
    \begin{align*}
        A+A=\{a+a' | a,a'\in A\} && A\cdot A = \{a\cdot a'| a,a'\in A\}
    \end{align*}
\end{defn}
\begin{thm}
    $\max(|A+A|,|A\cdot A|)\geq |A|^{1+\epsilon}$
\end{thm}
\noindent Suppose we have a set $A$, $X=A+A, Y=A\cdot A$. Let $\mathcal{P}=X\times Y=(A+A)\times (A\cdot A)$. Let $\mathcal{L}=\{\{y | y=a(x-a')\} | a,a'\in A\}$. Then $|\mathcal{L}|=|A|^2.$ \\\\
Define $i(\mathcal{L}, \mathcal{P})$ to be the number of incidences between the points and lines in $\mathcal{P}$ and $\mathcal{L}$.  \\
For any line containing $a, a'\in A$, the equation is $y=a(x-a')$. For a point $p=(a'+a'',a\cdot a'')$ we have $a'+a''\in A+A$ and $a\cdot a''\in A\cdot A.$ $$i(\mathcal{L}, \mathcal{P})\geq |\mathcal{L}|\cdot |A|=|A|^3$$
Then $$i(\mathcal{L}, \mathcal{P})=O(|\mathcal{L}|^{2/3}|\mathcal{P}|^{2/3}+|\mathcal{L}|+|\mathcal{P}|).$$
So $|A|^3\leq i(\mathcal{L}, \mathcal{P})\leq C(|A|^{4/3}(|A+A|\cdot |A\cdot A|^{2/3})$ as $|\mathcal{L}|+|\mathcal{P}|=O(|\mathcal{L}|^{2/3}|\mathcal{P}|^{2/3})$.
We also have $|A|^2\leq |\mathcal{P}|\leq |A|^4$ and $|\mathcal{P}|^{1/2}\leq |\mathcal
L|\leq |\mathcal{P}|$. So $C|A|^{5/2}\leq |A\cdot A||A+A|$ and consequently $$\max(|A+A|,|A\cdot A|)\geq \epsilon |A|^{5/4}$$
\subsection*{Planar Graphs}
\begin{thm}
    Euler's Formula for Planar Graphs: $$|V|-|E|+|F|=2$$
\end{thm}
\begin{thm}
    Suppose $G$ is a connected planar graph with $m\geq 3.$ Then $$m\leq 3n-6$$
    \begin{proof}
        Consider the bipartite graph of $E(G)$ and $|F(G)|.$ For each edge there is at most $2$ faces and each face is closed by at least $3$ edges. So $$2|E|\leq \sum \deg(e)=\sum\deg(f)\geq |F|\cdot 3$$
        So $n-m+f=2\implies n-m+\frac{2}{3}m\geq 2\implies n-\frac{1}{3}m\geq 2$ 
    \end{proof}
\end{thm}
\begin{defn}
    Let $Cr(G)$ is the minimum number of crossing in any drawing.
\end{defn}
\noindent Given $G$, if $e(G)\geq 3n$, $G$ is not planar $Cr(G)\geq m-3n$ since at least $m-3n$ edges must be removed to make $G$ planar. \\
Consider $G$ with $G_p=$graph where each vertex stays with probability $p.$ 
Then $\mathbb{E}(np)=pn$ and $\mathbb{E}(mp)=p^2m$. Then $$p^4Cr(G)\geq \mathbb{E}(Cr(G_p))\geq \mathbb{E}(m_p-3n_p)=\mathbb{E}(m_p)-3\mathbb{E}(n_p)\geq p^2m-3pn$$
So $Cr(G)\geq \frac{m}{p^2}-\frac{3n}{p^3}$ and is maximized when $p=\frac{4n}{m}$ only when $4n\leq m.$ Then $\frac{m}{p^2}-\frac{3n}{p^3}=\frac{m^3}{64n^2}.$ \\
\begin{thm}
    For any collection $\mathcal{L}$ of lines in $\mathbb{R}^3$, there are at most $O(|\mathcal{L}|^{3/2})$ joints. 
\end{thm}
\noindent We just have to show the following lemma to imply the theorem.
\begin{lem}
    In any collection of lines with $|J|$ joints, there exist some line in $\leq 3|J|^{1/3}$ joints. 
 \end{lem}
\noindent Lemma 5.26 $\implies$ theorem 5.25 as we define $J(L)=$ most joints in $|L|$ lines.
$$J(L)\leq J(L-1)+3J^{1/3}\leq J(L-2)+3(J-1)^{1/3}+3J^{1/3}\leq \cdots$$
So $J\leq 3J^{1/3}L\iff J^{2/3}\leq 3L\iff J\leq \sqrt{27}L^{3/2}$\\
Given an arbitrary field, $\poly{D}{\F^n}$ and $S=\{a_1,...,a_k\}$ for $a_i\in \F^n$. We want to find a nonzero polynomial that vanishes at $J.$\\\\ Let $T:\poly{D}{\F^n}\to \F^k$ defined as $T(p)=\begin{bmatrix}
    p(a_1)\\ \vdots \\ p(a_k)
\end{bmatrix}$\\
By rank nullity theorem, $$\dim(Im(T))\leq k\implies \dim(\ker (T))\geq \dim(\poly{D}{\F^n})-k$$
If $\dim(\poly{D}{\F^n}>k, \exists p\in \poly{D}{\F^n}$ vanishes at $S$, $|S|=k$.\\
Let $$\mathcal{D}=\{x_1^{d_1},x_2^{d_2},...,x_n^{d_n}|\sum d_i\leq D\}$$ This is a basis for $\poly{D}{\F^n}$. By stars and bars we have $|\mathcal{D}|={D+n \choose n}\geq \frac{D^n}{n!}>k.$ 
We need $\frac{D^3}{3!}>J$ so $D>3J^{1/3}.$\\
AFSOC each line has more than $D>3J^{1/3}$ joints.\\
If $p\in \poly{D}\F, \forall a \in \F, \exists c \in \F$ such that $p(x)=(x-a)q(x)+c.$ If $a$ is a root, $p(x)=(x-a)q(x).$ \\A line is a function $\gamma(t)=a+bt$ for $a,b\in \F^n$ then $q(t):=p(\gamma(t))$ is a polynomial in $\poly{D}{\F}$. $\deg(q)$ has to have at most the degree of $p$ so $\deg(q)<D.$ By our assumption the line has more than $D$ joints so $\deg(q)>D.$ The only way $q(t)$ can have more than $D$ roots is if $q$ is the zero polynomial. So we can conclude our polynomial $p$ must be identically $0$ on the union of all lines in $\mathcal{L}$\\\\
Each joint is the intersection of $3$ lines and $p$ is zero on all lines. So the direction derivative along each of the lines is $0$ and as they are linearly independent we have $\nabla p=0$ at every joint. Consider $p_1=\frac{\partial p}{dx}$, $\deg(p_1)\leq D-1$ and $p_1$ vanishes at every joint in $J$ so we contradict the minimality of $D$ so the assumption that all lines have $>D$ joints is false. 

\newpage
\begin{lem} \label{lem:vanishing_poly}
    If $P(x_1,...,x_n)$ is a non-zero polynomial over $\F_q$, with total degree $D\leq q-1$ then $P(x)$ cannot be zero for all $x\in \F_q^n.$
\end{lem}
\begin{proof}
    We can write $$P(x)=\sum_{k=0}^{D}q_k(x_1,...,x_{n-1})x_n^k$$
    We'll show the statement by induction on $n$.\\[1ex]
    \textbf{(Base Case $n=1$)} AFSOC $P(x_1)$ is nonzero and vanishes on all of $\F_q$. Since $\deg(P)\leq q-1$ but $P$ has $q$ distinct roots, $P$ must be the zero polynomial. This is a contradiction. \\[1ex]
    \textbf{(Inductive Step)} AFSOC $P(x)=0$ on all $x\in \F_q^n$. We write $$P(x)=\sum_{k=0}^{D}q_k(x_1,...,x_{n-1})x_n^k$$
    Since $P(x)$ is a nonzero polynomial, there must exist at least one $k$ for which $q_k(x_1,...,x_{n-1})$ is a non-zero polynomial.\\[1ex]
    Fix the first $n-1$ variables. Let $(a_1,...,a_{n-1})$ be an arbitrary point in $\F_q^{n-1}$. Define $$Q(t):=P(a_1,...,a_{n-1},t)$$ 
    We can express $Q(t)=\sum_{k=0}^D q_k(a_1,...,a_{n-1})t^k$. For this fixed $(a_1,...,a_{n-1})$, each $q_k(a_1,...,a_{n-1})$ is a constant in $\F_q$. This implies $\deg(Q)\leq D\leq q-1.$ \\[1ex]
    By assumption, $P(x)=0$ everywhere, so $Q(t)=P(a_1,...,a_{n-1},t)=0$ for all $t \in \F_q$. 
    From our base case, a single-variable polynomial of degree $\leq q-1$ that has $q$ roots must be the zero polynomial. This means all coefficients of $Q(t)$ must be zero. \\[1ex]
    Therefore $q_k(a_1,...,a_{n-1})=0$ for all $k.$ 
    Since $(a_1,...,a_{n-1})$ was arbitrarily chosen, this holds for all points in $\F_q^{n-1}.$\\[1ex]
    This means each $q_k$ is a polynomial in $n-1$ variables that vanishes on all of $\F_q^{n-1}$. The total degree of $P$ is $D = \max_k (\deg(q_k) + k)$, which implies $\deg(q_k) \le D \le q-1$. By our inductive hypothesis, a polynomial in $n-1$ variables of degree $\le q-1$ that vanishes everywhere must be the zero polynomial. \\[1ex]
    Thus, each $q_k$ is the zero polynomial. This implies $P(x)$ is the zero polynomial, which contradicts our initial assumption that $P(x)$ is a non-zero polynomial.
\end{proof}

\begin{thm}
    If $N\subseteq \F_q^n$ is a set with the property that for all $x\in \F_q^n$, there is a line $L_x$ such that $L_x\setminus \{x\}\subseteq N,$ then $|N|\geq \epsilon_nq^n$ where $\epsilon_n>0$ depends only on $n.$ 
    (The proof shows $\epsilon_n = (10n)^{-n}$).
\end{thm}
\begin{proof}
    Assume for the sake of contradiction that $|N| < \left( \frac{q}{10n} \right)^n$.\\[1ex]
    We know from the polynomial method that there exists a non-zero polynomial $p \in \text{Poly}_D(\F_q^n)$ that vanishes on $N$, with degree $D \le 2n|N|^{1/n}$.\\[1ex]
    Using our AFSOC, we can bound this degree $D$:
    $$D \le 2n|N|^{1/n} < 2n \left( \frac{q}{10n} \right) = \frac{q}{5}$$
    So, we have found a non-zero polynomial $p$ with total degree $D < q/5$.\\[1ex]
    Now, consider any arbitrary $x \in \F_q^n$. By the theorem's premise, there is a line $L_x$ through $x$ such that $L_x \setminus \{x\} \subseteq N$. We can parametrize this line as $\gamma(t) = x + d \cdot t$ for $t \in \F_q$, where $d \in \F_q^n \setminus \{0\}$ is a direction vector.
    Note that $\gamma(0) = x$, and $L_x \setminus \{x\} = \{\gamma(t) \mid t \in \F_q \setminus \{0\}\}$.\\[1ex]
    Define a new, single-variable polynomial $R(t) := p(\gamma(t))$.
    The degree of $R(t)$ is at most the total degree of $p$, so $\deg(R) \le D < q/5$.\\[1ex]
    Since $p$ vanishes on $N$, $p$ must vanish on $L_x \setminus \{x\}$. This means $R(t) = p(\gamma(t)) = 0$ for all $t \in \F_q \setminus \{0\}$.
    The set $\F_q \setminus \{0\}$ has $q-1$ elements, so $R(t)$ has $q-1$ distinct roots.\\[1ex]
    We have a polynomial $R(t)$ with $\deg(R) \le D < q/5$. For any $q \ge 3$, we have $q/5 \le q-2$ (since $10 \le 4q$).
    Thus, $R(t)$ is a polynomial with degree strictly less than $q-1$, but it has $q-1$ roots.
    A non-zero polynomial cannot have more roots than its degree. Therefore, $R(t)$ must be the zero polynomial.\\[1ex]
    If $R(t)$ is the zero polynomial, it must be zero for all $t$, including $t=0$.
    $$R(0) = p(\gamma(0)) = p(x) = 0$$
    Since $x \in \F_q^n$ was arbitrary, we have shown $p(x) = 0$ for all $x \in \F_q^n$.
    We also know $\deg(p) = D < q/5$, which implies $\deg(p) \le q-1$.\\[1ex]
    By Lemma \ref{lem:vanishing_poly}, any polynomial with degree $\le q-1$ that vanishes on all of $\F_q^n$ must be the zero polynomial.
    This contradicts our choice of $p$ as a non-zero polynomial.\\[1ex]
    Therefore, our initial assumption was false, and we must have $|N| \ge \left( \frac{q}{10n} \right)^n$.
\end{proof}
\begin{lem}
    If $p\in \text{poly}_{a-1}(\mathbb{F}^n_q)$ is nonzero, $|zero(P)|<q^n.$\\
Maximized when $x_1^{q-1}-1$
\end{lem}
\begin{thm}
    Schartz-Zippel: If nonzero $p\in poly_D(\mathbb{F}^n)$ and $S\subseteq \mathbb{F}$ a finite subset. For random $s_1,...,s_n\in S$ $$\mathbb{P}_{s_1,...,s_n}(p(s_1,...,s_n)=0)\leq \frac{D}{|S|}$$
In other words, $|zero(p)\cap S^n|\leq D|S|^{n-1}$
\end{thm} 
\begin{proof}
    Let $p\in poly_D(\mathbb{F}^n)$ be nonzero. We're done if $n=1$. Do induction on $n.$ $$p(x_1,...,x_n)=\sum_{k=0}^nq_k(x_1,...,x_{n-1})x^k_n.$$ Choose $k_0$to be largest such that $q_{k_0}\neq 0.$ By induction $$\mathbb{P}_{s_1,...,s_{n-1}}(q_{k_0}(s_1,...,s_{n-1})=0)\leq \frac{D-k_0}{|S|}$$
    $$\mathbb{P}_{s_1,...,s_n}(p(s_1,...,s_n)=0)\leq \mathbb{P}_{s_1,...,s_n}(q_{k_0}(s_1,...,s_{n-1})=0)+\mathbb{P}_{s_1,...,s_n}(p(s_1,...,s_n)=0|q_{k_0}(s_1,...s_{n-1})\neq 0)$$
    Note: This is just $\mathbb{P}(B)\leq \mathbb{P}(C)+\mathbb{P}(B|\neg C)\cdot \mathbb{P}(\neg C)$\\
    $q_{k_0}(s_1,...,s_{n-1})\neq 0\implies p(s_1,...,s_{n-1},x_n)$ has degree $k_0.$
    $$\frac{\sum_{s_1,...,s_{n-1}}\prod_{q(s_1,...,s_{n-1})\neq 0}\frac{1}{|S|^{n-1}}\mathbb{P}(p(s_1,...,s_n)=0|q_{k_0}(s_1,...,s_{n-1})\neq 0)}{\sum_{s_1,...,s_{n-1}}\prod_{q(s_1,...,s_{n-1})\neq 0}\frac{1}{|S|^{n-1}}}$$
    $$\mathbb{P}_{s_1,...,s_n}(q_{k_0}(s_1,...,s_{n-1})+\mathbb{P}_{s_1,...,s_n}(p(s_1,...,s_n)=0|q_{k_0}(s_1,...s_{n-1})\neq 0)\leq \frac{D-k_0}{|S|}+\frac{k_0}{|S|}=\frac{D}{|S|}$$
\end{proof}
\begin{thm}
    Extremal Schwartz-Zippel. If $p$ nonzero of degree $d$, $p=0$ on $S^n$ then $|S|\leq d.$
\end{thm}
\begin{eg}
    $$p=\prod_{s\in S}(x_i-s)$$ is $0$ for all $x\in S^n.$ This holds for $S\times \{1\}\times \{1\}\times \cdots.$ 
\end{eg}
\begin{eg}
    $$q=\prod_{a_1\in S_1}(x_1-a_1)\prod _{a_2\in S_2}(x_2-a_2)\prod_{a_3\in S_3}(x_3-a_3)$$
    Say $S_1, S_2,S_3$ has size $4,3,2$, respectively. If in a $5\times4\times3$ box then $q$ is definitely not zero polynomial.
\end{eg}
\begin{thm}
    Combinatorial Nullstellensatz: Suppose $p$ is a nonzero polynomial in $\poly{d}{\F^n}$ of degree $d$ and the monomial $x_1^{j_1}x_2^{j_2}\cdots x_n^{j_n}$ for $j_1+\cdots+j_n=d$ has nonzero coefficients then $\forall S_1,...,S_n\subseteq \F$ with $|S_i|\geq j_i+1$ for all $i,$ $p(x)\neq 0$ for some $x\in S_1\times S_2\times \cdots \times S_n.$
    \begin{proof}
        We'll show it by induction on $d$.\\
        Suppose $f\equiv 0$ on $S_1\times S_2\times\cdots\times S_n$ where $S_i\subseteq \F$.\\
        WLOG $j_i\geq 1$, $\forall s\in S_1$ we have $f(x)=(x_1-s)q_s(x)+r(x)$ then $\deg(q_s)=d-1,$ moreover coefficients of $x_1^{j_1-1}x_2^{j_2}\cdots x_n^{j_n}.$ For any $s\in S_1,s_2\in S_2,...$ we have $f(s,s_2,...,s_n)=0\implies r_s(s,s_2,...,s_n)=0 \space \forall s_2,...,s_n\implies r_s(s_1,...,s_n)=0 \space \space \forall s_1\in S_1,s_2\in S_2,...$  \\
        By our assumption we have $0=f(s_1,...,s_n)=(s_1-s)q_s(s_1,...,s_n)$ for all $s,s_1\in S_1,...,s_n\in S_n$. For $s\neq s_1$ we learn $q_s(s_1,...,s_n)=0.$ Since $q_s$ is zero on $S_1\setminus \{s\} \times S_2\times \cdots \times S_n\implies$ we must have $|S_1\setminus \{s\}|\leq j_1-1$ or for some $i,$ $|S_i|\leq j_i.$ by induction hypothesis.
    \end{proof}
\end{thm}
\begin{eg}
    $\mathcal{X}'(G)$ list chromatic number of $G$. We want to find $\mathcal{X}'(C_{n})$.\\
    Consider an assignment $c_i$ to each vertex $i, c_i\in \N.$ $f(c_1,...,c_n)=(c_2-c_1)(c_3-c_2)\cdots(c_n-c_{n-1})(c_1-c_n).$ The leading term of $f$ is $2c_1c_2\cdots c_n$. For all sets $S_1,...,S_n\subseteq \N,$ chromatic number implies $\exists c_1\in S_1, c_2\in S_2,...,c_n\in S_n$ such that $f(c_1,...,c_n)\neq0.$ 
\end{eg}
\begin{eg}
    Cauchy Davenport: Let $p$ prime, $A,B\subseteq \Z_p$ $$|A+B|\geq \min(p,|A|+|B|-1)$$
    \begin{proof}
        Case 1: If $|A|+|B|-1\geq p$\\
        Consider $x\in \Z_p$ then $|x-A|=|A|$ so $|A-x|+|B|\geq p+1$ and $\exists y\in A_x\cap B\implies \exists a\in A, b\in B$ such that $y=b,y=x-a$ so $x=a+b.$\\\\
        Case 2: If $|A|+|B|-1<p$\\
        Consider any set $C\subseteq \Z$ of size $|C|=|A|+|B|-2$. We want to show $\exists x\in A+B, x\notin C.$ \\
        Define $$f(a,b)=\prod_{c\in C}(a+b-c)$$ We have $\deg(f)=|C|=|A|+|B|-2$, consider the monomial $a^{|A|-1}b^{|B|-1}$ then the coefficient is ${|A|+|B|-2 \choose |A|-1}\neq 0$ in $\Z_p.$ So for $|A|, |B|$ we have a choice of $a,b$ such that $a+b\notin C.$ 
    \end{proof}
\end{eg}
\begin{defn}
    Finite Kakeya: In $\mathbb{F}^n_a,\forall a,\exists b$ such that $\{at+b | t\in \mathbb{F}_a\}\subseteq K.$ Then $K$ is a kakeya set. 
\end{defn}
\begin{thm}
    Chevalley–Warning theorem: \\Let $a=p^l$ for $f_1,...,f_k\in \F_a[x_1,...,x_n]$.\\
    If $\sum_i \deg(f_i)<n$ then the number of common zeros is a multiple of $p$. In particular: if there's $1$ common zero then there is more.
\end{thm}
\begin{eg}
    Given any $n$ numbers $a_1,...,a_n$ there is a nonempty subset that sums to $0$ (mod $n$).
    \begin{proof}
        Let $S_0=\{\}, S_1=\{a_1\},...,S_n=\{a_1,...,a_n\}$ then there exists $i,j$ such that $S_i=S_j$ so 
    \end{proof}
\end{eg}
\begin{thm}
    Erdos-Ginzburg-Ziv Theorem:\\
    How large a collection of numbers do I require to ensure that some $n$-subset sum to a multiple of $n$?\\
    $2n-1$ is enough
    \begin{proof}
        (Main Case) $n=p$ is a prime\\
        Given numbers $a_1,...,a_{2p-1}$, we'll give two polynomials in $2p-1$ variables $x_1,...,x_{p-1}$\\
        We want a polynomial such that $x_i$ behaves like indicators for $a_i\in S.$ So $x_i^{p-1}\equiv 1$ mod $p$ by FLT.
        $$f(x_1,..,x_{2p-1})=\sum_{x_i}x_i^{p-1}=\#\{i | x_i\neq 0\}$$
        $$g(x_1,...,x_{2p-1})=\sum_{x_i}a_ix_i^{p-1}=\sum_{x_i\neq 0}a_i$$
        We have $2p-2<2p-1$ and the trivial solution exist so a non-trivial solution exist by Chevalley-Warning\\
        (General Case) Induction on $n$, $a_1,...,a_{2n-1}$\\
        If $n$ not prime, let $p$ be a prime factor of $n$, $m=\frac{n}{p}$. Find a set $I_i$, $|I_i|=p$ such that $\sum_{j\in I_i}a_j=0$ (mod $p$) for $i\in [2m-1]$\\
        Say $\sum_{i\in I_j}a_i=b_i \equiv 0$ (mod $p$)\\
        Let $c_i=\frac{b_i}{p}$, we can find $c_{i_1},c_{i_2},...,c_{i_m}$ such that $$\sum_{j=1}^{m}c_{i_j}=\sum_{j=1}^{m}\sum_{t\in I_{i_j}}t=\left (\sum_{j=1}^{m}c_{i_j} \right )p\equiv 0$$
    \end{proof}
\end{thm}
\begin{thm}
    There exist an order of at least $d^2$ $2$-distance set in $\R^d$?
\end{thm}
\begin{proof}
    Suppose $S=\{p_1,...,p_m\}$ has just $2$ distances $\alpha$ and $\beta.$
Consider the polymomial, $f\in \R[x_1,...,x_d]$ defined as $$f_i(X)=(||X-p_i||^2-\alpha^2)(||X-p_i||^2-\beta^2)$$
$f_i(X)=\begin{cases}
    \alpha^2\beta^2 & \text{if } X=p_i\\
    0 & \text{otherwise}
\end{cases}$\\
Claim: $f_i$'s are independent. \\
Suppose $\alpha_1f_1+...+\alpha_mf_m=0$, $\forall i$, plug in $p_i$ gives $\alpha_i \alpha^2\beta^2=0\implies \alpha_i=0$\\
Claim: $x_i^{d_i}x_j^{d_j}$ with $d_1+d_2\leq 4$ covers all possible terms. So there is at most $O(d^2)$ possible choices. 
\end{proof}
\begin{eg}
    Eventown where each club has even size and even intersection, $\geq 2^{\ceil{n+\frac{1}{2}}}$
\end{eg}
\begin{eg}
    Oddtown where each club has odd size and even intersection\\
    Let $v_i=$indicence vector of club $i$ in $\F_2$. $v_i\cdot v_j=\begin{cases}
        0 & \text{if } i\neq j \\ 1 & \text{otherwise}
    \end{cases}$\\
    Suppose $\alpha_1\Vec{v_1}+\cdots+\alpha_n\Vec{v_n}=\Vec{0}$\\
    $\Vec{v}_i$ on both sides so $\alpha_i=0$
\end{eg}
\begin{thm}
    Every Polygon has a triangulation
    \begin{proof} 
    Choose a convex vertex of the polygon (a vertex that is a vertex of the convex hull) with neighbors $q,r$. If $\overline{qr}\subset P^\circ$ then we're done. Othewise we can move the point.
    \end{proof}
\end{thm}
\begin{defn}
    We say polygon $P\sim Q$ by scissor congruency if we can cut up $P$ and reassembled to be $Q.$
\end{defn}
\begin{lem}
    (Any rectangle) $\sim$ (Any Unit Size Rectangle)\\(Any triangle) $\sim$ (Unit Side Rectangle)\\(Triangle) $\sim$ (2 Right Triangle)\\ (Right Triangle) $\sim$ (Rectangle)
\end{lem}
\begin{rmk}
    From this lemma we can conclude any polygon is congruent to a rectangle of $1\times d$
\end{rmk}
\begin{thm}
    Are equal-area polyhedra necessarily plane-dissection equivalent? This is not true.
    \begin{eg}
        Unit cube and volume $1$ reg-tetrahedron are not dissection equivalent.\\
    \end{eg}
    \begin{proof}
        Dihedral angle is an irrational multiple of $\pi$\\
        A list of vectors $\Vec{v_1},...$ is independent if for every finite sum, $$\sum_{j=1}^k\alpha_{i_j}v_{i_j}=0 \implies \alpha_{i_j}=0 \space \forall j$$
        $Span(\mathcal{L})$ is the set of vectors representable as finite linear combinations. $\mathcal{L}$ is a basis for $V$ if $\mathcal{L}$ is independent and $Span(\mathcal{L)}=V$
        \begin{lem}
            Zorn's Lemma: $P$ is a poset in which every chain has an upper bound then $P$ has a maximal element.
        \end{lem}
        \noindent Doset is a set of independent sequences, ordered by inclusion. 
        $$\mathcal{L}_1\subseteq \mathcal{L}_2\cdots\subseteq$$ Then $\cup{\mathcal{L}_i}$ is still independent. 
        By Zorn's lemma and Doset we have every vector space has a (possibly infinite) basis.\\\\
        Define $\alpha$ to be dihedral angle of tetrahedron, we'll use that $\frac{\alpha}{\pi}$ is irrational.\\
        In general, we can define a linear transformation $f:\R\to\R$ such that $f(\pi)=0$ and $f(\alpha)=1$.\\
        Since $\alpha$ and $\pi$ are independent over $\Q$ extended to a basis $\alpha, \pi, v_3,v_4,...$\\
        Define $f$ using this basis by defining $f(\alpha)=1,f(\pi)=0$\\\\
        If a plane goes through an angle then $l_e=l_{e_1}=l_{e_2}$ and $\theta_e=\theta_{e_1}+\theta_{e_2}$.\\
        If a plane goes through an edge then $l_{e_1}+l_{e_2}=l_e$ and $\theta_e=\theta_{e_1}=\theta_{e_2}$\\
        If a plane goes through another plane and creates a new edge then $l_{e_1}=l_{e_2}$ and $\theta_{e_1}+\theta_{e_2}=\pi$\\
        We can assign a real number to each polytope by $$\sum_{e\in P}l_e\cdot f(\theta_e)$$
        In the first case we would have $$R(P)=\sum_{e\in P}l_ef(\theta_e)$$
        If $P$ is a cube then $R(P)=0$ as each dihedral angle is $90^\circ$ so it's a rational multiple of $\pi.$ However if $P$ is a tetrahedron has irrational multiple of $\pi$ so $P(R)\neq 0$.
    \end{proof}
\end{thm}
\section*{Linear Algebra}
\begin{defn}
    Adjacency Matrices: On the vertex set $[n] $$$A=\begin{bmatrix}
        && \\\\
    \end{bmatrix}$$
    $A_{i,j}=\begin{cases}
        1 & \text{if } i\sim j \\ 0 & \text{otherwise}
    \end{cases}$
\end{defn}
\noindent If $f:V\to \R$ then $Af=g$ and if $Af=\lambda f$ then it's an eigen function. \\
\begin{rmk}
    We'll denote the $n\times n$ matrix of all ones as $J_n$.
\end{rmk}
\begin{thm}
    $J_n$'s eigenvalues are $\lambda_1=n$ with multiplicity $1$ and $\lambda_2=0$ with multiplicity $n-1.$
    \begin{proof}
        To show $\lambda=0$ has multiplicity $n-1$ the associated eigenvectors are $$\begin{bmatrix}
            1 \\ -1 \\0\\\vdots
        \end{bmatrix}, \begin{bmatrix}
            1 \\ 0 \\-1\\\vdots
        \end{bmatrix},\cdots,$$
    \end{proof}
\end{thm}
\begin{thm}
    $K_n=J_n-I_n$ has eigenvalue $n-1$ with multiplicity $1$ and $-1$ with multiplicity $n-1.$
\end{thm}
\begin{rmk}
    For any regular graph with degree $d$, $d$ is an eigenvalue value with multiplicity $1.$
\end{rmk}
\begin{lem}
    For any adjacency matrix $A,$ $A^2$ has the property that $A^2_{i,j}=\#$walks of length $2$ from $i\to j.$ This generalizes easily to the general case for $A^k$
    \begin{proof}
        This is easily shown from the matrix multiplication $$A_{i,j}=\sum_{k\in [n]}A(i,k)A(k,j)$$
    \end{proof}
\end{lem}
\begin{lem}
    For a $d$-regular graph with diameter $2$ graph then the maximum number of vertices is $1+d+d(d-1)=d^2+1$ vertices.\\
    To achieve this bound, I require $\text{girth}(G)\geq 5$ \\
    Note: Girth is the length of a shortest cycle. \\
    Another way to achieve this bound is the Peterson Graph 
\end{lem}
\begin{lem}
    Any Moore graph (regular graph whose girth is at least twice its diameter) has $A^2+A-(d-1)I=J$. 
    \begin{proof}
        We know $\lambda=d$ an eigenvalue for $f\equiv 1$. By spectral theorem, $A$ has an orthogonal basis of real eigenvectors. Since $f_j\perp f_1$ for $j\neq 1$ then $J\cdot f_j=\Vec{0}$ as $J$ is the matrix of all ones. So $$A^2f_j+Af_j-(d-1)If_i=0\iff \lambda_j^2+\lambda_j-(d-1)=0$$
        We can conclude $\lambda=\frac{-1\pm\sqrt{4d-3}}{2}$\\
        We need $1+m_2+m_3=n$ and $d+m_2\lambda_2+m_3\lambda_3=0$\\
        So $2d-(m_2+m_3)+(m_2-m_3)\sqrt{4d-3}=0$\\
        (Case 1) If $\sqrt{4d+3}\notin \Q$ then $m_2=m_3$ and $2d=d^2\implies d=2$ \\
        (Case 2) If $\sqrt{4d-3}=s\in \Z$ then $d=\frac{s^2+3}{4}$ then $$2d-d^2+(m_2-m_3)s=0\iff 8\left (\frac{s^2+3}{4} \right)-(s^2+3)^2+16(m_2-m_3)s=0$$.\\
        Expanding out we have $as^4+bs^3+cs^2+ds+15=0$ then $s|15\implies s=1,3,5,15\implies d=1,3,7,57.$ 
    \end{proof}
\end{lem}
\noindent We'll be covering graph where for any $2$ vertices $u,v$ there is exactly one common neighbor of $u,v$\\
\begin{thm}
    "There is a politician": $\exists v_0$ such that $\forall u, v_0\sim u$\\
    Note: This doesn't hold for infinite vertices by $H_0=$ 5-cycle and $H_{i+1}$ is $H_i$ with independent path of length 2 added between parts that don't have a common neighbor in $H_i.$\\
    \begin{enumerate}
        \item[Step 1] A counterexample must be regular 
        \item[Step 1A] $u\not \sim v\implies \deg(v)\geq \deg(u)$. By symmetry $\deg(v)=\deg(u)$. This is from $w_1$ being the common neighbor of $u,v$ and $w_2$ being the common neighbor of $w_1,u$ and $z_{1}$ being common neighbor of $w_1$ and $v.$
        \item[Step 1B] Let $\deg(u)=d, \forall v\not= w_i$ we get $\deg=d$ for all $w_2,...,w_d$, we get $\deg(w_i)=\deg(v)=d$. All but $w_1$ are known to be degree $d.$ Since $w_i$ not a politican then $w_1$ must be the politican.
    \end{enumerate}
\end{thm}
\noindent Going back to the graph with diameter $2$ graph then $n=1+d(d-1)=d^2-d+1$. There are exactly 1 path of length 2 between $u,v$ $\deg(u)=d\implies A^2$ is $d$ along diagonal and $1$ everywhere else. $A^2=J+(d-1)I$. $J$ has e.v. $n$ with multiplicity $1$ and $0$ with multipicity $n-1.$ So $A^2$ has e.v. $n+d-1$ with multiplicity $1$ and $d-1$ with multiplicity $n-1.$\\
$A$ has eignevalues $d$ with multiplcitiy $1$, $\sqrt{d-1}$ with multiplicity $s$ and $-\sqrt{d-1}$ with multiplicity $t.$ Also $s+t=n-1.$
The trace of $A$ is $0$ so $d+\sqrt{d-1}s-\sqrt{d-1}t=0\implies d+(s-t)\sqrt{d-1}=0.$ So $\sqrt{d-1}\in \Q\implies h:=\sqrt{d-1}\in \N$. \\
We have $d=\sqrt{d-1}^2+1=h^2+1$. So $d+h(s-t)=0\implies h^2+1=h(t-s)\implies h=1\implies d=2$
\begin{thm}
    Oddtown: Clubs have odd size and intersections are even. The clubs are less than number of people.
\end{thm}
\begin{lem}[Fisher's Inequality]
Let $\mathcal{F} = \{A_1, \dots, A_m\}$ be a family of $m$ distinct subsets of a universe $X$ where $|X| = n$. Suppose there exists a constant $k$ such that $|A_i \cap A_j| = k$ for all $i \neq j$. Furthermore, assume that $|A_i| > k$ for all $i$. Then:
\[
    m \leq n
\]
\end{lem}

\begin{proof}
Let $v_1, \dots, v_m \in \mathbb{R}^n$ be the incidence vectors of the sets $A_1, \dots, A_m$. That is, the $x$-th component of vector $v_i$ is $1$ if $x \in A_i$ and $0$ otherwise.\\
We aim to show that these vectors are linearly independent. Consider a linear combination of these vectors equal to the zero vector, with coefficients $\alpha_1, \dots, \alpha_m \in \mathbb{R}$:
\begin{equation}
    \sum_{i=1}^m \alpha_i v_i = 0
\end{equation}
We take the squared Euclidean norm (the dot product with itself) of both sides:
\[
    \left\langle \sum_{i=1}^m \alpha_i v_i, \sum_{j=1}^m \alpha_j v_j \right\rangle = 0
\]
Expanding using the linearity of the inner product:
\[
    \sum_{i=1}^m \alpha_i^2 \langle v_i, v_i \rangle + \sum_{i \neq j} \alpha_i \alpha_j \langle v_i, v_j \rangle = 0
\]
We observe the following properties of the incidence vectors:
\begin{itemize}
    \item $\langle v_i, v_i \rangle = |A_i|$
    \item $\langle v_i, v_j \rangle = |A_i \cap A_j| = k$ (for $i \neq j$)
\end{itemize}
Substituting these values into the equation:
\[
    \sum_{i=1}^m \alpha_i^2 |A_i| + k \sum_{i \neq j} \alpha_i \alpha_j = 0
\]
To simplify the second term, we use the identity $(\sum \alpha_i)^2 = \sum \alpha_i^2 + \sum_{i \neq j} \alpha_i \alpha_j$. Rearranging this gives $\sum_{i \neq j} \alpha_i \alpha_j = (\sum \alpha_i)^2 - \sum \alpha_i^2$. We substitute this back into our equation:
\begin{align*}
    \sum_{i=1}^m \alpha_i^2 |A_i| + k \left[ \left(\sum_{i=1}^m \alpha_i\right)^2 - \sum_{i=1}^m \alpha_i^2 \right] &= 0 \\
    \sum_{i=1}^m \alpha_i^2 (|A_i| - k) + k \left(\sum_{i=1}^m \alpha_i\right)^2 &= 0
\end{align*}
Since we assumed $|A_i| > k$, we have $|A_i| - k > 0$. Also, squares of real numbers are non-negative (assuming $k > 0$ and observing $(\sum \alpha_i)^2 \geq 0$). Therefore, we have a sum of non-negative terms equaling zero. This implies that every individual term must be zero. Specifically:
\[
    \alpha_i^2 (|A_i| - k) = 0 \quad \forall i
\]
Since $|A_i| - k \neq 0$, it must be that $\alpha_i = 0$ for all $i$.\\
Thus, the vectors $v_1, \dots, v_m$ are linearly independent. Since they exist in $\mathbb{R}^n$, the dimension of the subspace they span cannot exceed $n$, implying $m \leq n$.
\end{proof}
\begin{thm}
    $R(k+1)\geq {k \choose 3}+1.$ We can group every 3 vertices then color it red 
\end{thm}
\noindent A quadratic form/homogeneous polynomial say $q(x,y)=x^2+2xy+3y^2=\begin{bmatrix}x & y \end{bmatrix} \begin{bmatrix} 1 & 1 \\ 1 & 3 \end{bmatrix} \begin{bmatrix}
        x \\ y\end{bmatrix}$. There is an bijection between quadratic form and symmetric matrices. \\
We can write $A=P^TBP$ where $B$ is a diagonal matrix with eigenvalues $$B=\begin{bmatrix}
    \lambda_1 & 0 \\ 0 & \lambda_2
\end{bmatrix}$$
\begin{defn}
    For a symmetric matrix $A$,
    \begin{enumerate}
        \item Positive definite if $\lambda_i >0$ for all $i$
        \item Negative definite if $\lambda_i<0$ for all $i$
    \end{enumerate}
\end{defn}
\begin{thm}
    Given symmetric $A, q_A(X)=X^TAX$. Let $A$ be real, symmetric $n\times n $. Let $\lambda_1\geq \lambda_2\geq \cdots \geq \lambda_n$ be the eigen values of $A.$ We can conclude $$\lambda_k=\max_{U\subseteq V \dim(U)=k}\min_{X\in U}\frac{X^TAX}{X^TX}$$
\end{thm}
\begin{rmk}
    This statement doesn't care of the magnitude of $X$ and only the direction. We're looking for the direction of greatest change. 
\end{rmk}
\begin{proof}
    We'll first show $\lambda_k\leq \max_U$, For this direction suffices to exhibit one good $u.$ For $v_1,...,v_n$ orthonormal eigen basis, $v_i$ eigenvalues for $\lambda_i$. Let $U_k=\{v_1,...,v_k\}$ and let $X\in \text{Span}(U_k)$ so $X=\sum_{i=1}^{k}\alpha_iv_i.$ WLOG $|X|=1$ since the magnitude does not change the result. This implies $\sum\alpha_j^2=1.$ \\
    \begin{align}
        X^TAX&=\left (\sum\alpha_jv_j\right )^TA\left (\sum\alpha_jv_j\right )\\
        &=\left (\sum\alpha_jv_j\right )^T\left (\sum\alpha_j\lambda_j v_j\right )\\
        &=\sum\lambda_j\alpha_j^2
    \end{align}
    This is a weighted average of $\lambda_1,...,\lambda_k$ so this is at least the $\min(\lambda_1,...,\lambda_k)=\lambda_k.$\\\\
    For the other direction $\lambda_k\geq \max_U$. Given any $U_k,$ we want to show $\exists X\in U_K$ such that $\frac{X^TAX}{X^TX}\leq \lambda_k.$\\
    Let $W=\text{Span}(v_k,...,v_n)$ then $W=n-k+1$. So there exist a vector $X\neq 0, X\in W\cap U_k.$ WLOG, take $|X|=1$. Since $X\in W, X=\sum_{j=k}^n\alpha_jv_j$ and $$X^TAX=\sum_{j=k}^n\lambda_j\alpha_j^2\leq \max(\lambda_k,...,\lambda_n)=\lambda_k$$
\end{proof}
\begin{eg}
    Given a $d$-regular graph $G$ with adjacency matrix $A$ with $\lambda_1\geq...\geq \lambda_n\geq -d.$ Suppose I had a negative eigenvalue that is less than $-d$ then any vertex when applied the adjacency matrix will be the sum of the neighboring vertices, but we can't have $|\sum|>d^2.$ \\
    Given an independent $S$ of size $\alpha.$ Define a vector $v=nI_S-\alpha I=(n-\alpha)I_S-\alpha I_{\overline{S}}$. Then $v\cdot I = 0.$ We know $$\min_{X\subseteq \R^n}\frac{X^TAX}{X^TX}=\lambda_k$$
    So we know $\lambda_k\leq \frac{v^TAv}{v^Tv}$. \begin{align}
        v^Tv&=(nI_S-\alpha I)(nI_S-\alpha I)=\alpha n^2-2\alpha^2n+\alpha^2n=\alpha n (n-\alpha) \\
        v^TAv &= (nI_S-\alpha I)A(nI_S-\alpha I)\\
        &=n I_S A n I_S - 2\alpha^2 I_S A I +\alpha^2 I A I\\
        &= 0-2n\alpha I_S\begin{bmatrix}
            d \\ d \\ \vdots \\ d
        \end{bmatrix} +\alpha^2nd\\
        &=-n\alpha^2 d
    \end{align}
    So we have $\frac{-n\alpha^2d}{\alpha n (n-\alpha)}=\frac{-\alpha d}{(n-\alpha)}=\frac{d}{1-\frac{n}{\alpha}}\geq \lambda_n.$ When we solve for $\alpha$
\end{eg}
\begin{defn}
    Expander Graphs
\end{defn}
\begin{defn}
    Let $G=(V,E)$ and $|V|=n$\\
    Cheeger Constant $$h(G)=\min_{S\subseteq V, |S|\leq \frac{n}{2}}\frac{e(S,\overline{S})}{|S|}$$
\end{defn}
\begin{rmk}
    $h(G)\leq d$ and $h(G)=0\iff G$ is disconnected. 
\end{rmk}
\begin{defn}
    $G$ bipartite on $(L,R)$ with $|L|=|R|=n$ is a $(d,\alpha)$ expander if 
    \begin{enumerate}
        \item every degree in $L$ is $d$
        \item every set $S$ of size $\leq \frac{n}{d}$ in $L$ has $\alpha |S|$ neighbors (in $R$)
    \end{enumerate}
\end{defn}
\begin{thm}
    Let $d\equiv 4,$ choose $d$ edges from each vertex in $L$ independent and 1 and only. \\
    Claim: With constant probability, result is a $(d,\frac{d}{10})$ bipartite expander.
    \begin{proof}
        Let the bad events be for sets $S\subseteq L, T\subseteq R$, $|S|\leq \frac{n}{d}, |T|< \alpha|S|, E_{S,T}=\{N(S)\subseteq T\}.$ Then $\mathbb{P}(E_{S,T})=\left (\frac{|T|}{n} \right )^{d|S|}$
        \\
        \begin{align}
            \mathbb{P}(\exists S,T, |S|\leq \frac{n}{d}, |T|=\alpha|S|,E_{S,T})&\leq \sum_{s=1}^{n/d}{n \choose s}{n \choose \alpha s}\left (\frac{\alpha s}{n} \right )^ds\\
            &\leq \sum_{s=1}^{n/d}{n \choose \alpha s}^2\left ( \frac{\alpha s}{n}\right ) ^ds\\
            &\leq \sum_{s=1}^\infty \left ( \frac{en}{\alpha s}\right ) ^{2\alpha s} \left ( \frac{\alpha s}{n}\right )^{ds}\\
            &=\sum_{s=1}^\infty \left (\frac{n}{d} \right )^{(2\alpha -d)s}e^{2\alpha s}s^{(d-2\alpha)s}\\
            &=\sum_{s=1}^\infty \left ( \frac{n}{\alpha s}\right )^{2\alpha-d)s}e^{2\alpha s}\\
            &\leq \sum_{s=1}^\infty 10^{(2\alpha -d)s}e^{2\alpha s} \tag{$\alpha \leq d/10$}\\
            &=\sum_{s=1}^{\infty}\left (10^{2\alpha-d}e^{2\alpha} \right)^s<1 
        \end{align}
    \end{proof}
\end{thm}
\begin{lem}
    If $p$ has $1$ fermat witnesses, half of the $a$'s relatively prime to $p$ are fermat witnesses. 
\end{lem}
\begin{eg}
    Prime Algorithm
    \begin{enumerate}
        \item Randomly choose $a$
        \item Compute $a^p$ (mod $p$)\\
        If $\not \equiv a$, report not prime else report maybe prime
    \end{enumerate}
    From the previous lemma if not prime, at least half the $a$ will show it.\\
    We want a deterministic expander graph and on the $L$ 
\end{eg}
\begin{thm}
    Similar setup to theroem 5.64, $$\lambda_k=\min_{\dim(U)=k-1}\max_{X\perp U}\frac{X^TAX}{X^TX}$$
\end{thm}
\begin{thm}
    We want to relate the spectral gap to the Creeger Constant
\end{thm}
\begin{proof}
    $\Vec{v}=nI_S-sI=(n-s)I_S-sI_{\overline{S}}$ with $s=|S|$\\
    G is a $d$-regular graph, $A$ is an adjacency matrix, we have $$\lambda_2=\max_{x\perp I}\frac{x^TAx}{x^Tx}\geq \frac{\Vec{v}^TA\Vec{v}}{\Vec{v}^Tv}$$
    Note: $\Vec{v}\cdot I=ns-ns=0$ and $\Vec{v}\cdot \Vec{v} = (n-s)^2s+s^2(n-s)=s(n-s)n$.\\
    Since the graph is $d$-regular we have $ds=2e(S)+e(S,\overline{S})$ and $d(n-s)=2e(\overline{S})+e(S,\overline{S})$.
    \begin{align*}
        \vec{v}^TA\vec{v}=\sum_{(i,j)\in E(G)}v_iv_j&=2e(S)(n-s)^2-2e(S,\overline{S})s(n-s)+2e(\overline{S})s^2\\
        &=(ds-e(S,\overline{S}))(n-s)^2-2e(S,\overline{S})s(n-s)+(d(n-s)e(S,\overline{S}))s^2\\
        &=ds(n-s)n-e(S,\overline{S})[(n-s)^2+2s(n-s)+s^2]\\
        &=ds(n-s)n-e(S,\overline{S})n^2
    \end{align*}
    Then substituting back we have $$\lambda_2\geq \frac{ds(n-s)n-e(S,\overline{S})n^2}{s(n-s)n}=d-\frac{e(S,\overline{S})n}{s(n-s)}$$
    $\forall S\subseteq V, |S|\leq \frac{n}{2}$ we have $$d-\lambda_2\leq \frac{2e(S,\overline{S})}{|S|}\implies \lambda_1-\lambda_2\leq 2h(G)$$
\end{proof}
\begin{eg}
    Consider $A_{K_n}=J-I$ has eigenvalues $n-1$ with multiplicity $1$ and $-1$ with multiplicity $n-1$.
\end{eg}
\begin{lem}
    Lower bound on $\lambda_2$ with $G$ is $d$-regular, $A=A_G$, $A^2$ has eigen values $\lambda_1^2,...,\lambda_n^2$\\
    $Trace(A^2)=nd=\lambda_1^2+\cdots+\lambda_n^2$\\
    \begin{align*}
        nd-d^2=\sum_{k=2}^n\lambda_k^2\leq (n-1)\lambda_*^2 \tag{$\lambda_*=\max_{i\neq 1}|\lambda_i|$}
    \end{align*}
    So we have $d-o(1)=\frac{nd-d^2}{n-1}\leq \lambda_*^2\implies \lambda_*\geq \sqrt{d}-o(1)$
\end{lem}
\begin{thm}
    Consider simple random walk on a graph $G$, $G$ has adjacency matrix $A$ and transition matrix $P=\begin{bmatrix}
        \frac{1}{\deg(v_1)} \cdots \\ \vdots \ddots \\ &\frac{1}{\deg(v_n)} 
    \end{bmatrix}A$\\
    \begin{rmk}
        $P(i,j)=\mathbb{P}(\text{next at $j$} | \text{now at $i$}$
    \end{rmk}
    \noindent If $v_1,...,v_n$ are orthonormal eigenbasis to eiganvalues for $A,$ also true for $P.$ Let corresponding eigen values be $\lambda_1\geq \cdots \geq \lambda_n$ \\
    Consider now a stochastic vector $x$ with $\sum_{i=1}^nx_i=1,x_i\geq 0$, the product $xP=y$ where $y$ is a stochastic vector with the new distribution given we start at distribution $x$. $$x^T=\sum\alpha_i\vec{v_i}$$
    Since $P$ is symmetric 
    \begin{align}
        xP^t&=P^t(\sum\alpha_iv_i)\\
        &=\sum_{i=1}^n(\alpha_i\lambda_i^tv)i\\
        &=\alpha_1v_1+\sum_{i=2}^n\lambda_i^t\alpha_iv_i\\
        &\leq \lambda_*^t\left (\sum\alpha_iv_i\right )
    \end{align}
    So we can conclude $xP^t\to \frac{1}{n}I$
\end{thm}
\begin{thm}
    A knight makes random knight moves: let $\tau$ be the time to return to the bottom left corner. What is $\mathbb{E}[X]$?
\end{thm}
\begin{defn}
    $\rho=$ probability of returning to origin for a random walk on $\R$ then a random walk is recurrent if $\rho =1$ and transient otherwise.
\end{defn}
\begin{rmk}
    A random walk is recurrent on $\Gamma$ iff $\mathbb{E}[\text{visits to origin}]$  is $\infty$\\
    $\mathbb{P}(\text{I visit exactly $k$ times})=p^{k-1}(1-p)$\\
    $\mathbb{E}(\text{visits})=\sum_{k=1}^\infty kp^{k-1}(1-p)=\frac{1}{1-p}$
\end{rmk}
\noindent For our random walk on $\R$, $$\mathbb{E}[\text{visit}]=\sum_{n=0}^\infty \mathbb{E}[I_n]=\sum_{n=0}^{\infty}\frac{{2n \choose n}}{2^{2n}}=\sum_{n=0}^{\infty}a_n\geq \sum_{n\geq N}\frac{1}{\sqrt{\pi n}}$$
Where $a_n\sim \frac{1}{\pi n}$ by Stirling's Formula\\\\
The expected time to return to origin is not finite. 
\begin{thm}
    The probability starting from $j$ we reach $n$ before reaching $0$ is $p_j=\frac{j}{n}.$ $p_j=\frac{1}{2}p_{j-1}+\frac{1}{2}p_{j+1}$
\end{thm}
\noindent For a random walk on $\R^2$ let $I_n=1$ if at $0$ at $2n.$\\
$$\Pr(I_n)=\sum_n\frac{1}{4^{2n}}\sum_{k}\frac{(2n)!}{n!n!(n-k)!(n-k)!}=\sum_n\frac{1}{4^{2n}}{2n \choose n}\sum_k {n \choose k}^2=\sum_{n}\frac{1}{4^{2n}}{2n \choose n}^2$$
For a random walk on $\R^3$ let $I_n=1$ if I return after $2n$ steps.\\
$$\mathbb{E}[\text{visits}]=\sum_n \frac{1}{6^{2n}}\sum_{j,k}\frac{(2n)!}{j!^2k!^2(n-j-k)!^2}\leq \sum_n \frac{{2n \choose n}{n \choose n/3, n/3, n/3}}{2^{2n}3^n}\sum\frac{{n \choose j,k,n-j-k}}{3^n}$$
By Stirling's formula, $\frac{n!}{\left (\frac{n}{3} \right )!^3}\approx \frac{(n/e)^n}{\left (\frac{n}{3e} \right )}$\\
For a random walk on a directed graph, let $$\pi (y)=\mathbb{E}(\text{visits a random direction from $z$ and makes to $y$ before visiting $z$ again})$$
\end{document}
