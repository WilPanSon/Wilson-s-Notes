\documentclass{amsart}

\title{Algebra I 21-610 Spring 2026 Homework III}

\DeclareMathOperator{\Aut}{Aut}
\newtheorem{thm}{Theorem}[section]
\newtheorem{lem}[thm]{Lemma}
\newtheorem{prop}[thm]{Proposition}
\newtheorem{cor}[thm]{Corollary}

\theoremstyle{definition}
\newtheorem{defn}[thm]{Definition}
\newtheorem{eg}[thm]{Example}
\newtheorem{ex}[thm]{Exercise}

\theoremstyle{remark}
\newtheorem{rmk}[thm]{Remark}
\newtheorem{claim}[thm]{Claim}

\newcommand{\Syl}{\text{Syl}_p}
\newcommand{\Stab}{\text{Stab}}
\newcommand{\Orb}{\text{Orb}}
\newcommand{\Sym}{S}
\newcommand{\N}{\mathbb{N}}
\newcommand{\Z}{\mathbb{Z}}
\newcommand{\Q}{\mathbb{Q}}
\newcommand{\R}{\mathbb{R}}
\newcommand{\C}{\mathbb{C}}
\newcommand{\F}{\mathbb{F}}
\newcommand{\gen}[1]{\langle #1 \rangle}
\newcommand{\normal}{\triangleleft}
\newcommand{\res}{\upharpoonright}
\usepackage{amssymb} 


\begin{document}


\maketitle

Homework must be typeset (preferably in \TeX) and submitted
online in Canvas as a PDF file.

Due by class time on Wed Feb 4.

All groups mentioned may be infinite unless I tell you explicitly that they are finite.
\newpage
\begin{enumerate}
  
\item Prove that if $G$ is cyclic of order $n$ then
  $\Aut(G)$ is isomorphic to the group formed by
  $\{ m : \mbox{$0 < m < n$ and $\gcd(m, n) = 1$} \}$
  with the operation of multiplication mod $n$.
  Prove that if $n$ is prime then $\Aut(G)$ is cyclic of order
  $n-1$.  \\
  Let $G=\gen{g}$ and $M=\{ m : \mbox{$0 < m < n$ and $\gcd(m, n) = 1$} \}$
  To show the first statement, consider $\phi: M\to \Aut(G)$ defined by $\phi(m)=f_m$ where $f_m(k)=k^m$ for $k\in G$. $f_m\in \Aut(G)$ as \[
    f_m(g^p\cdot g^q)=g^{m(p+q)}=g^{mp}\cdot g^{mq}=f_m(g^p)\cdot f_m(g^q) \tag{for $p,q\leq n$}
  .\]
  To show $\phi$ is a homomorphism, for all $p,q\in M$, need to show \[
  \phi(pq)=\phi (p) \circ \phi (q)
  .\] 
  We can show this by showing they agree on all values of $k\in G$. We have \[
  \phi(pq)(k)=f_{pq}(k)=k^{pq}=f_p(k^q)=f_p(f_q(k))=\phi(p)(f_q(k))=\phi(p)(\phi(q)(k))
  .\] 
  So $\phi$ is a homomorphism.
  To show $\phi$ is injective, we need to show that if $\phi(p)=\phi(q)$ then $p=q$. We have \[
  \phi_p(g^m)=\phi_q(g^m) \iff g^{mp}=g^{mq} \iff mp \equiv mq \pmod{n} \iff p \equiv q \pmod{n}
  .\] 
  Last line from $p,q\in M$ as it's relatively prime to $n$ so $p=q$ as $p,q<n.$ So $\phi$ is injective. Since $\phi$ is injective on finite sets, it's also surjective and thus $\phi$ is an isomorphism.\\\\
  To show the second statement, since $n$ is prime, for all $0<m<n$, $\gcd(m,n)=1$ so $|M|=n-1$ and $\Aut(G)$ is cyclic of order $n-1$. From previous part, $\Aut(G)$ is isomorphic to $M$ so $\Aut(G)$ is cyclic of order $n-1$ as there must exist an element of order $n-1$ in $\Aut(G).$
\newpage
\item  Let $H$ and $N$ be groups and let $\alpha: H \rightarrow \Aut(N)$ be a HM.
  Define a binary operation $\cdot$ on $\{ (h, n): h \in H, n \in N \}$ as follows:
  $ (h_1, n_1)\cdot (h_2, n_2) = (h_1 *_H h_2, \alpha(h_2^{-1})(n_1) *_N n_2)$.  
  \begin{enumerate}

  \item  Prove that $\cdot$ makes the given set of ordered pairs into a group
    which we will denote by $H \ltimes_\alpha N$. Identify $1$ and $(h, n)^{-1}$.
    
  
    \item Let  $H'= H \times 1$ and $N' = 1 \times N$. Prove that
      $H' \simeq H$, $N' \simeq N$,
      $H' \le H \ltimes_\alpha N$, $N' \lhd H \ltimes_\alpha N$,
      $H' \cap N' = 1$ and $H \ltimes_\alpha N = H' N'$.

   \item Prove that $H \ltimes_\alpha N/N' \simeq H$.    
    
  \end{enumerate}
  \subsection*{Solutions}
  \begin{enumerate}
    \item  Let $H$ and $N$ be groups and let $\alpha: H \rightarrow \Aut(N)$ be a HM.
    Define a binary operation $\cdot$ on $\{ (h, n): h \in H, n \in N \}$ as follows:
    $ (h_1, n_1)\cdot (h_2, n_2) = (h_1 h_2, \alpha(h_2^{-1})(n_1) n_2)$.  
    \begin{enumerate}
  
    \item  Prove that $\cdot$ makes the given set of ordered pairs into a group
      which we will denote by $H \ltimes_\alpha N$. Identify $1$ and $(h, n)^{-1}$.
      
      \begin{proof}
      \textbf{Closure:} Since $H$ and $N$ are groups and $\alpha(h_2^{-1}) \in \Aut(N)$, the product of any two elements results in components within $H$ and $N$.
      
      \textbf{Associativity:} Let $(h_1, n_1), (h_2, n_2), (h_3, n_3) \in H \ltimes_\alpha N$.
      \begin{align*}
          ((h_1, n_1)(h_2, n_2))(h_3, n_3) &= (h_1 h_2, \alpha(h_2^{-1})(n_1)n_2)(h_3, n_3) \\
          &= (h_1 h_2 h_3, \alpha(h_3^{-1})[\alpha(h_2^{-1})(n_1)n_2] n_3) \\
          &= (h_1 h_2 h_3, \alpha(h_3^{-1})(\alpha(h_2^{-1})(n_1)) \cdot \alpha(h_3^{-1})(n_2) \cdot n_3) \\
          &= (h_1 h_2 h_3, \alpha((h_2 h_3)^{-1})(n_1) \cdot \alpha(h_3^{-1})(n_2) \cdot n_3)
      \end{align*}
      Calculating the other grouping:
      \begin{align*}
          (h_1, n_1)((h_2, n_2)(h_3, n_3)) &= (h_1, n_1)(h_2 h_3, \alpha(h_3^{-1})(n_2) n_3) \\
          &= (h_1 h_2 h_3, \alpha((h_2 h_3)^{-1})(n_1) \cdot [\alpha(h_3^{-1})(n_2) n_3])
      \end{align*}
      The terms are identical, so associativity holds.
  
      \textbf{Identity:} The identity is $(1_H, 1_N)$.
      $$(h, n)(1, 1) = (h \cdot 1, \alpha(1^{-1})(n) \cdot 1) = (h, n)$$
      $$(1, 1)(h, n) = (1 \cdot h, \alpha(h^{-1})(1) \cdot n) = (h, 1 \cdot n) = (h, n)$$
  
      \textbf{Inverses:} Let $(x, y)$ be the inverse of $(h, n)$. We require $(h, n)(x, y) = (1, 1)$.
      $$(h x, \alpha(x^{-1})(n) y) = (1, 1)$$
      From the first component, $hx = 1 \implies x = h^{-1}$.
      Substituting $x$ into the second component:
      $$\alpha((h^{-1})^{-1})(n) y = 1 \implies \alpha(h)(n) y = 1$$
      $$y = [\alpha(h)(n)]^{-1} = \alpha(h)(n^{-1})$$
      Thus, $(h, n)^{-1} = (h^{-1}, \alpha(h)(n^{-1}))$.\\
      To verify, we have $(h, n)(h^{-1}, \alpha(h)(n^{-1})) = (h h^{-1}, \alpha(h)(n) \alpha(h)(n^{-1})) = (1, \alpha(h)(n n^{-1})) = (1, 1)$.
      \end{proof}
    
      \item Let  $H'= H \times 1$ and $N' = 1 \times N$. Prove that
        $H' \simeq H$, $N' \simeq N$,
        $H' \le H \ltimes_\alpha N$, $N' \lhd H \ltimes_\alpha N$,
        $H' \cap N' = 1$ and $H \ltimes_\alpha N = H' N'$.
  
      \begin{proof}
      \textbf{Isomorphisms:} The maps $\phi: H \to H'$ via $h \mapsto (h, 1)$ and $\psi: N \to N'$ via $n \mapsto (1, n)$ are clearly bijective homomorphisms (verification omitted for brevity). Thus $H' \simeq H$ and $N' \simeq N$.
  
      \textbf{Subgroups:} 
      $H'$ is a subgroup because $(h_1, 1)(h_2, 1)^{-1} = (h_1, 1)(h_2^{-1}, 1) = (h_1 h_2^{-1}, \alpha(h_2)(1) \cdot 1) = (h_1 h_2^{-1}, 1) \in H'$.
      $N'$ is a subgroup because $(1, n_1)(1, n_2)^{-1} = (1, n_1)(1, n_2^{-1}) = (1, \alpha(1)(n_1)n_2^{-1}) = (1, n_1 n_2^{-1}) \in N'$.
  
      \textbf{Normality of $N'$:} Let $g = (h, n) \in G$ and $k = (1, m) \in N'$. We compute $g k g^{-1}$.
      First, compute $g k$:
      $$ (h, n)(1, m) = (h \cdot 1, \alpha(1^{-1})(n) m) = (h, nm) $$
      Now multiply by $g^{-1} = (h^{-1}, \alpha(h)(n^{-1}))$:
      \begin{align*}
          (h, nm)(h^{-1}, \alpha(h)(n^{-1})) &= (h h^{-1}, \alpha((h^{-1})^{-1})(nm) \cdot \alpha(h)(n^{-1})) \\
          &= (1, \alpha(h)(nm) \cdot \alpha(h)(n^{-1})) \\
          &= (1, \alpha(h)(n m n^{-1}))
      \end{align*}
      Since $n m n^{-1} \in N$ and $\alpha(h) \in \Aut(N)$, the second component is in $N$. Thus $g k g^{-1} \in N'$, so $N' \lhd G$.
  
      \textbf{Intersection and Product:}
      $H' \cap N' = \{(h, 1)\} \cap \{(1, n)\} = \{(1, 1)\}$.
      $H' N' = \{ (h, 1)(1, n) \mid h \in H, n \in N \} = \{ (h, \alpha(1)(1)n) \} = \{ (h, n) \} = G$.
      \end{proof}
  
     \item Prove that $H \ltimes_\alpha N/N' \simeq H$.    
      
     \begin{proof}
     Define $\pi: H \ltimes_\alpha N \to H$ by $\pi(h, n) = h$.
     This is a homomorphism because:
     $$\pi((h_1, n_1)(h_2, n_2)) = \pi(h_1 h_2, \dots) = h_1 h_2 = \pi(h_1, n_1) \pi(h_2, n_2)$$
     The map is surjective (for any $h$, $\pi(h, 1) = h$) and $\ker(\pi) = \{(h, n) \mid h = 1\} = \{(1, n)\} = N'$.
     By the First Isomorphism Theorem, $G/N' \simeq H$.
     \end{proof}
  \end{enumerate}
  \end{enumerate}
\newpage
\item  Let $H \le G$ and $N \lhd G$ with $H \cap N = 1$ and $H N = G$. Prove that
  $G \simeq H \ltimes_\alpha N$ for a suitable choice of HM $\alpha: H \rightarrow \Aut(N)$. 
\subsection*{Solution}
Define $\alpha_h(n)=hnh^{-1}$ and this would indeed map from $N\to N$ as $N\normal G$. This is indeed a homomorphism as \[
  a_{h_1h_2}(n)=h_1h_2 n h_2^{-1}h_1^{-1}=h_1(h_2nh_2^{-1})h_1^{-1}=\alpha_{h_1}(\alpha_{h_2}(n))
.\]
Consider $\phi:H\ltimes_\alpha N\to G$ defined as $\phi(h,n)=hn$. This is a homomorphism as 
\begin{align*}
  \phi((h_1,n_1)\cdot (h_2,n_2))&=\phi(h_1*h_2, \alpha_{h_2^{-1}}(n_1)*n_2)\\
  &=\phi(h_1*h_2, h_2^{-1}n_1h_2n_2)\\
  &=h_1h_2h_2^{-1}n_1h_2n_2\\
  &=h_1n_1h_2n_2\\
  &=\phi(h_1,n_1)\phi(h_2,n_2)
\end{align*}
This is surjective as by assumption $HN=G$ so for any $g\in G$ we can write it as $g=hn$ for some $h\in H$ and $n\in N$. \\
We also have \[
\ker(\phi)=\{(h,n)\in H \ltimes_\alpha N : \phi(h,n)=e_G\}=\{(h,n)\in H \ltimes_\alpha N : hn=e_G\}= \{e_G\}
.\] 
The last line is from $H\cap N=1$ since $n=h^{-1}$ but $h,h^{-1}\in H$ then $h$ has to be identity as otherwise $|H\cap N| \geq 2.$ So by first isomorphism theorem we have \[
H\ltimes_\alpha N\simeq G
.\] 
\newpage
\item Let $p$ be an odd prime and let $\vert G \vert = 2 p$.

  \begin{enumerate}

    \item Prove that there exist subgroups $H$ and $N$ with $\vert H \vert = 2$ and $\vert N \vert = p$.
      
    \item Prove that $N \lhd G$, $H \cap N = 1$ and $G = H N$. 
  
    \item Prove that there are two possible structures for $G$: $G$ is cyclic of order $2 p$ or
      $G$ is isomorphic to the group of symmetries of a regular $p$-gon.

  \end{enumerate}
\subsection*{Part A}
By Sylow's Theorem, we know $\text{Syl}_2(G)$ and $\text{Syl}_p(G)$ are not empty. For any subgroup in $\text{Syl}_2(G)$, the order is $2$ and similarly for $\text{Syl}_p(G)$ the order is $p$.
\subsection*{Part B}
\begin{enumerate}
  \item To show $N\normal G$ we'll show $n_p:=|\Syl(G)|=1$. By the third Sylow's theorem, we know $n_p \equiv 1 \pmod{p}$ and $n_p \mid 2$ so $n_p=1$. We know that any two Sylow $p$-subgroups are conjugate and since there is only one Sylow $p$-subgroup, it must be normal as $\forall g\in G, gNg^{-1}=N.$
  \item To show $H\cap N =1$, we know $H= \{e, h\}$ for some $h\in H$ and $|h|=2$, by lagrange's theorem, the order of it's element must divide the order of the group which is $p$ but $p$ is odd prime so $2 \nmid p$ so $H\cap N =1.$
  \item We can use \[
  |HN|=\frac{|H||N|}{|H\cap N|}=\frac{2p}{1}=2p
  .\] 
  So $G=HN.$ 
\end{enumerate}
\subsection*{Part C}
Since $|N|=p$ then $N$ is cyclic so let $N=\gen{n}$ then for any $h\in H$ we have $hnh^{-1}\in N$ so $hnh^{-1}=n^k$ for some $k\in \{1,2,...,p-1\}$. We know $h^2=1$ so $n=h^2nh^{-2}$ and \[
  h(hnh^{-1})h^{-1}=h(n^k)h^{-1}=(hnh^{-1})^k=n^{k^2}
.\]
So we can conclude $n=h^2nh^{-2}=n^{k^2}.$ So $k^2\equiv 1 \pmod{p}$, this occurs only when $k=1$ or $k=-1.$ \begin{enumerate}
  \item If $k=1$ then $hnh^{-1}=n\implies hn=nh$ so $G$ is abelian as all elements can be expressed as $g=hn$ by $G=HN$. Additionally, as $G$ contains an element of order $2$ and $p$ then there exist an element of order $2p$ as $\gcd(2,p)=1$ so $G$ is cyclic. Thus isomorphic to $\Z_{2p}$
  \item If $k=-1$ then $hnh^{-1}=n^{-1}\implies hn=n^{-1}h$ with $|h|=2$, $|n|=p$ so $G$ is isomorphic to the group of symmetries of a regular $p$-gon.
\end{enumerate}
\newpage
\item Let $Q$ be the group of invertible complex $2 \times 2$ matrices generated by $A$ and $B$ where
  \[
  A = \left ( \begin{array}{rr} i & 0 \\
                                0 & -i \\
  \end{array} \right ),
  B = \left ( \begin{array}{rr} 0 & 1 \\
                                -1 &  0 \\
  \end{array} \right ).
  \]
  
  \begin{enumerate}

  \item  Find $\vert Q \vert$.

  \item  Compute $Z(Q)$ and $[Q, Q]$.

  \item  Find an increasing chain $(N_i)_{0 \le i \le 3}$ of normal subgroups of
    $Q$ such that $\vert N_i \vert = 2^i$.

  \end{enumerate}

\begin{enumerate}
  \item We have $|A|=4$ as \[
    A=\begin{pmatrix} i & 0 \\ 0 & -i \end{pmatrix}, A^2=\begin{pmatrix} -1 & 0 \\ 0 & -1 \end{pmatrix}=-I,A^3=A^2(-I)=-A, A^4=(A^2)^2=I
  .\]
  Similarly $|B|=4$, \[
    B=\begin{pmatrix} 0 & 1 \\ -1 & 0 \end{pmatrix}, B^2=\begin{pmatrix} -1 & 0 \\ 0 & -1 \end{pmatrix}=-I, B^3=B^2(-I)=-B, B^4=(B^2)^2=I
  .\] 
  Analyzing the product we have \[
  AB=\begin{pmatrix} i & 0 \\ 0 & -i \end{pmatrix}\begin{pmatrix} 0 & 1 \\ -1 & 0 \end{pmatrix}=\begin{pmatrix} 0 & i \\ -i & 0 \end{pmatrix} \text{ and } BA=\begin{pmatrix} 0 & 1 \\ -1 & 0 \end{pmatrix}\begin{pmatrix} i & 0 \\ 0 & -i \end{pmatrix}=\begin{pmatrix} 0 & -i \\ i & 0 \end{pmatrix}
  .\] 
  So we can conclude $AB=-BA$ and lastly we look at $(AB)^2$\[
  (AB)^2=\begin{pmatrix} 0 & i \\ -i & 0 \end{pmatrix}\begin{pmatrix} 0 & i \\ -i & 0 \end{pmatrix}=\begin{pmatrix} -1 & 0 \\ 0 & -1 \end{pmatrix}=-I
  .\] 
  Because of the relation $BA = -AB$, we can take any string of $A$'s and $B$'s (e.g., $BABA^2B$) and move all $A$'s to the left and all $B$'s to the right. Every time we swap a $B$ past an $A$, we just change the sign. Thus there are $8$ elements in $Q$ namely, $I, -I, A, -A, B, -B, AB, -AB.$
  \item Clearly, $I,-I\in Z(Q)$ and $A,-A,B,-B$ not in $Z(Q)$ as $AB=-BA$ so $AB\neq BA$. For $AB$ we have $(AB)A=(-BA)A=B$ but $A(AB)=A(-BA)=-B$. So $AB$ and $-AB$ are not in $Z(Q)$. So $Z(G)= \{I, -I\}$. \\\\
  We know $|Q/ \{I, -I\}|=\frac{8}{2}=4$ so $Q/ \{I, -I\}$ is abelian as it's order is a prime squared. We can conclude $[Q,Q]\leq \{I, -I\}$. Take $A,B\in Q$ then we have $[A,B]=ABA^{-1}B^{-1}=AB(-A)(-B)=(AB)^2=-I$ and $[I,A]=IAI^{-1}A^{-1}=I$ so $[Q,Q]=\{I, -I\}$.
  \item Trivially, $N_0=\{I\}$, $N_1=Z(Q)=\{I, -I\}$, and $N_3=Q$. We can choose a subgroup say $ K=\{I, -I, A, -A\}$ as we've shown earlier in part $a$ this is a subgroup ($A^i\in K$). As $|Q/K|=2$ then $K\normal Q$ so we're done by letting $N_2=K$
\end{enumerate}
\newpage
\item  Recall from class that:

  \begin{itemize}

  \item A {\em central series} in a group $G$ is a normal series $(H_i)_{0 \le i \le t}$
    such that $H_{i+1}/H_i \le Z(G/H_i)$ for all $i < t$, equivalently
    $[G, H_{i+1}] \le H_i$ for all $i < t$.

  \item $G$ is {\em nilpotent} if and only if it has a central series. 

  \item The {\em ascending central series} is defined by the recursion
    $Z_0(G) = 1$, $Z_{n+1}(G)/Z_n(G) = Z(G/Z_n(G))$. It's an increasing sequence of characteristic subgroups.

  \item The {\em descending central series} is defined by the recursion
    $\gamma_1(G) = G$, $\gamma_{n+1}(G) = [G, \gamma_n(G)]$.
    It's a decreasing sequence of characteristic subgroups.

  \end{itemize}

  Let $G$ be a nilpotent group with a central series $(H_i)_{0 \le i \le t}$.
  Prove that
  \begin{enumerate}
  \item $H_i \subseteq Z_i(G)$ for all $i$, so that $Z_t(G) = G$.
  \item $\gamma_i(G) \subseteq H_{t - i +1}$ for all $i$, so that $\gamma_{t+1}(G) = 1$.
  \item The least $n$ such that $Z_n(G) = G$, the least $n$ such that
    $\gamma_{n+1}(G) = 1$, and the least $n$ such that there is some central series
   $(K_j)_{0 \le j \le n}$ are all equal.  
  \end{enumerate}
  \subsection*{Solutions}
  \begin{enumerate}
    \item We prove $H_i \subseteq Z_i(G)$ by induction on $i$.
    
    \textbf{Base Case ($i=0$):} $H_0 = 1$ and $Z_0(G) = 1$, so $H_0 \subseteq Z_0(G)$ holds trivially.
    
    \textbf{Inductive Step:} Assume $H_i \subseteq Z_i(G)$.
    By the definition of a central series, $[G, H_{i+1}] \subseteq H_i$.
    By the induction hypothesis, $H_i \subseteq Z_i(G)$, so $[G, H_{i+1}] \subseteq Z_i(G)$.
    
    Recall the definition of the upper central series: $Z_{i+1}(G)$ is the subgroup such that $Z_{i+1}(G)/Z_i(G) = Z(G/Z_i(G))$. Equivalently, $x \in Z_{i+1}(G) \iff [G, x] \subseteq Z_i(G)$.
    
    Since $[G, H_{i+1}] \subseteq Z_i(G)$, it follows by definition that $H_{i+1} \subseteq Z_{i+1}(G)$.
    
    Thus, by induction, $H_i \subseteq Z_i(G)$ for all $i$. For $i=t$, since $H_t = G$, we have $G \subseteq Z_t(G)$, which implies $Z_t(G) = G$.

    \item We prove $\gamma_i(G) \subseteq H_{t - i + 1}$ by induction on $i$.
    
    \textbf{Base Case ($i=1$):} $\gamma_1(G) = G$ and $H_{t - 1 + 1} = H_t = G$. Thus $\gamma_1(G) \subseteq H_t$ holds.
    
    \textbf{Inductive Step:} Assume $\gamma_i(G) \subseteq H_{t - i + 1}$.
    By definition, $\gamma_{i+1}(G) = [G, \gamma_i(G)]$.
    Using the induction hypothesis:
    \[
    \gamma_{i+1}(G) = [G, \gamma_i(G)] \subseteq [G, H_{t - i + 1}]
    \]
    Since $(H_k)$ is a central series, we know $[G, H_k] \subseteq H_{k-1}$. Letting $k = t - i + 1$, we get:
    \[
    [G, H_{t - i + 1}] \subseteq H_{t - i} = H_{t - (i+1) + 1}
    \]
    Therefore, $\gamma_{i+1}(G) \subseteq H_{t - (i+1) + 1}$.
    
    By induction, the inclusion holds for all $i$.
    In particular, for $i=t+1$, we have $\gamma_{t+1}(G) \subseteq H_{t - (t+1) + 1} = H_0 = 1$. Thus $\gamma_{t+1}(G) = 1$.

    \item Let $n_Z$ be the least integer such that $Z_{n_Z}(G) = G$.
    Let $n_\gamma$ be the least integer such that $\gamma_{n_\gamma+1}(G) = 1$.
    Let $n_{min}$ be the minimal length of any central series of $G$.\\
    Since the Upper Central Series $(Z_i)$ is itself a central series of length $n_Z$, we must have $n_{min} \le n_Z$.\\
    Let $(H_i)_{0 \le i \le t}$ be any central series. From part (a), we know $H_t \subseteq Z_t(G)$. Since $H_t=G$, $Z_t(G)=G$. This implies $n_Z \le t$. Since this holds for \emph{any} central series of length $t$, it holds for the minimal one, so $n_Z \le n_{min}$.\\
     Combining the above, $n_Z = n_{min}$.\\
     From part (b), given any central series of length $t$, $\gamma_{t+1}(G) = 1$. This implies $n_\gamma \le t$. Taking the minimal central series (where $t = n_{min}$), we get $n_\gamma \le n_{min}$.\\
     Conversely, consider the series defined by the lower central series: set $K_j = \gamma_{n_\gamma - j + 1}(G)$ for $0 \le j \le n_\gamma$.
        Then $K_0 = \gamma_{n_\gamma+1}(G) = 1$ and $K_{n_\gamma} = \gamma_1(G) = G$.
        Check the central condition:
        $[G, K_{j+1}] = [G, \gamma_{n_\gamma - j}(G)] = \gamma_{n_\gamma - j + 1}(G) = K_j$.
        Thus, this is a valid central series of length $n_\gamma$. Since $n_{min}$ is the minimal length of any central series, $n_{min} \le n_\gamma$.\\Combining these inequalities, $n_\gamma = n_{min}$.
    
    Conclusion: $n_Z = n_{min} = n_\gamma$.
  \end{enumerate}

\newpage
\item Let $U_n$ be the group of $n \times n$ real invertible upper triangular matrices with $1$'s on the diagonal, and let
  $B_n$ be the group of all $n \times n$ real invertible upper triangular matrices.
  Prove that:
  
\begin{enumerate}
\item $U_n \lhd B_n$.    
\item $U_n$ is solvable.
\item $B_n/U_n$ is abelian.
\item $B_n$ is solvable.
\end{enumerate}
\subsection*{Solutions}
\begin{enumerate}
  \item Define a map $\phi: B_n \to (\mathbb{R}^\times)^n$ (where $(\mathbb{R}^\times)^n$ is the group of diagonal matrices under multiplication) by $\phi(A) = (A_{11}, A_{22}, \dots, A_{nn})$.
  
  This is a homomorphism because for upper triangular matrices $A$ and $B$, the diagonal entries of the product are the products of the diagonal entries: $(AB)_{ii} = A_{ii}B_{ii}$.
  
  The kernel of this map is the set of matrices where every diagonal entry is $1$, which is exactly $U_n$. Since the kernel of any homomorphism is a normal subgroup, $U_n \lhd B_n$.

  \item We prove $U_n$ is solvable by induction on $n$.
  
  \textbf{Base Case ($n=2$):}
  $U_2 = \left\{ \begin{pmatrix} 1 & x \\ 0 & 1 \end{pmatrix} : x \in \mathbb{R} \right\}$.
  This group is isomorphic to the additive group of real numbers $(\mathbb{R}, +)$ via the map $\begin{pmatrix} 1 & x \\ 0 & 1 \end{pmatrix} \mapsto x$. Since $(\mathbb{R}, +)$ is abelian, $U_2$ is solvable.
  
  \textbf{Inductive Step:}
  Assume $U_n$ is solvable. Consider the homomorphism $\psi: U_{n+1} \to U_n$ defined by removing the last row and the last column (mapping the $(n+1) \times (n+1)$ matrix to its top-left $n \times n$ block).
  
  Let $A = \begin{pmatrix} A' & v \\ 0 & 1 \end{pmatrix} \in U_{n+1}$, where $A' \in U_n$ and $v \in \mathbb{R}^n$. Then $\psi(A) = A'$.
  This map is surjective. The kernel is:
  \[
  K = \ker(\psi) = \left\{ \begin{pmatrix} I_n & v \\ 0 & 1 \end{pmatrix} : v \in \mathbb{R}^n \right\}.
  \]
  We can observe that $K \cong (\mathbb{R}^n, +)$, which is an abelian group. Thus $K$ is solvable.
  
  By the First Isomorphism Theorem, $U_{n+1}/K \cong U_n$. Since both $K$ (the kernel) and $U_{n+1}/K$ (the quotient, isomorphic to $U_n$) are solvable, the group $U_{n+1}$ is solvable.
  
  Thus, by induction, $U_n$ is solvable for all $n \ge 1$.

  \item To show $B_n/U_n$ is abelian, it suffices to show that the derived subgroup $[B_n, B_n]$ is contained in $U_n$.
  
  Let $A, B \in B_n$. Since diagonal entries of upper triangular matrices commute, for any $1 \le i \le n$:
  \[
  (ABA^{-1}B^{-1})_{ii} = A_{ii} B_{ii} A_{ii}^{-1} B_{ii}^{-1} = A_{ii} A_{ii}^{-1} B_{ii} B_{ii}^{-1} = 1.
  \]
  Since all diagonal entries of the commutator $ABA^{-1}B^{-1}$ are $1$, the commutator belongs to $U_n$. Therefore, $[B_n, B_n] \le U_n$, which implies $B_n/U_n$ is abelian.

  \item We recall the property that if $N \lhd G$ and both $N$ and $G/N$ are solvable, then $G$ is solvable.
  
  From part (a), $U_n \lhd B_n$.
  From part (b), $U_n$ is solvable.
  From part (c), $B_n/U_n$ is abelian, and all abelian groups are solvable.
  
  Therefore, $B_n$ is solvable.
\end{enumerate}
\end{enumerate}
\end{document}   


