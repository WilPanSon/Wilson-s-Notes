\documentclass[a4paper]{article}
\usepackage[margin=1in]{geometry}
\usepackage[english]{babel}
\usepackage[utf8]{inputenc}
\usepackage{amsmath}
\usepackage{graphicx}
\usepackage{amssymb}
\usepackage{amsthm}
\usepackage{tikz-cd}
\usepackage{mathrsfs}
\usepackage[colorinlistoftodos]{todonotes}
\usepackage{enumitem}
\usepackage{yfonts}
\usepackage{dsfont}
\usepackage{mathtools}
\usepackage{hyperref}

\DeclarePairedDelimiter\ceil{\lceil}{\rceil}
\DeclarePairedDelimiter\floor{\lfloor}{\rfloor}

\title{21-610 Homework 2}
\author{Wilson Pan}
\date{\today}

\newtheorem{thm}{Theorem}[section]
\newtheorem{lem}[thm]{Lemma}
\newtheorem{defn}[thm]{Definition}
\newtheorem{eg}[thm]{Example}
\newtheorem{ex}[thm]{Exercise}
\newtheorem{conj}[thm]{Conjecture}
\newtheorem{cor}[thm]{Corollary}
\newtheorem{claim}[thm]{Claim}
\newtheorem{rmk}[thm]{Remark}

\newcommand{\ie}{\emph{i.e.} }
\newcommand{\cf}{\emph{cf.} }
\newcommand{\into}{\hookrightarrow}
\newcommand{\dirac}{\slashed{\partial}}
\newcommand{\R}{\mathbb{R}}
\newcommand{\C}{\mathbb{C}}
\newcommand{\Z}{\mathbb{Z}}
\newcommand{\N}{\mathbb{N}}
\newcommand{\Q}{\mathbb{Q}}
\newcommand{\LieT}{\mathfrak{t}}
\newcommand{\T}{\mathbb{T}}
\newcommand{\A}{\mathds{A}}
\newcommand{\HG}{\mathcal{H}}
\newcommand{\F}{\mathbb{F}}
\newcommand{\poly}[2]{\text{Poly}_{#1}(#2)}
\newcommand{\gen}[1]{\langle #1 \rangle}
\newcommand{\Hom}{\text{Hom}}
\newcommand{\E}{\mathbb{E}} 
\newcommand{\Syl}{\text{Syl}_p}
\begin{document}

\maketitle
\section*{Problem 1}
\begin{lem}
    If $H_1,H_2\leq G$ then their intersection $H_1\cap H_2\leq G$
    \begin{proof}
        $H_1\cap H_2$ is non-empty as $e\in H_1\cap H_2$. To show closure under multiplication, if $x,y\in H_1\cap H_2$ then $x,y\in H_1$ and $x,y\in H_2$ so $xy\in H_1$ and $xy\in H_2$. Thus, $xy\in H_1\cap H_2$. To show closure under inverses if $x\in H_1 \cap H_2$ then $x\in H_1$ so $x^{-1}\in H_1$ similarly for $H_2$. So $x^{-1}\in H_1\cap H_2.$ 
    \end{proof}
\end{lem}
\subsection*{Part A}
Let $G=p^n m$ where $p \nmid m$ then $|H|=p^n$ and suppose $|N|=p^r q$ where $r\leq n$ and $q | m$ by lagrange theorem. 
We first show $H\cap N$ is a subgroup. $N$ is trivially a subgroup and $H$ is a subgroup of $G$ by definition of $Syl_p(G)$. So by lemma 0.1 $H\cap N$ is a subgroup. \\
Since $N\triangleleft G$ then $HN$ is a subgroup. By second isomorphism theorem we have \[
|H\cap N|=\frac{|H||N|}{|HN|}
.\] 
We know $H\leq HN \leq G$ so $|HN|=p^n s$ where $s\leq m$ as $|H| \Bigm | |HN|$. So we can conclude \[
|H \cap N | = \frac{p^n \cdot p^r q}{p^n s}=p^r \frac{q}{s}
.\] 
Since $H \cap N \leq H$ then $\frac{q}{s}=1$ by lagrange as $H\cap N$ is a $p$-subgroup. Thus $|H\cap N|=p^r$ and $H\cap N\in Syl_p(N)$. \\
\subsection*{Part B}
Consider any $Q\in \Syl(N)$ and take any $H\in \Syl(G)$ then $H\cap N\in \Syl(N)$ by part A. By Sylow's Theorem II there exist $n\in N$ such that $n(H\cap N)n^{-1}=Q$. Expanding we have \[
(nHn^{-1})\cap (n Nn^{-1})=Q\iff (nHn^{-1}) \cap N=Q
.\]  
Since $n\in G$ then $nHn^{-1}\in \Syl(G)$ as conjugation is a bijection. So we're done.
\newpage
\section*{Problem 2}
\subsection*{Part A}
Let $|G|=p^n m$ where $p \nmid m$ then $|H|=p^n$ and we can let $|N|=p^r q$ where $r\leq n$ and $q \mid m$. \\
Then the order of subgroups in $\Syl(G/N)$ is $p^{n-r}$ as $|G/N|=\frac{p^n m}{p^r \cdot q}=p^{n-r} \frac{m}{q}.$ \\
Since $N \triangleleft G$, $N \triangleleft HN$. Thus the quotient group $HN/N$ is well-defined. By second isomorphism theorem, \[
\left | \frac{HN}{N}\right |=\frac{|H|}{|H\cap N|}
.\] 
We've previously shown in problem 1A that $|H\cap N|=p^r$ so $\left | \frac{HN}{N}\right |=\frac{p^n}{p^r}=p^{n-r}$. So $\frac{HN}{N}\in \Syl(G/N)$
\subsection*{Part B}
Consider any $Q\in \Syl(G/N)$ we want to show that $Q=\frac{H'N}{N}$ for some $H'\in \Syl(G).$ From part A, we know $P=\frac{HN}{N}\in \Syl(G/N)$ for $H\in \Syl(G).$ So there exist $gN\in G/N$ such that $Q=gN P (gN)^{-1}$ by second Sylow Theorem. We can rearrange using normality of $N$ \[
Q=(gN) P (gN)^{-1} \iff Q=g\frac{HN}{N}g^{-1} \iff Q= \frac{gHNg^{-1}}{N}\iff Q= \frac{gHg^{-1}N}{N}
.\] 
We can let $H'=gHg^{-1}$ as we know $gHg^{-1}\in \Syl(G)$ by Sylow's second theorem. So $Q=\frac{H'N}{N}$ satisfies the problem. 
\newpage
\section*{Problem 3}
Let $H$ act on $X:=G/H$ by left multiplication. So for some $h\in H$ and $gH\in X$, we have $h(gH)=(hg)H$. Then by fixed point theorem \[
|X|\equiv |X^H| \text{ mod }p
.\] 
So we want to find $gH\in X$ such that $h(gH)=gH$ for all $h\in H.$ Rearranging we have $g^{-1}hgH=H$ and thus $g^{-1}hg\in H.$ We want this to hold for all $h\in H$ so $g\in N_G(H)$ and $gH\in N_G(H)/H$. Thus, \[
[G:H]\equiv [N_G(H):H]
.\]   
\newpage
\section*{Problem 4}
We factor $48=2^4\cdot 3$ so $Syl_2(G)$ has subgroups of order $16$. We know $|Syl_2(G)|$ is odd and divides $3$ by Sylow's Third Theorem so $|Syl_2(G)|\in \{1,3\}.$ Since the problem posits two distinct Sylow 2-subgroups, $|Syl_2(G)|=3.$ Take distinct $P,Q\in Syl_2(G)$ we have for just subsets \[
|PQ|=\frac{|P||Q|}{|P\cap Q|}
.\] 
Since $P,Q$ are subgroups of $G$ then $|PQ|\leq 48.$ Then rearranging \[
|P\cap Q|\geq \frac{|P||Q|}{48}=\frac{16^2}{48}\approx 5.33
.\] 
Since $P\cap Q\leq P$ then by Lagrange's Theorem $|P \cap Q| \Bigm | 16$ and as $P$ and $Q$ are distinct $|P\cap Q|<16.$ So $|P\cap Q|=8$ as the factors of $16$ are $1,2,4,8,16.$ 
\newpage
\section*{Problem 5}
\begin{lem}
    If $S$ is a $p$-subgroup of $G$ and $P\in \Syl(G)$, then there exists $g\in G$ such that $S\subseteq gPg^{-1}$
    \begin{proof}
        Let $\Omega= G/P$ then $|\Omega|$ is not divisible by $p$ as $P$ has the same power of $p$ as $G.$ Let $S$ act on $\Omega$ by left multiplication. So $s \cdot (gP)=(sg)P.$. Then by fixed point theorem \[
        |\Omega|\equiv \left |\Omega^{ \text{fix}}\right | \text{ mod } p
        .\] 
        Since $\Omega$ is not divisible by $p$, there exist at least one fixed point. Let $gP$ be such a fixed point then $s(gP)=gP$ for all $s\in S.$ So 
        \[
        s(gP)=gP \iff g^{-1}sgP=P\iff g^{-1}sg\in P
        .\] 
        Since this holds for all $s\in S, g^{-1}Sg\subseteq P$ and thus $S\subseteq gPg^{-1}.$
    \end{proof}
\end{lem}
\noindent Now to prove the main statement. \\
Let $G$ be a finite group, $H\leq G$ and $P\in \Syl(G).$ Then by Sylow's first theorem, $\Syl(H)$ is not empty. So take $S\in \Syl(H)$ then by lemma 0.2, there exists $g\in G$ such that $S\subseteq gPg^{-1}.$ So $S\leq H \cap P^g$ and $H\cap P^g\leq H$ so $H\cap P^g$ is a $p$-subgroup in $H$ and as $S$ is a maximal $p$-subgroup of $H$ so is $H\cap P^g.$ Thus $H\cap P^g\in \Syl(H).$
\end{document}