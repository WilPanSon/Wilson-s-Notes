\documentclass[a4paper]{article}
\usepackage[margin=1in]{geometry}
\usepackage[english]{babel}
\usepackage[utf8]{inputenc}
\usepackage{amsmath}
\usepackage{graphicx}
\usepackage{amssymb}
\usepackage{amsthm}
\usepackage{tikz-cd}
\usepackage{mathrsfs}
\usepackage[colorinlistoftodos]{todonotes}
\usepackage{enumitem}
\usepackage{yfonts}
\usepackage{dsfont}
\usepackage{mathtools}
\usepackage{hyperref}

\DeclarePairedDelimiter\ceil{\lceil}{\rceil}
\DeclarePairedDelimiter\floor{\lfloor}{\rfloor}

\title{21-610 HW \#1}
\author{Wilson Pan}
\date{\today}

\newtheorem{thm}{Theorem}[section]
\newtheorem{lem}[thm]{Lemma}
\newtheorem{defn}[thm]{Definition}
\newtheorem{eg}[thm]{Example}
\newtheorem{ex}[thm]{Exercise}
\newtheorem{conj}[thm]{Conjecture}
\newtheorem{cor}[thm]{Corollary}
\newtheorem{claim}[thm]{Claim}
\newtheorem{rmk}[thm]{Remark}

\newcommand{\ie}{\emph{i.e.} }
\newcommand{\cf}{\emph{cf.} }
\newcommand{\into}{\hookrightarrow}
\newcommand{\dirac}{\slashed{\partial}}
\newcommand{\R}{\mathbb{R}}
\newcommand{\C}{\mathbb{C}}
\newcommand{\Z}{\mathbb{Z}}
\newcommand{\N}{\mathbb{N}}
\newcommand{\Q}{\mathbb{Q}}
\newcommand{\LieT}{\mathfrak{t}}
\newcommand{\T}{\mathbb{T}}
\newcommand{\A}{\mathds{A}}
\newcommand{\HG}{\mathcal{H}}
\newcommand{\F}{\mathbb{F}}
\newcommand{\poly}[2]{\text{Poly}_{#1}(#2)}
\newcommand{\gen}[1]{\langle #1 \rangle}
\newcommand{\Hom}{\text{Hom}}
\newcommand{\E}{\mathbb{E}} 

\begin{document}
\maketitle
\section*{Problem 1}
Since $G/Z(G)$ is cyclic then there exist some generator $xH$ where $H:=Z(G)$ such that $\gen{xH}=Z/Z(G).$ For any $g\in G$, $g$ must be in one of the cosets in $Z/Z(G)$ since the cosets partition $G$. So take any $g,g'\in G$ then suppose $g\in x^nH$ and $g'\in x^m H$. We can write \[
g=x^n h \text{ and } g'=x^m h' \text{ for some $h,h'\in H$}
.\] 
Then \[
g\cdot g'=x^n h x^m h'=h'x^nx^mh=h'x^{n+m}h=h'x^mx^nh=x^mh'x^nh=g'\cdot g
.\] 
Such manipulation is possible as $h,h'\in H=Z(G)$. Thus, $G$ is abelian.\\\\
A counter example to $G/Z(G)$ abelian but $G$ not is $D_8$ as $|Z(D_8)|=2$ as $Z(G)= \{e, r^2\}$. So by Lagrange's Theorem, $|D_8/Z(D_8)|=4$ and by problem 2A is abelian but $D_8$ is not. 
\newpage
\section*{Problem 2}
\begin{lem}
    For a prime $p$, groups of order $p^n$ have non-trivial centers.
    \begin{proof}
        Let our group be $G$ with $|G|=p^n$. Consider the conjugacy actions then the orbits under conjugacy partition $G$ so \[
        |G|=\sum_{x\in G}|\mathcal{C}_x|=\sum_{x\in Z(G)}|\mathcal{C}_x|+\sum_{x\notin Z(G)}|\mathcal{C}_x|=|Z(G)|+\sum_{x\notin Z(G)}|\mathcal{C}_x|
        .\] 
        The last equality is true by if $x\in Z(G)$ then $\forall g\in G$, $gx=xg\iff gxg^{-1}=x$ so $|\mathcal{C}_x|=1.$
        Since conjugacy classes are subgroups of $G,$ by lagrange's theorem, for $x\notin Z(G)$ we have $|\mathcal{C}_x| \Bigm ||G|$. With $|\mathcal{C}_x|>1$ otherwise we'll have $x\in Z(G)$, we have $|C_x|=p^m$ for some integer $m<n.$ Consequently, since $p \Bigm |G|$ then \[
        p \Bigm ||Z(G)|+\sum_{x\notin Z(G)}|\mathcal{C}_x|
        .\] 
        So $p\Bigm | |Z(G)|$ so $|Z(G)|>1$ and is thus non-trivial
    \end{proof}
\end{lem}
\subsection*{Part A}
\noindent To show all groups say $G$ of order $p^2$ is abelian, consider the center $Z(G).$ Since $Z(G)\leq G$ then $|Z(G)| \Bigm | |G|$ so by Lagrange's Theorem $|Z(G)|$ can be potentially $1,p$ or $p^2$. However, by lemma 0.1 we have $|Z(G)|\neq 1$. If $|Z(G)|=p$ then by Lagrange's Theorem, $|G/Z(G)|=\frac{|G|}{|Z(G)|}=p$ and all groups of order $p$ are cyclic so by the result in problem $1$, $G$ is abelian. In the case of $|Z(G)|=p^2$ then $G=Z(G)$ so $G$ is abelian. 
\subsection*{Part B}
If $G$ is a non-abelian group of order $p^3$, we want to show $|Z(G)|=p$. 
We know $|Z(G)| \Bigm| |G|$, so $|Z(G)|$ can potentially be $1, p, p^2, \text{or } p^3$. \\\\
By Lemma 0.1, $|Z(G)|\neq 1$. 
Additionally, $|Z(G)|\neq p^3$, otherwise $Z(G)=G$ and $G$ would be abelian, a contradiction. \\
If $|Z(G)|=p^2$, then $|G/Z(G)|=\frac{|G|}{|Z(G)|}=\frac{p^3}{p^2}=p$. \\
Since every group of prime order is cyclic, $G/Z(G)$ would be cyclic. \\
However, by the result in problem 1 (if $G/Z(G)$ is cyclic then $G$ is abelian), this implies $G$ is abelian, which is a contradiction. 
Thus, $|Z(G)|\neq p^2$.\\
Therefore, the only remaining possibility is $|Z(G)|=p$.\\
To show $G/Z(G)$ is abelian, observe that $|G/Z(G)|=\frac{p^3}{p}=p^2$. 
By the result proved in part A, any group of order $p^2$ is abelian. Thus, $G/Z(G)$ is abelian. 
Note that $G/Z(G)$ is not cyclic, as otherwise (by Problem 1 again) $G$ would be abelian.
\newpage
\section*{Problem 3}
Let $\phi\in Z(Aut(G))$ then consider the automorphism defined by $c_{g}(x)=gxg^{-1}$ for some $g\in G$ then \[
\phi(g)\phi(x)\phi(g)^{-1}=\phi(gxg^{-1})=\phi\circ c_g = c_g \circ \phi=c_g(\phi(x))=g\phi(x)g^{-1}
.\]
So we have \[
\phi(g)\phi(x)\phi(g)^{-1}=g\phi(x)g^{-1}
.\]
Let $y=\phi(x)$ then \[
\phi(g)y\phi(g)^{-1}=gyg^{-1}\iff g^{-1}\phi(g)y=yg^{-1}\phi(g)
.\] 
We can conclude $g^{-1}\phi(g)\in Z(G)$\\
Since $|Z(G)|=1$ then $Z(G)= \{e\}$ so $g^{-1}\phi(g)=e$ and thus $\phi(g)=g$ and this holds for arbitrary $g\in G$ so $|Z(Aut(G))|=1$ and is the identity automorphism.
\newpage
\section*{Problem 4}
Let $|a|=n$ and $|b|=m$ then as $G$ is abelian\[
(ab)^{nm}=a^{nm}b^{nm}= \left( a^n \right)^m \left( b^m \right)^n=e^me^n=e
.\] 
So $|ab|\leq nm$.\\\\
To show this may fail when $G$ is not abelian, consider $G=GL_2(\R)$ and two matrices of order $2$ namely \begin{align*}
    A=\begin{pmatrix}
        {1} & {1} \\ 
        {0} & {-1}
    \end{pmatrix}, \text{ } B=\begin{pmatrix}
        {1} & {0} \\ 
        {0} & {-1}
    \end{pmatrix}
\end{align*}
We have \[
A^2=\begin{pmatrix}
    {1} & {0} \\ 
    {0} & {1}
\end{pmatrix}
.\] 
So $|A|=2$ and 
\[
B^2=\begin{pmatrix}
    {1} & {0} \\ 
    {0} & {1}
\end{pmatrix}
.\] 
So $|B|=2$. \\
Claim: $(AB)^n=\begin{pmatrix}
    {1} & {-n} \\ 
    {0} & {1}
\end{pmatrix}$\\
We'll show this by induction. \\
(Base Case) When $n=1$, $AB=\begin{pmatrix}
    {1} & {-1} \\ 
    {0} & {1}
\end{pmatrix}$, so it holds true for $n=1$\\
(Induction Step) Assume it holds true for $n$ then \[
(AB)^{n+1}=(AB)^n(AB)=\begin{pmatrix}
    {1} & {-n} \\ 
    {0} & {1}
\end{pmatrix} \begin{pmatrix}
    {1} & {-1} \\ 
    {0} & {1}
\end{pmatrix}= \begin{pmatrix}
    {1} & {-(n+1)} \\ 
    {0} & {1}
\end{pmatrix}
.\] 
So it holds true by induction. Consequently, $(AB)^{n+1}\neq \begin{pmatrix}
    {1} & {0} \\ 
    {0} & {1}
\end{pmatrix}$ as $-n\neq 0$ for $n\geq 1.$
\newpage
\section*{Problem 5}
\subsection*{Part A}
Consider a matrix where the rows are group elements of $G$ and columns are set elements of $X$. Then for any grid $(g,x)\in G\times X$, it is $1$ if $g\cdot x = x$ and $0$ otherwise. Then the LHS $\sum_{g\in G}|Fix(g)|$ goes along each row and counts the number of ones in each row. Similarly for $\sum_{x\in X}|G_x|$ counts the number of ones in each column. So they are equal as both counts the number of ones in the entire matrix.  
\subsection*{Part B}
We can rewrite with part A to get \[
\frac{\sum_{g\in G}|Fix(g)|}{|G|}=\frac{\sum_{x\in X}|G_x|}{|G|}=\frac{\sum_{x\in X}\frac{|G|}{|\mathcal{O}_x|}}{|G|}=\sum_{x\in X}\frac{1}{|\mathcal{O}_x|}
.\]
Consider any orbit say $\theta$ and if $|\theta|=n$ then for each $x\in \theta$ it contributes $\frac{1}{n}$ to the sum but there are $n$ elements so the total contribution of any orbit is $1$. 
\[
\sum_{x\in X}\frac{1}{|\mathcal{O}_x|}=\# \text{ orbits}
.\] 
\newpage
\section*{Problem 6}
Let $G=D_{24}$ and $X= \{(c_1,...,c_{12}) | c_i\in \{ \text{red, green, blue}\}\}$. Then let $G$ act on $X$ where
\begin{align*}
    r \cdot (c_1,...,c_{12})&=(c_2,...,c_{12},c_1)\\
    s \cdot (c_1,...,c_{12})&=(c_{12},...,c_1)
\end{align*}
For any $r^k\in D_{24}$, it'll break the beads into $\gcd(12,k)$ cycles and each cycle must be the same color since if a cycle $c$ has length say $m+1$ and  $c=(c_i,c_{i+k},...c_{i+mk})$ then $c_i=c_{i+k},c_{i+k}=c_{i+2k},...,c_{i+mk}=c_i$. So $c_i=c_{i+k}=\cdots=c_{i+mk}.$\\
For each of these cycles there are $3$ possible ways to color them so $|Fix(r^k)|=3^{\gcd(12,k)}.$\\
For the reflections, if the line of symmetry goes through two vertices then we fixed $2$ vertices (vertices the line passed through) and there are $5$ cycles. So there are $3^2\cdot 3^5=3^7$ fixed elements in this case. Otherwise, if the line of symmetry does not go through vertices we have $6$ cycles so $3^6.$ Each case has $6$ cases so $6(3^7+3^6).$\\
By Problem 5, we have
\[
\# \text{Orbits}= \frac{\sum_{g\in G}|Fix(G)|}{|G|}=\frac{6(3^7+3^6)+\sum_{k=1}^{12}3^{\gcd(12,k)}}{24}=\frac{6(3^7+3^6)+532416}{24}=22913
.\] 
\end{document}