\documentclass[a4paper,12pt]{article}
\usepackage[margin=1in]{geometry}
\usepackage[english]{babel}
\usepackage[utf8]{inputenc}
\usepackage{amsmath, amssymb, amsthm, mathtools}
\usepackage{tikz-cd}
\usepackage{mathrsfs}
\usepackage{enumitem}
\usepackage{dsfont}
\usepackage{hyperref}

% --- Formatting & Environments ---
\newtheorem{thm}{Theorem}[section]
\newtheorem{lem}[thm]{Lemma}
\newtheorem{prop}[thm]{Proposition}
\newtheorem{cor}[thm]{Corollary}

\theoremstyle{definition}
\newtheorem{defn}[thm]{Definition}
\newtheorem{eg}[thm]{Example}
\newtheorem{ex}[thm]{Exercise}

\theoremstyle{remark}
\newtheorem{rmk}[thm]{Remark}
\newtheorem{claim}[thm]{Claim}

\newcommand{\Aut}{\text{Aut}}
\newcommand{\Stab}{\text{Stab}}
\newcommand{\Orb}{\text{Orb}}
\newcommand{\Sym}{S}
\newcommand{\N}{\mathbb{N}}
\newcommand{\Z}{\mathbb{Z}}
\newcommand{\Q}{\mathbb{Q}}
\newcommand{\R}{\mathbb{R}}
\newcommand{\C}{\mathbb{C}}
\newcommand{\F}{\mathbb{F}}
\newcommand{\gen}[1]{\langle #1 \rangle}
\newcommand{\normal}{\triangleleft}
\newcommand{\res}{\upharpoonright}
\newcommand{\dom}{\text{dom}}
\newcommand{\cod}{\text{cod}}
\renewcommand{\hom}{\text{hom}}
\newcommand{\im}{\text{im}}
\newcommand{\Span}{\text{span}}
\title{\textbf{Algebra I: Class Notes}}
\author{Wilson Pan}
\date{\today}

\begin{document}
\maketitle
\tableofcontents
\newpage

\section{Lecture 1: Group Actions \& The Orbit-Stabilizer Theorem (Jan 12)}

\subsection{Permutations and Automorphisms}

\begin{defn}[Symmetric Group]
    Let $X$ be a set. The set of all permutations (bijections) of $X$ is denoted by $\Sym_X$ (or sometimes $\Sigma_X$). Under function composition, $\Sym_X$ forms a group.
\end{defn}

\begin{defn}[Automorphism Group]
    Let $(G, \cdot)$ be a group. An automorphism of $G$ is a bijection $\phi: G \to G$ that is also a homomorphism, i.e.,
    \[ \phi(g_1 g_2) = \phi(g_1)\phi(g_2) \quad \forall g_1, g_2 \in G. \]
    The set of all such automorphisms is denoted by $\Aut(G)$. It forms a group under composition.
\end{defn}

\subsection{Group Actions}

\begin{defn}[Group Action]
    Let $G$ be a group and $X$ be a set. An \textbf{action} of $G$ on $X$ is a homomorphism $\phi: G \to \Sym_X$.
    
    We typically write the action as $g \cdot x \coloneqq \phi(g)(x)$. This notation satisfies two axioms (equivalent to the homomorphism property):
    \begin{enumerate}
        \item \textbf{Identity:} $1 \cdot x = x$ for all $x \in X$.
        \item \textbf{Compatibility:} $g_1 \cdot (g_2 \cdot x) = (g_1 g_2) \cdot x$ for all $g_1, g_2 \in G, x \in X$.
    \end{enumerate}
\end{defn}

\begin{thm}[Equivalence Relation on $X$]
    Let $G$ act on $X$. Define a relation $\sim$ on $X$ by:
    \[ x_1 \sim x_2 \iff \exists g \in G \text{ such that } g \cdot x_1 = x_2. \]
    Then $\sim$ is an equivalence relation.
\end{thm}
\begin{proof}
    We check the properties:
    \begin{enumerate}
        \item \textbf{Reflexive:} $1 \cdot x = x \implies x \sim x$.
        \item \textbf{Symmetric:} If $g \cdot x_1 = x_2$, acting by $g^{-1}$ gives $g^{-1} \cdot (g \cdot x_1) = g^{-1} \cdot x_2 \implies (g^{-1}g) \cdot x_1 = g^{-1} \cdot x_2 \implies x_1 = g^{-1} \cdot x_2$. Thus $x_2 \sim x_1$.
        \item \textbf{Transitive:} If $g \cdot x_1 = x_2$ and $h \cdot x_2 = x_3$, then $h \cdot (g \cdot x_1) = x_3 \implies (hg) \cdot x_1 = x_3$. Thus $x_1 \sim x_3$.
    \end{enumerate}
\end{proof}

\subsection{Orbits and Stabilizers}

The equivalence classes under the relation $\sim$ partition the set $X$. These classes are called \textbf{orbits}.

\begin{defn}[Orbit]
    For $x \in X$, the orbit of $x$ is the set of all places $x$ can be moved to by $G$:
    \[ \mathcal{O}_x = \text{Orb}(x) = \{g \cdot x : g \in G\}. \]
\end{defn}

\begin{defn}[Stabilizer]
    For $x \in X$, the stabilizer of $x$ is the set of elements in $G$ that fix $x$:
    \[ G_x = \Stab(x) = \{g \in G : g \cdot x = x\}. \]
    Note that $G_x$ is a subgroup of $G$ (denoted $G_x \leq G$).
\end{defn}

\begin{thm}[\textbf{Orbit-Stabilizer Theorem}]
    Let $G$ act on $X$. For any $x \in X$, there is a bijection between the orbit $\mathcal{O}_x$ and the set of left cosets $G/G_x$. Consequently:
    \[ |\mathcal{O}_x| = [G : G_x]. \]
    If $G$ is finite, $|G| = |\mathcal{O}_x| \cdot |G_x|$.
\end{thm}
\begin{proof}
    Define a map $\psi: G/G_x \to \mathcal{O}_x$ by $\psi(gG_x) = g \cdot x$.
    \begin{enumerate}
        \item \textbf{Well-defined:} Suppose $g_1 G_x = g_2 G_x$. Then $g_1^{-1}g_2 \in G_x$, so $(g_1^{-1}g_2) \cdot x = x$, implying $g_2 \cdot x = g_1 \cdot x$.
        \item \textbf{Injectivity:} If $\psi(g_1 G_x) = \psi(g_2 G_x)$, then $g_1 \cdot x = g_2 \cdot x$. Multiplying by $g_1^{-1}$, we get $x = g_1^{-1}g_2 \cdot x$, so $g_1^{-1}g_2 \in G_x$, implying $g_1 G_x = g_2 G_x$.
        \item \textbf{Surjectivity:} By definition, any $y \in \mathcal{O}_x$ is of the form $g \cdot x$ for some $g$, which is exactly $\psi(gG_x)$.
    \end{enumerate}
    Thus, $\psi$ is a bijection.
\end{proof}
% 
% The set X is partitioned into disjoint orbits. Within each orbit, elements are transitively connected by G.

\newpage
\section{Lecture 2: Cauchy's Theorem \& Conjugation (Jan 14)}

\subsection{Cauchy's Theorem}

\begin{lem}
    If $X$ is a finite set and $G$ is a $p$-group (a group of order $p^k$) acting on $X$, then:
    \[ |X| \equiv |X^{G}| \pmod{p}, \]
    where $X^G = \{x \in X : g \cdot x = x, \forall g \in G\}$ is the set of fixed points.
\end{lem}

\begin{thm}[Cauchy's Theorem]
    If $G$ is a finite group and $p$ is a prime dividing $|G|$, then $G$ has an element of order $p$.
\end{thm}
\begin{proof}[Proof (McKay's Proof)]
    Let $X$ be the set of $p$-tuples of elements of $G$ whose product is the identity:
    \[ X = \{(g_1, \dots, g_p) \in G^p : g_1 g_2 \cdots g_p = 1\}. \]
    Notice that $g_p$ is uniquely determined by the first $p-1$ elements ($g_p = (g_1 \cdots g_{p-1})^{-1}$), so $|X| = |G|^{p-1}$. Since $p \mid |G|$, we have $p \mid |X|$. \\
    Let $\Z_p$ (cyclic group of order $p$) act on $X$ by cyclic shift:
    \[ \sigma \cdot (g_1, g_2, \dots, g_p) = (g_2, g_3, \dots, g_p, g_1). \]
    This is a valid action because if $g_1 \cdots g_p = 1$, then $g_1 (g_2 \cdots g_p) = 1 \implies g_2 \cdots g_p = g_1^{-1}$, so $(g_2 \cdots g_p) g_1 = g_1^{-1} g_1 = 1$.
    
    By the Lemma, $|X| \equiv |X^{\Z_p}| \pmod{p}$.
    Fixed points are tuples $(a, a, \dots, a)$ such that $a^p = 1$.
    Since $(1, \dots, 1) \in X^{\Z_p}$, the set of fixed points is non-empty. Since $p$ divides $|X|$ and $p$ divides the congruence $|X| - |X^{\Z_p}|$, $p$ must divide $|X^{\Z_p}|$.
    Therefore, there are at least $p$ fixed points. There must exist some $a \neq 1$ such that $a^p = 1$.
\end{proof}

\subsection{Conjugation}

Conjugation is a specific action of $G$ on itself.

\begin{defn}
    For $g, h \in G$, the \textbf{conjugate} of $g$ by $h$ is $g^h \coloneqq hgh^{-1}$.
    The map $\phi_h: G \to G$ defined by $g \mapsto hgh^{-1}$ is an automorphism (an inner automorphism).
\end{defn}

\begin{defn}[Classes and Centralizers]
    Let $G$ act on itself by conjugation ($h \cdot g = hgh^{-1}$).
    \begin{itemize}
        \item The orbit of $g$ is the \textbf{Conjugacy Class} of $g$: $\text{Cl}(g) = \{hgh^{-1} : h \in G\}$.
        \item The stabilizer of $g$ is the \textbf{Centralizer} of $g$: $C_G(g) = \{h \in G : hg = gh\}$.
    \end{itemize}
    By Orbit-Stabilizer: $|\text{Cl}(g)| = [G : C_G(g)]$.
\end{defn}

\begin{defn}[Center of $G$]
    The \textbf{Center}, $Z(G)$, is the set of elements that commute with everything:
    \[ Z(G) = \{z \in G : zg = gz, \forall g \in G\} = \bigcap_{g \in G} C_G(g). \]
    $Z(G)$ is the kernel of the conjugation homomorphism $G \to \Aut(G)$. Thus $Z(G) \normal G$.
\end{defn}

\begin{defn}[Normalizer]
    Let $H \leq G$. The \textbf{Normalizer} of $H$ in $G$ is:
    \[ N_G(H) = \{g \in G : gHg^{-1} = H\}. \]
    $N_G(H)$ is the largest subgroup of $G$ in which $H$ is normal.
\end{defn}

\newpage
\section{Lecture 3: Sylow Preliminaries \& Isomorphism Theorems (Jan 16)}

\subsection{Product of Subgroups}

\begin{defn}
    Let $K, N \leq G$. Define $KN = \{kn : k \in K, n \in N\}$.
    \begin{itemize}
        \item In general, $KN$ is not a subgroup.
        \item If $N \normal G$ (or just $N \subseteq N_G(K)$), then $KN$ is a subgroup.
        \item Size formula: $|KN| = \frac{|K||N|}{|K \cap N|}$.
    \end{itemize}
\end{defn}

\begin{thm}[Second Isomorphism Theorem]
    Let $K \leq G$ and $N \normal G$. Then $K \cap N \normal K$, and
    \[ \frac{KN}{N} \cong \frac{K}{K \cap N}. \]
\end{thm}
\begin{proof}
    Consider the natural projection $\pi: G \to G/N$. Restrict it to $K$, i.e., $\phi = \pi\res_K : K \to G/N$.
    The image of $\phi$ is $\{kN : k \in K\} = KN/N$.
    The kernel of $\phi$ is $\{k \in K : kN = N\} = \{k \in K : k \in N\} = K \cap N$.
    By the First Isomorphism Theorem, $K / \ker(\phi) \cong \text{Im}(\phi)$.
\end{proof}
% 
% Visualizing K, N, KN, and K \cap N in a diamond lattice structure.

\subsection{Sylow Definitions}

\begin{defn}[$p$-group]
    Let $p$ be a prime. A group $G$ is a $p$-group if every element has order a power of $p$. For finite groups, this is equivalent to $|G| = p^k$.
\end{defn}

\begin{defn}[Sylow $p$-subgroup]
    Let $|G| = p^n m$ where $p \nmid m$. A subgroup $P \leq G$ is called a \textbf{Sylow $p$-subgroup} if $|P| = p^n$.
    \[ \text{Syl}_p(G) = \{P \leq G : P \text{ is a Sylow } p\text{-subgroup}\}. \]
\end{defn}

\newpage
\section{Lecture 4: The Sylow Theorems (Jan 21)}

\begin{thm}[\textbf{The Sylow Theorems}]
    Let $G$ be a finite group of order $p^n m$ where $p \nmid m$.
    \begin{enumerate}
        \item \textbf{Existence:} $\text{Syl}_p(G) \neq \emptyset$. (There exists a subgroup of order $p^n$).
        \item \textbf{Conjugacy:} Any two Sylow $p$-subgroups are conjugate in $G$. That is, if $P, Q \in \text{Syl}_p(G)$, $\exists g \in G$ such that $gPg^{-1} = Q$.
        \item \textbf{Number:} Let $n_p = |\text{Syl}_p(G)|$. Then:
            \begin{itemize}
                \item $n_p \equiv 1 \pmod{p}$.
                \item $n_p \mid m$ (equivalently $n_p \mid |G|$).
            \end{itemize}
    \end{enumerate}
\end{thm}
\begin{proof}
    Let $p$ be a prime such that $p \Bigm| |G|$. Define the set of all $p$-subgroups:
    \[
    \Sigma = \{ H : H \leq G, |H| = p^n \text{ for some } n > 0 \}
    .\]
    
    Define $\Omega$ as the set of maximal elements in $\Sigma$ under inclusion:
    \[
    \Omega = \{ H : H \in \Sigma, \text{ there is no } K \in \Sigma \text{ such that } H \subsetneq K \}
    .\]
    
    We let $G$ act on the set of subgroups $\{H : H \leq G\}$ by conjugation. Since conjugation is an isomorphism ($H^g \simeq H$), it preserves the order of subgroups. Thus:
    \[ H \in \Sigma \iff H^g \in \Sigma \quad \text{and} \quad H \in \Omega \iff H^g \in \Omega. \]
    Therefore, $G$ acts on $\Omega$ by $g \cdot H = H^g$.

    \subsubsection*{Claim 1: Let $H \in \Omega$. Consider the action of $H$ on $\Omega$ by conjugation. Then $H$ is the \emph{unique} fixed point of this action.}
    \textit{Proof of Claim:}
    \begin{itemize}
        \item \textbf{Existence:} It is trivial to see that $H$ is a fixed point, as $H^h = hHh^{-1} = H$ for any $h \in H$.
        \item \textbf{Uniqueness:} Let $K \in \Omega$ be a fixed point of the action of $H$.
        \[
        K \text{ is fixed by } H \iff K^h = K \quad \forall h \in H \iff H \leq N_G(K).
        \]
        Since $K \normal N_G(K)$, and $H \leq N_G(K)$, the product $HK$ is a subgroup of $N_G(K)$, and thus $HK \leq G$.
        Using our counting lemma:
        \[
        |HK| = \frac{|H||K|}{|H \cap K|}.
        \]
        Since $H, K \in \Omega$, their orders are powers of $p$. Thus $|HK|$ is a power of $p$, meaning $HK \in \Sigma$.
        
        We have $H \leq HK$. Since $H$ is maximal in $\Sigma$ (by definition of $\Omega$) and $HK \in \Sigma$, it must be that $H = HK$. Similarly, $K \leq HK$ implies $K = HK$.
        Therefore, $H = K$.
    \end{itemize}

    \subsubsection*{Claim 2: Size of $\Omega$ Modulo $p$}
    \textbf{Statement:} $|\Omega| \equiv 1 \pmod{p}$.
    
    \textit{Proof of Claim:}
    Let $H \in \Omega$. We decompose $\Omega$ into disjoint orbits under the action of $H$.
    \[
    \Omega = \{H\} \cup \bigcup_{K \neq H} \Orb_H(K).
    \]
    By Claim 1, $H$ is the only fixed point (an orbit of size 1). For any $K \in \Omega$ with $K \neq H$, the stabilizer of $K$ in $H$ is a proper subgroup, so by the Orbit-Stabilizer theorem for $p$-groups, $|\Orb_H(K)|$ is divisible by $p$.
    \[
    |\Omega| = 1 + \sum (\text{multiples of } p) \implies |\Omega| \equiv 1 \pmod{p}.
    \]

    \subsubsection*{Claim 3: Transitivity (Conjugacy of Elements in $\Omega$)}
    \textbf{Statement:} Any two elements of $\Omega$ are conjugate.\\
    \textit{Proof of Claim:}
    Suppose for contradiction that $H, K \in \Omega$ are not conjugates. Let $\theta_1 = \Orb_G(H)$ and $\theta_2 = \Orb_G(K)$. Since they are distinct orbits, $\theta_1 \cap \theta_2 = \emptyset$.
    
    Consider the action of $H$ on these sets:
    \begin{itemize}
        \item On $\theta_1$: $H \in \theta_1$. By Claim 1, $H$ is the unique fixed point of $H$ in $\Omega$. Since $\theta_1 \subseteq \Omega$, $H$ is the unique fixed point in $\theta_1$.
        Thus, $|\theta_1| \equiv 1 \pmod{p}$.
        
        \item On $\theta_2$: Since $H \in \theta_1$ and $\theta_1 \cap \theta_2 = \emptyset$, we have $H \notin \theta_2$.
        Therefore, $\theta_2$ contains \emph{no} fixed points under the action of $H$ (because the only fixed point in all of $\Omega$ is $H$).
        Thus, every orbit of $H$ inside $\theta_2$ has size divisible by $p$.
        This implies $|\theta_2| \equiv 0 \pmod{p}$.
    \end{itemize}
    
    By symmetry, we can swap the roles of $H$ and $K$. Running the same argument with $K$ acting on the sets implies $|\theta_2| \equiv 1 \pmod{p}$ and $|\theta_1| \equiv 0 \pmod{p}$.
    
    This results in a contradiction (e.g., $|\theta_1| \equiv 1$ and $|\theta_1| \equiv 0$). Thus, $\theta_1 = \theta_2$, so $H$ and $K$ are conjugate.
    
    \textit{Corollary:} Since there is only one orbit, $\Omega = \theta_1$. Thus $|\Omega| = |\theta_1| \equiv 1 \pmod{p}$.

    \subsubsection*{Claim 4: Divisibility}
    \textbf{Statement:} $|\Omega| \Bigm| |G|$.\\
    \textit{Proof of Claim:}
    Let $H \in \Omega$. Since $\Omega$ is exactly the orbit of $H$ under conjugation (from Claim 3), the size of the orbit is the index of the stabilizer:
    \[
    |\Omega| = [G : \Stab_G(H)] = [G : N_G(H)] = \frac{|G|}{|N_G(H)|}.
    \]
    Thus $|\Omega|$ divides $|G|$.

    \subsubsection*{Claim 5: Identification with Sylow Subgroups}
    \textbf{Statement:} $\Omega = Syl_p(G)$.\\
    \textit{Proof of Claim:}
    We know $Syl_p(G) \subseteq \Omega$ because Sylow subgroups are maximal $p$-subgroups by definition. We must show the reverse: every $H \in \Omega$ is a Sylow $p$-subgroup.\\ 
    Let $H \in \Omega$. Assume for contradiction that $H \notin Syl_p(G)$.
    Let $|G| = p^s m$ where $p \nmid m$. Since $H$ is a $p$-group but not Sylow, $|H| = p^t$ with $t < s$. \\
    Consider the normalizer $N_G(H)$.
    From Claim 2, we know $|\Omega| = [G : N_G(H)] \equiv 1 \pmod{p}$.
    Therefore, $p$ does \emph{not} divide $[G : N_G(H)]$.
    Since $|G| = [G : N_G(H)] \cdot |N_G(H)|$, all factors of $p$ in $|G|$ must reside in $|N_G(H)|$.
    Thus, $|N_G(H)|$ is divisible by $p^s$.\\
    Consequently, the index $[N_G(H) : H] = \frac{|N_G(H)|}{|H|} = \frac{(\text{multiple of } p^s)}{p^t}$ is divisible by $p$ (since $s > t$).\\
    By Cauchy's Theorem applied to the quotient group $N_G(H)/H$, there exists a subgroup of order $p$, say $\bar{K} \leq N_G(H)/H$.
    By the Correspondence Theorem, there exists a subgroup $K$ such that $H \leq K \leq N_G(H)$ with $|K| = p \cdot |H| = p^{t+1}$.\\
    Since $K$ is a $p$-group, $K \in \Sigma$. However, $H \subsetneq K$, which contradicts the definition of $H$ as a maximal element in $\Omega$.\\
    Therefore, $H$ must already be of order $p^s$. Thus $\Omega = Syl_p(G)$.
\end{proof}

\newpage
\section{Lecture 5: Normal Series \& Solvability (Jan 23)}

\subsection{Normal and Subnormal Series}

\begin{defn}
    A \textbf{subnormal series} of a group $G$ is a chain of subgroups:
    \[ \{1\} = H_0 \leq H_1 \leq \dots \leq H_k = G \]
    such that $H_i \normal H_{i+1}$ for all $i$.
    The quotient groups $H_{i+1}/H_i$ are called the \textbf{factors} of the series.
\end{defn}

\begin{defn}[Composition Series]
    A subnormal series is a \textbf{composition series} if all factors $H_{i+1}/H_i$ are \textbf{simple} groups (non-trivial groups with no normal subgroups other than $\{1\}$ and themselves).
\end{defn}

\begin{thm}[Jordan-Hölder]
    Any two composition series of a finite group $G$ are equivalent. That is, they have the same length, and their factors are isomorphic (up to reordering).
\end{thm}

\subsection{Solvable Groups}

\begin{defn}
    A group $G$ is \textbf{solvable} if it has a subnormal series where all factors $H_{i+1}/H_i$ are \textbf{abelian}.
\end{defn}

\begin{defn}[Commutator]
    The commutator of $g, h$ is $[g,h] = ghg^{-1}h^{-1}$.
    The \textbf{Derived Subgroup} (or Commutator Subgroup) $G'$ or $[G,G]$ is the subgroup generated by all commutators.
\end{defn}

\begin{prop}
    $G/N$ is abelian if and only if $[G,G] \leq N$.
    Thus, $[G,G]$ is the smallest normal subgroup such that the quotient is abelian.
\end{prop}

\begin{rmk}[Derived Series]
    Define $G^{(0)} = G$, $G^{(1)} = [G,G]$, and $G^{(i+1)} = [G^{(i)}, G^{(i)}]$.
    $G$ is solvable if and only if the derived series terminates at $\{1\}$ (i.e., $G^{(n)} = \{1\}$ for some $n$).
\end{rmk}
\newpage
\section{Lecture 6: Solvable and Derived Series (Jan 26)}
\begin{lem}
    Let $\alpha\in Aut(G)$ then $\alpha([g,h])=[\alpha(g),\alpha(h)]$.
\end{lem}
\begin{lem}
    For $N \normal G$ then $G/N$ is abelian iff $[G,G]\leq N$.
\end{lem}
\begin{thm}
    If $N\normal G$, $g,h\in G$ and $[gN,hN]=[g,h]N$. Additionally, $gN$ and $hN$ commutes iff $[g,h]N=N$ iff $[g,h]\subseteq N$. 
\end{thm}
\begin{defn}
    $G$ is solvable iff there is a subnormal series $\gen{H_i : 0\leq i \leq t}$ such that $[H_{i+1},H_{i+1}]\leq H_i$ for all $0\leq i \leq t-1$. 
\end{defn}
\begin{thm}
    The following are true
    \begin{enumerate}
        \item If $G$ solvable and $K\leq G$ then $K$ is solvable
        \item If $G$ is solvable and $N\normal G$ then $G/N$ is solvable. 
        \item For $N \normal G$, G solvable iff both N and $G/N$ are solvable
    \end{enumerate}
\end{thm}
\begin{proof}
    We prove the previous theorems
    \begin{enumerate}
        \item Let $\gen{H_i : 0\leq i \leq t}$ be a series in $G$ such that $[H_{i+1},H_{i+1}]\leq H_i$ for all $0\leq i \leq t-1$. \\
        We can let $H'_i=K\cap H_i$ for $0\leq i \leq t$ and we need to verify $\gen{H'_i : 0\leq i \leq t}$ is a solvable series in $K$.
        \item Let $\gen{H_i : 0\leq i \leq t}$ be a series in $G$ such that $[H_{i+1},H_{i+1}]\leq H_i$ for all $0\leq i \leq t-1$. \\
        Let $H_i'=\phi_N[H_i]=H_iN/N$. Verify it is solvable. 
        \item The forward direction is trivial by (1) and (2). For the reverse direction, $N$ solvable and $G/N$ solvable $\gen{N_i : 0\leq i \leq s}$ subnormal in $N$ and $N_{i+1}/N_i$ abelian and similar for $\gen{H_j N/N : 0\leq j \leq t}$ subnormal in $G/N$ and $(H_{j+1} N/N)_{j+1}/(H_j N/N)$ abelian. We know $H_{j+1}\normal H_{j}$ and $[H_{j+1}, H_{j+1}]\leq H_j$ for all $0\leq j \leq t-1$. 
        So $N_0=1 \normal N_1 \cdots \normal N_s=N=H_0\normal H_1 \cdots \normal H_t=G$.
    \end{enumerate}
\end{proof}
\begin{defn}
    Given $G$, the derived series of $G$ is given by $G_0'=G$ and $G_{i+1}'=[G_i',G_i']$ for $i\geq 0$.
\end{defn}
\begin{thm}
    $G$ is solvable iff there is $n$ such that $G_n'=\{1\}$.
    \begin{proof}
        For the backwards direction, by construction $G_i'/G_{i+1}'$ is abelian. Let $H_j=G_{n-j}$.
        For the forward direction, let $\gen{H_j : 0\leq j \leq n}$ be a subnormal series in $G$, $H_{j+1}/H_j$ abelian. Show by induction that $G_i'\leq H_{n-i}$ for all $i.$
    \end{proof}
\end{thm}
\begin{thm}
    Suppose $G$ abelian and simple, let $a\in G$ and $a\neq 1$ then $G=\gen{a}$ and $|a|$ is prime. 
\end{thm}
\begin{thm}
    Let $G$ be simple and let $\gen{H_i : 0\leq i \leq t}$ be subnormal series in $G$. Then there is $n<t$ such that $H_i=1$ for $0\leq i \leq n$ and $H_i=G$ for $n<i\leq t$.
\end{thm}
\newpage
\section{Lecture 7: Nilpotent Groups and Free Groups (Jan 28)}
\subsection{Nilpotent Groups}
\begin{defn}
    For $H,K\leq G$, $[H,K]=\gen{[h,k]: h\in H, k\in K}$
\end{defn}
\begin{rmk}
    $G_i'$ is characteristic in $G$ for all $i,$ in particular $G_i'\normal G$. 
\end{rmk}  
\begin{lem}
    If $N\normal G, N\leq H \leq G$ then \[
    \frac{H}{N}\leq Z \left( \frac{G}{N} \right) \iff [G,H]\leq N
    .\]
\end{lem}
\begin{defn}
    A normal series in $G$, $\gen{H_i : 0\leq i \leq t}$ is a central series if $\frac{H_{i+1}}{H_i}\leq Z \left( \frac{G}{H_i} \right)$ for all $0\leq i \leq t-1$. Equivalently, $[G,H_{i+1}]\leq H_i$ for all $0\leq i \leq t-1$.
\end{defn}
\begin{defn}
    $G$ is nilpotent iff $G$ has a central series.
\end{defn}
\begin{rmk}
    Nilpotent groups are solvable.
\end{rmk}
\begin{defn}
    Let $G$ be a group
    \begin{enumerate}
        \item The descending central series is the sequence of subgroup given by $\gamma_1(G)=G$ and $\gamma_{i+1}(G)=[G,\gamma_i(G)]$ for $i\geq 1$.
        \item The ascending central series is the sequence of subgroup given by $Z_0(G)=1$ and \[
        \frac{Z_{n+1}(G)}{Z_n(G)}= Z \left( \frac{G}{Z_n(G)} \right)
        .\] 
    \end{enumerate}
\end{defn}
\begin{rmk}
    By induction, $Z_n(G)$ and $\gamma_n(G)$ are characteristic in $G$. and $Z_n(G)\normal G$ and $\gamma_n(G)\normal G$ for all $n\geq 0$.
\end{rmk}
\begin{rmk}
    For any $G$ and $N\normal G, [G,N]\leq N$ because for any $g\in G, n\in N$ we have $[g,n]=gng^{-1}n^{-1}$ and $n^g\in N$ since $N\normal G$. So $[G,N]\leq N$.
\end{rmk}
\begin{thm}
    $G$ is nilpotent iff there is a $N\geq 0$ such that $Z_n(G)=G$ iff there is $n\geq 0$ such that $\gamma_{n+1}(G)=\{1\}$. Moreover, the least $n$ such that $Z_n(G)=G$ is the least $n$ such that $\gamma_{n+1}(G)=\{1\}$.
    \begin{proof}
    Take any central series then the decreasing and ascending central series will grow at least as fast as the arbitrary central series. 
    \end{proof}
\end{thm}
\subsection{Free Groups}
\begin{eg}
    Consider $\Z=(\mathbb{Z},+)$, $1$ is a generator $\Z=\gen{1}$. The universal property of $\Z$, 1. For any group $G$ and any $g\in G,$ there is a unique homomorphism $\phi:\Z\to G$ and $\phi(1)=g$. 
    \begin{proof}
        If $\phi$ exist let $\phi(n)=g^n$ then $\phi$ is a homomorphism. 
    \end{proof}
    \noindent $g$ is free because whatever $1$ is mapped to in the homomorphism it won't mess you up compared to a finite group. 
\end{eg}
\begin{defn}
    $F$ is a free group on $2$ element if \begin{enumerate}
        \item There exist $a,b\in F$, $F=\gen{a,b}$
        \item For all $G,$ all $g,h\in G$ there is a unique homomorphism $\phi:F\to G$ such that $\phi(a)=g$ and $\phi(b)=h$. 
    \end{enumerate}
\end{defn}
\newpage
\section{Lecture 8: Free Groups}
\begin{thm}
    For any set $X,$ there are a group $F$ and an injective function $i:X\to F$ such that \begin{enumerate}
        \item For any group $G$ and any function $f:X\to G$ there is a unique homomorphism $\phi:F\to G$ such that $\phi(i(x))=f(x)$ for all $x\in X$.
        \begin{center}
            \begin{tikzcd}
                X \arrow[r, "i", hook] \arrow[rd, "f"'] & F \arrow[d, "\phi", dashed] \\
                & G
            \end{tikzcd}
        \end{center}
        The diagram commutes: $\phi \circ i = f$. The homomorphism $\phi$ is unique.
        \item $F$ is generated by $i[X]$
    \end{enumerate}
    \begin{proof}
        A word (on alphabet $X$) is a (possibly empty) finite sequence from $X \times \Z$. We will write $x_1^{n_1}x_2^{n_2}\cdots x_k^{n_k}$ for the word $x_1,x_2,\ldots,x_k$ for $\gen{(x_1,n_1),(x_2,n_2),\ldots,(x_k,n_k)}$. 
        \begin{eg}
            Let $X= \{x,y,z\}$ then $w=x^0 x^{15}y^{-23}z^{27}$ is a word.
        \end{eg}
        \noindent Let $w$ be the set of words. Given a word, we can possibly perform a reduction step 
        \begin{enumerate}
            \item If word contains an entry $\gen{x, 0}$, remove it.
            \item If the word contains successive entries $\gen{x, n}$ and $\gen{x, m}$ then replace them with $\gen{x, n+m}$. 
        \end{enumerate}
        \begin{eg}
            $y^{15}x^3x^{-2}x^{-1}y^2z^5\to y^{15}x^1x^{-1}y^2z^5\to y^{15}x^0y^2z^5\to y^{15}y^2z^5\to y^{17}z^5$.
        \end{eg}
        \noindent Let $R$ be set of all reduced words, where no more reductions can be done. \\
        For any word $w,$ there is a unique reduced word $w'$ such that $w$ can be transformed into $w'$ by a finite set of reductions. \\
        We will not be bale to show this property using group properties but instead we'll produce a group $F$ with the property that every word is equivalent to a unique reduced word. \\\\
        Given $x\in X,$ I will replace $f_{x}:F\to F$
        \begin{enumerate}
            \item $w$ does not begin with some $x^m$ then $f_x(w)=xw$
            \item $w$ is of the form $x^{-1}v$ then $f_x(w)=v$
            \item $w$ is of the form $x^nv$ for $n\neq -1$ then $f_x(w)=x^{n+1}v$
        \end{enumerate}
        I also define $g_x:R\to R$ by \begin{enumerate}
            \item $w$ does not begin with some $x^n$ then $g_x(w)=x^{-1}w$
            \item $w$ is of the form $x^1 v$ then $g_x(w)=v$
            \item $w$ is of the form $x^nv$ for $n\neq 1$ then $g_x(w)=x^{n-1}v$
        \end{enumerate}
        We can observe that $f_x \circ g_x = g_x \circ f_x= 1_R$. \\
        For each $x\in X$, $f_x\in \sum_R$ (the group of permutations of $R$) and $g_x=f_x^{-1}$. Let $F$ be the subgroup of $\sum_R$ generated by $ \{f_x : x\in X\}$ or $ \{f_{x_1}^{n_1},...,f_{x_t}^{n_t}: x_i\in X, n_i\in \Z\}$. \\
        Let $i:X\to F$ with $i(x)=f_x$. For any word $w=x_1^{n_1}x_2^{n_2}\cdots x_t^{n_t}$. Let $f_w=f_{x_1}^{n_1}f_{x_2}^{n_2}\cdots f_{x_t}^{n_t}\in F.$\\
        Key Facts: \begin{enumerate}
            \item If $w$ is obtained from $\overline{w}$ by a single reduction step, $f_w=f_{\overline{w}}$.
            \item If $w$ is obtained from $\overline{w}$ by any sequence of reduction steps, then $f_w=f_{\overline{w}}$.
            \item If $w\in R$ then $f_w(\gen{})=w$
            \item For any word $w, f_w(\gen{})$ the unique reduced word to which $w$ can be transformed by reduction step.
            \item If $w_1,w_2$ are reduced words then $v=$ unique reduction of $w_1w_2,$ $f_v=f_{w_1}f_{w_2}$
        \end{enumerate}
        Let $F'$ be the group whose underlying set is $R$ whose operation is "concatenate and reduce". Then $F \simeq F'$ by $f_v \mapsto v$. To finish, let $G$ be a group, $h:X\to G$ function. We want to find there is a unique homomorphism $\phi$ with $F'\to G$ and $\phi \circ i = h.$
        \begin{enumerate}
            \item If $\phi$ exists then \begin{align*}
                \phi(x_1^{n_1}x_2^{n_2}\cdots x_t^{n_t})&=\phi(i(x_1))^{n_1}\phi(i(x_2))^{n_2}\cdots \phi(i(x_t))^{n_t}\\
                &=h(x_1)^{n_1}h(x_2)^{n_2}\cdots h(x_t)^{n_t}
            \end{align*}
            \item Verify $\phi$ is a homomorphism.
        \end{enumerate}  
    \end{proof}
\end{thm}
\begin{eg}
    Let $X=\gen{x,y}$, let $G$ be any group with $2$ generators say $G=\gen{a,b}$. Let $F$ be a free group on $X$. Let $\phi: F\to G$ be unique homomorphism such that $\phi(x)=a$ and $\phi(y)=b$. $\phi$ is surjective because $G$ is generated by $a$ and $b$. Thus, $G \simeq \frac{F}{\ker{\phi}}$.
\end{eg}
\newpage
\section{Lecture 9: Generators and Relations (Feb 2)}
\begin{defn}
    let $X$ be a set, let $F$ be a free group on $X$. Let $A \subseteq F$ and $N=$ least normal subgroup of $F$ containing $A$. 
\end{defn}
\begin{thm}
    Let $\overline{F}=F/N$ and $j:X\to \overline{F}$ by $j(x)=xN$.  \\
    \begin{center}
        \begin{tikzcd}
            F \arrow[r, "\phi_N"] & \overline{F} \\
            X \arrow[u, "i"] \arrow[ru, "j"]
        \end{tikzcd}
    \end{center}
    The following are true:
    \begin{enumerate}
        \item $\overline{F}$ is generated by $ \{j(x): x\in X\}.$ Let $\overline{x}=j(x)$
        \item If $a=x_1^{n_1}x_2^{n_2}\cdots x_t^{n_t}\in A$ then $a\in N$ and $\overline{x_1}^{n_1}\overline{x_2}^{n_2}\cdots \overline{x_t}^{n_t}=\phi_N(a)=1_{\overline{F}}$
    \end{enumerate}
    \noindent "Universal Property of $\overline{F}$": Let $G$ be a group and let $h:X\to G$ such that $a=x_1^{n_1}x_2^{n_2}\cdots x_t^{n_t}\in A$ and $h(x_1)^{n_1}h(x_2)^{n_2}\cdots h(x_t)^{n_t}=1_G$. There is an unique homomorphism $\mathcal{T}:\overline{F}\to G$ such that 
    \begin{center}
        \begin{tikzcd}
            X \arrow[r, "j"] \arrow[rd, "h"'] & \overline{F} \arrow[d, "\mathcal{T}", dashed] \\
            & G
        \end{tikzcd}
    \end{center}
    So $h=\mathcal{T}\circ j$
    \begin{proof}
        Since $ \{\overline{x}: x\in X\}$ generates $\overline{F}$ there is at most one $\tau$. By universal property of $F$, there is unique homomorphism $\phi:F\to G$ such that \[
        \phi(x^1)=h(x) \tag{For all $x\in X$}
        .\] 
        Use property of $F$
        \begin{center}
            \begin{tikzcd}
                X \arrow[r, "i"] \arrow[rd, "h"'] & F \arrow[d, "\phi", dashed] \\
                & G
            \end{tikzcd}
        \end{center}
        Claim: By hypothesis on $h$, $A\subseteq \ker(\phi)$.\\
        Let $a=x_1^{t_1}x_2^{t_2}\cdots x_k^{t_k}\in A$ then $\phi(a)=\phi(x_1)^{t_1}\phi(x_2)^{t_2}\cdots \phi(x_k)^{t_k}=h(x_1)^{t_1}h(x_2)^{t_2}\cdots h(x_k)^{t_k}=1_G$ since $h(x_1)^{t_1}h(x_2)^{t_2}\cdots h(x_k)^{t_k}=1_G$. Thus, $a\in \ker(\phi)$.\\\\
        By choice of $N$, $N\leq \ker(\phi)$, we'll attempt to define $\tau:\overline{F}\to G$ by $\tau(wN)=\phi(w)$ for $w\in F$. \\
        $\tau$ is well defined because $w_1N=w_2N\implies w_1^{-1}w_2\in N \implies w_1^{-1}w_2\in \ker(\phi)\implies \phi(w_1)^{-1}\phi(w_2)=1\implies \phi(w_1)=\phi(w_2)$. Showing $\tau$ is a homomorphism is straightforward.
        \begin{center}
            \begin{tikzcd}
                X \arrow[r, "i"] \arrow[rd, "h"'] & F \arrow[d, "\phi_N"] \arrow[rd, "\phi"'] \\
                & \overline{F} \arrow[r, "\mathcal{T}", dashed] & G
            \end{tikzcd}
        \end{center}
     \end{proof}
\end{thm}
\newpage
\section{Lecture 10: Presentation of Groups (Feb 4)}
\begin{eg}
    Dihedral Group: $D_n=$ symmetry group of a regular $n$-gon. \\
    Let $D_n=\gen{a,b}$ where $a=$ rotation by $\frac{2\pi}{n}$ and $b=$ reflection about the vertical axis. Then $D_n$ is generated by $a$ and $b$ with the relations $a^n=1$ and $b^2=1$. \\
    We have $bab^{-1}a=1$ \\
    Let $X= \{x,y\}$ and $F$ be free group and $A= \{x^n, y^2, yxy^{-1}x\}$ and $N=$ least normal subgroup $\supseteq A$. Let $\overline{x}=xN$ and $\overline{y}=yN$. Then $F/N=\gen{\overline{x},\overline{y}}$ and \[
    \overline{yxy^{-1}x}=\overline{y}\overline{x}\overline{y}^{-1}\overline{x}^{-1}=1_{\overline{F}}
    .\] 
    There is a unique homomorphism $\tau:F/N\to D_n$ such that $\tau(\overline{x})=a$ and $\tau(\overline{y})=b$ as $D_n=\gen{a,b}$ so $\tau$ is surjective. So $|F/N|\geq 2n$. \\
    Claim: $|F/N|\leq 2n$ (so $\tau$ is isomorphism)\\
    Proof: Using relations from before, we can simply write any expression in $\overline{x},\overline{y}$ down to $\overline{x}^s\overline{y}^t$ for $0\leq s \leq n$ and $0\leq t \leq 2$. 
\end{eg}
\newpage
\section{Lecture 11: Category Theory (Feb 6)}
\subsection{Categories}
\begin{eg}
    \begin{enumerate}
        \item Category of sets, if we have two sets $A$ and $B$ then $A\to B$ is a function $f:A\to B$
        \item Category of groups, if we have two groups $G$ and $H$ then $G\to H$ is a homomorphism $\phi:G\to H$
        \item Category of topological spaces, if we have two topological spaces $X$ and $Y$ then $X\to Y$ is a continuous function $f:X\to Y$
    \end{enumerate}
\end{eg}
\begin{defn}
    A category $\mathcal{C}$ consists of the following data:
    \begin{enumerate}
        \item A collection of objects $Ob(\mathcal{C})$
        \item A collection of morphisms $Mor(\mathcal{C})$
        \item For each morphism $f$, there are objects $\dom(f)$ and $\cod(f)$ such that $dom(f)\to cod(f)$ is a morphism in $Mor(\mathcal{C})$
        \item A composition operation $Mor(\mathcal{C})\times Mor(\mathcal{C})\to Mor(\mathcal{C})$
        \item An identity morphism $1_A:A\to A$ for each object $A\in Ob(\mathcal{C})$
    \end{enumerate}
\end{defn}
\begin{lem}
    The following are true:
    \begin{enumerate}
        \item If $a \xrightarrow{f} b$ then $f 1_a = f$ and $1_b f = f$.
        \item If $a \xrightarrow{f} b \xrightarrow{g} c \xrightarrow{h} d$ then $h(gf)=(hg)f$
    \end{enumerate}
\end{lem}
\begin{eg}
    Posets: $\mathbb{P}$ is a poset with $\leq$ binary relation such that \begin{enumerate}
        \item $a\leq a$ for all $a\in \mathbb{P}$
        \item $a\leq b$ and $b\leq a\implies a=b$
        \item $a\leq b$ and $b\leq c\implies a\leq c$
    \end{enumerate}
    Given poset $\mathbb{P}$, make "poset category" with objects the elements of $\mathbb{P}$. There is only one morphism $a\to b$ if $a\leq b$ and no morphism if $a\not\leq b$.
\end{eg}
\begin{defn}
    Let $\mathcal{C}$ be a category. An object is an initial if for all object $b$ there is a exactly one arrow $a\to b.$
\end{defn}
\begin{defn}
    An arrow $a \xrightarrow{f} b$ in $\mathcal{C}$ is an isomorphism iff there is another $b \xrightarrow{f} a$ such that $gf=1_b$ and $fg=1_a$.
\end{defn}
\begin{lem}
    If $a,b$ are initial objects in $\mathcal{C}$ they are uniquely isomorphic.
\end{lem}
\begin{proof}
    Let $a \xrightarrow{f} b$ and $b \xrightarrow{g} a$ be unique arrows from $a$ to $b$ and $b$ to $a$ respectively. Then we have $a \xrightarrow{gf} a$ and $b \xrightarrow{fg}$ so $gf=1_a$ and $fg=1_b$. 
\end{proof}
\begin{defn}
    Object $b$ is terminal iff for all $a$ there is a unique $a\to b$ iff $b$ is initial in $\mathcal{C}^{op}$
\end{defn}
\begin{lem}
    Two terminal objects are uniquely isomorphic. 
\end{lem}
\begin{defn}
    Let $\mathcal{C}$ and $\mathcal{D}$ be categories. A functor from $\mathcal{C}$ to $\mathcal{D}$ is $\mathcal{C} \xrightarrow{F} \mathcal{D}$. Assign to each object $a$ of  $\mathcal{C}$ an object $F(a)$ of $\mathcal{D}$. For each arrow $a \xrightarrow{f} b$ there is an arrow $F(a)\xrightarrow{F(f)} F(b)$ with $f(1_a)=1_{F(a)}$. Additionally, $F(gf)=F(g)F(f)$
\end{defn}
\newpage
\section{Lecture 12: Ring Theory (Feb 8)}
\begin{defn}
    A ring is a set $R$ with two binary operations $+$ and $\times$ such that \begin{enumerate}
        \item $(R,+)$ is an abelian group
        \item $(R,\times)$ is associative
        \item $a\times (b+c)=a\times b+a\times c$ and $(b+c)\times a=b\times a+c\times a$ for all $a,b,c\in R$
    \end{enumerate}
\end{defn}
\begin{eg}
    The following are rings
    \begin{enumerate}
        \item $\mathbb{Z}, 2\mathbb{Z}\mathbb{Q}, \mathbb{R}, \mathbb{C}$
        \item $M_2(\mathbb{Z})$
        \item $ \{f: \R\to \R \text{ continuous}\}$
        \item $ \{f: \R\to \R \text{ differentiable}\}$
    \end{enumerate}
\end{eg}
\begin{defn}
    $R$ is commutative iff $rs=sr$ for all $rs\in R$
\end{defn}
\begin{defn}
    $R$ is unital iff there exists $1_R\in R$ such that $1_R r=r1_R=r$ for all $r\in R$
\end{defn}
\begin{defn}
    Let $R,S$ be rings. A ring homomorphism from $R$ to $S$ if $\phi:R\to S$ satisfy:
    \begin{enumerate}
        \item $\phi(r+s)=\phi(r)+\phi(s)$ for all $r,s\in R$
        \item $\phi(rs)=\phi(r)\phi(s)$ for all $r,s\in R$
        \item $\phi(1_R)=1_S$
    \end{enumerate}
\end{defn}
\begin{defn}
    Let $S$ be a subring of $R$ if $S$ is a ring and the inclusion map $i:S\to R$ is a ring homomorphism.
\end{defn}
\begin{defn}
    Let $\phi:R\to S$ be a ring homomorphism then 
    \begin{enumerate}
        \item $\ker(\phi)=\{r\in R: \phi(r)=0\}$ is a subring of $R$
        \item $\phi(R)=\{s\in S: \exists r\in R, \phi(r)=s\}$ is a subring of $S$
    \end{enumerate}
\end{defn}
\begin{rmk}
    In general, not the case that $\ker(\phi)$ is a subring of $R$. 
\end{rmk}
\begin{defn}
    Let $R$ be a ring. An ideal of $R$ is a subring $I$ of $R$ such that
    \begin{enumerate}
        \item $I\leq (R,+)$
        \item For all $a\in R$ and $b\in I$, $ab\in I.$ 
    \end{enumerate}
\end{defn}
\begin{defn}
    Let $I$ be an ideal of $R$ then we define $r+I=I+r$
\end{defn}
\newpage
\section{Lecture 13: Ideals and Quotient Rings (Feb 10)}
\begin{thm}
    First Isomorphism Theorem of Rings: Let $R,S$ be rings and $\phi:R\to S$ be a homomorphism with $\ker(\phi)$ an ideal of $R$ and $\im(\phi)$ a subring of $S$. Then we get isomorphism \[
    \tau: \frac{R}{\ker(\phi)}\xrightarrow{\cong} \im(\phi) \text{ and } \tau:r+\ker(\phi)\mapsto \phi(r) 
    .\]  
\end{thm}
\begin{thm}
    The following are true for any ideals $I,J$ of $R$:
    \begin{enumerate}
        \item $I+J=$ smallest ideal containing $I$ and $J$
        \item $I\cap J=$ largest ideal contained in $I$ and $J$
        \item $IJ= \left\{\sum_{i=1}^nr_i s_i | n\in \N, r_i\in I, s_i\in J\right\}\subseteq I\cap J$ 
        \item Ideals of $R/I$ are in bijection with ideals of $R$ that contain $I$ 
    \end{enumerate}
\end{thm}
\begin{defn}
    $R$ is a zero ring if $R=\gen{0_R}$\\
    Note: If $1_R=0_R$ then $r\cdot 1=r \cdot 0 = 0$ and $R$ is a zero ring.
\end{defn}
\begin{defn}
    $R$ is an integral domain iff 
        \begin{enumerate}
            \item $1_R\neq 0_R$ ($R$ is not a zero ring)
            \item For all $a,b\in R$ \[
            ab=0 \implies a=0 \text{ or } b=0
            .\] 
        \end{enumerate}
\end{defn}
\begin{defn}
    $R$ is a field iff \begin{enumerate}
        \item $1_R\neq 0_R$
        \item Every element of $R\setminus \{0_R\}$ has a multiplicative inverse
    \end{enumerate}
\end{defn}
\begin{ex}
    If $R$ is a field then $R$ is an integral domain.
\end{ex}
\begin{defn}
    Let $R$ be a ring. Then $u\in R$ is a unit iff $u$ has a multiplicative inverse. \\
    We let \[
        U(R)=\{u\in R: u \text{ is a unit}\}
    .\] 
    $U(R)$ is a group under multiplication.
\end{defn}
\begin{defn}
    An ideal $I$ of $R$ is principal iff $I=aR$
\end{defn}
\begin{lem}
    Let $R$ be any ring then $R$ is the largest ideal and $ \{ 0_R\}$ is the smallest ideal. 
\end{lem}
\begin{rmk}
    We have $R/R=\{0_R\}$ and $R/\{0_R\}\simeq R$
\end{rmk}
\begin{thm}
    Let $R$ be a field and let $I$ be an ideal then $I=0$ or $I=R$
\end{thm}
\begin{proof}
    Assume $I\neq 0$, let $a\in I$ and $a\neq 0$. There exist $b\in R$ such that $ab=1\in I$. In any ring $1\in I$ iff $I=R$. \\
\end{proof}
\begin{thm}
    If $R\neq 0$ are the only two ideals of $R$ then $R$ is a field. 
\end{thm}
\begin{proof}
    Let $a\in R, a\neq 0$ then $(a)\neq 0$ as $a\in (a)$ so $(a)=R$ so $1\in (a)$ and $1=ab$ for some $b.$  
\end{proof}
\begin{defn}
    An ideal $I$ of $R$ is maximal iff 
    \begin{enumerate}
        \item $I\neq R$
        \item For all ideals $J\supseteq I$, $J=I$ or $J=R$ 
    \end{enumerate}
    $I$ is maximal in poset $ \{J: J \text{ ideal}, J\neq R\}$ ordered by $\subseteq$
\end{defn}
\begin{lem}
    If $I$ is an ideal of $R$, \[
    I \text{ maximal} \iff R/I \text{ is a field}
    .\] 
\end{lem}
\begin{proof}
    $R\neq I \iff 1_{R/I}\neq 0_{R/I}$. By $R$ maximal iff there are only two $2$ ideals of $R/I$ which are $0$ and $R/I$. 
\end{proof}
\begin{defn}
    Ideal $I$ is prime iff 
    \begin{enumerate}
        \item $I \neq R$
        \item For all $a,b\in R$, $ab\in I \implies a\in I \text{ or } b\in I$
    \end{enumerate}
\end{defn}
\begin{lem}
    $R/I$ is an integral domain iff $I$ is a prime ideal.
\end{lem}
\begin{defn}
    $ \text{Spec}(R)=\{I: I \text{ is a prime ideal of } R\}$
\end{defn}
\begin{thm}
    Let $R$ be a ring. $I$ is an ideal of $R$, $I\neq R$. Then there is a maximal ideal $J$ such that $I\subseteq J.$ 
\end{thm}
\newpage
\section{Lecture 14: Zorn's Lemma and Modules (Feb 13)}
\begin{lem}
    Zorn's Lemma: If $\mathbb{P}$ is a poset such that every chain in $\mathbb{P}$ has an upper bound, then for every element $p$ of $\mathbb{P}$ there is a $q\geq p$ with $q$ maximal. 
\end{lem}
\begin{eg}
    Claim: $\Q$ satisfies the hypothesis of Zorn's Lemma.
\end{eg}
\begin{proof}
    Let $(I_a:a\in A)$ be a chain in $\Q$. That is, $I_a$ is ideal with $I_a\neq R$ and for $a,b\in A, I_a\subseteq I_b$ or $I_b \subseteq I_a$. We need upper bound, let \[
    J=\bigcup_{a\in A}I_a= \{r: \text{there is }a\in A, r\in I_a\}
    .\] 
    Verify $J$ is an ideal and for $r,s\in J$ then $r\in I_a$ and $s\in I_b$. WLOG let $I_a \subseteq I_b$ so $r,s\in I_b$ so $r+s\in I_b\subseteq J$. \\
    Verify $J\neq R$, easy as $1\notin I_a$ all $a$ so $1 \notin J$. 
\end{proof}
\subsection{Modules}
\begin{defn}
    Let $R$ be a ring. A $R$-module is an abelian group $(M,+)$ equipped with a function $R\times M\to M$ by $(r,m)\mapsto rm$ such that 
    \begin{enumerate}
        \item $0_Rm=0_M, 1m=1$ for all $m\in M$
        \item $(r_1+r_2)=r_1m+r_2m$ and $m(r_1+r_2)=mr_1+mr_2$
        \item $r(sm)=(rs)m$ for $r,s\in R$ and $m\in M$
    \end{enumerate}
\end{defn}
\begin{defn}
    If $M$ is a $R$-module, we can form "linear combinations" \[
    \sum_{i=1}^n r_i m_i \text{ for $r_i\in R$ and $m_i\in M$}
    .\] 
\end{defn}
\begin{eg}
    Let $R=\Z$ and $(G,+)$ be an abelian group. Then $G$ is a $\Z$-module.
\end{eg}
\begin{defn}
    If $M$ is a $R$-module, a submodule of $M$ is $N$ such that $N\leq (M,+)$, $rn\in N$ for $r\in R$ and $n\in N$. 
\end{defn}
\begin{rmk}
    If $R$ is a ring, we can view $R$ as a $R$-module. The ideals of $R$ are the submodules of $R$ (viewed as a $R$-module). 
\end{rmk}
\begin{defn}
    Let $M,N$ be $R$-modules. A $R$-module homomorphism is a function $\phi:M\to N$ such that \begin{enumerate}
        \item $\phi$ is a group homomorphism for $(M,+)$ to $(N,+)$
        \item $\phi(rm)=r\phi(m)$ for $r\in R$ and $m\in M$
    \end{enumerate}
    Note: This is like linear transformation in linear algebra. 
\end{defn}
\newpage
\section{Lecture 15: Modules (Feb 16)}
\begin{defn}
    Let $M$ and $N$ be $R$-modules then we say $\phi:M\to N$ is $R$-linear if \[
    \phi\left(\sum_{i=1}^n r_im_i\right)=\sum_{i=1}^nr_i \phi(m_i) \text{ for } r_i\in R \text{ and } m_i\in M
    .\] 
\end{defn}
\begin{defn}
    Let $M\leq N$ ($M$ is submodule of $N$), $N/M=\{n+M: n\in N\}$. We define \begin{enumerate}
        \item $n_1+M+n_2+M=(n_1+n_2)+M$
        \item $r(n+M)=(rn)+M$
    \end{enumerate}
    We can define the $R$-linear quotient map $\phi_M:N\to N/M$ by $\phi:n\mapsto n+M$ with $\ker(\phi_M)=M$.
\end{defn}
\begin{thm}
    First Isomorphism Theorem for Modules: Let $\phi:M_1\to M_2$ linear, then \[
    \im(\phi)\simeq \frac{M_1}{\ker(\phi)}
    .\] 
    via $m+\ker(\phi)\mapsto \phi(m)$
\end{thm}
\begin{defn}
    Given $R$-module $M$ and set $X\subseteq M$, the least submodule containing $X$ is \[
    \Span(X)= \left\{\sum_{i=1}^n r_i x_i | n\in \N, r_i\in R, x_i\in X\right\}
    .\] 
    We specifically define $\Span(\emptyset)= \{0_M\}$
\end{defn}
\begin{defn}
    If $R$ is a ring, $X\subseteq R$ we write $\gen{X}$ for the least ideal containing $X$ \[
    \gen{X}= \left\{\sum_{i=1}^n r_i x_i | n\in \N, r_i\in R, x_i\in X\right\}
    .\] 
\end{defn}
\begin{defn}
    A ring $R$ is a principal ideal ring iff all ideals of $R$ are principal (a principal ideal is of form $Ra$ for some $a\in R$).
\end{defn}
\begin{defn}
    $R$ is a principal ideal domain iff $R$ is integral domain and principal ideal ring. 
\end{defn}
\begin{defn}
    A ring $R$ is noetherian iff every ideal of $R$ is finitely generated. (An ideal is finitely generated if $I=Ra_1+\cdots+Ra_n$ for some $a_1, \ldots, a_n\in R$.)
\end{defn}
\begin{thm}
    If $I$ is an ideal of $R$ the ideals $R/I$ are in bijection with $ \left\{J: J \text{ideal of } R, J\supseteq I \right\}$.\\
    Also prime ideals of $R/I$ correspond to prime ideals of $R$ that contain $I$. 
\end{thm}
\begin{eg}
    a
\end{eg}
\end{document}