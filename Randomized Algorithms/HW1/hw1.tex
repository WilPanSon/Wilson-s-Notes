\documentclass[a4paper]{article}
\usepackage[margin=1in]{geometry}
\usepackage[english]{babel}
\usepackage[utf8]{inputenc}
\usepackage{amsmath}
\usepackage{graphicx}
\usepackage{amssymb}
\usepackage{amsthm}
\usepackage{tikz-cd}
\usepackage{mathrsfs}
\usepackage[colorinlistoftodos]{todonotes}
\usepackage{enumitem}
\usepackage{yfonts}
\usepackage{dsfont}
\usepackage{mathtools}
\usepackage{hyperref}

\DeclarePairedDelimiter\ceil{\lceil}{\rceil}
\DeclarePairedDelimiter\floor{\lfloor}{\rfloor}

\title{15-756 Homework 1}
\author{Wilson Pan}
\date{\today}

\newtheorem{thm}{Theorem}[section]
\newtheorem{lem}[thm]{Lemma}
\newtheorem{defn}[thm]{Definition}
\newtheorem{eg}[thm]{Example}
\newtheorem{ex}[thm]{Exercise}
\newtheorem{conj}[thm]{Conjecture}
\newtheorem{cor}[thm]{Corollary}
\newtheorem{claim}[thm]{Claim}
\newtheorem{rmk}[thm]{Remark}

\newcommand{\ie}{\emph{i.e.} }
\newcommand{\cf}{\emph{cf.} }
\newcommand{\into}{\hookrightarrow}
\newcommand{\dirac}{\slashed{\partial}}
\newcommand{\R}{\mathbb{R}}
\newcommand{\C}{\mathbb{C}}
\newcommand{\Z}{\mathbb{Z}}
\newcommand{\N}{\mathbb{N}}
\newcommand{\Q}{\mathbb{Q}}
\newcommand{\LieT}{\mathfrak{t}}
\newcommand{\T}{\mathbb{T}}
\newcommand{\A}{\mathds{A}}
\newcommand{\HG}{\mathcal{H}}
\newcommand{\F}{\mathbb{F}}
\newcommand{\poly}[2]{\text{Poly}_{#1}(#2)}
\newcommand{\gen}[1]{\langle #1 \rangle}
\newcommand{\Hom}{\text{Hom}}
\newcommand{\E}{\mathbb{E}} 
\begin{document}

\maketitle
\section*{Problem 1}
Let $x_i,x_j$ be the amounts in the envelopes. WLOG let $w_i<w_j$  then the algorithm Bob will use is if he sees $k$ in the envelope he draws then he swaps with probability $1-\frac{k}{n}$ and stays with $\frac{k}{n}$. Then the probability he wins using law of total probability is 
\begin{align*}
    \Pr( \text{Wins})&=\Pr( \text{Wins} | \text{Draws }x_i)\Pr(\text{Draws }x_i)  + \Pr( \text{Wins} | \text{Draws }x_j)\Pr(\text{Draws }x_j) \\
    &= \left( 1- \frac{x_1}{n} \right)\cdot \frac{1}{2}+ \left( \frac{x_j}{n} \right)\cdot \frac{1}{2}\\
    &=\frac{1}{2} \left( 1+ \frac{x_j-x_i}{n} \right)\\
    &=\frac{1}{2}+\frac{1}{2}\cdot \left( \frac{x_j-x_i}{n} \right)\\
    &=\frac{1}{2}+\Omega\left (\frac{1}{n}\right )
\end{align*}
The last line as $x_j-x_i\geq 1$ so $\frac{x_j-x_i}{n}\geq \frac{1}{n}$.
\newpage
\section*{Problem 2}
We can write the difference in times as a sum of two differences.
\[
t_3^{(i)}-t_1^{(i)}=t_3^{(i)}-t_2^{(i)}+t_2^{(i)}-t_1^{(i)}
.\] 
For scenario 1, $t_2^{(1)}-t_1^{(1)}=0$. Using the fact that the expected value of the next bus arriving in $1$ minute then $\E[t_3^{(1)}-t_2^{(1)}]=1$ so we can conclude that\[
\E\left [t_3^{(1)}-t_1^{(1)}\right ]=1
.\] 
For scenario 2, by linearity of expectation 
\[
\E\left [t_3^{(2)}-t_1^{(2)}\right ]=\E\left [t_3^{(2)}-t_2^{(2)}+t_2^{(2)}-t_1^{(2)}\right ]=\E\left [t_3^{(2)}-t_2^{(2)}\right ]+\E\left[t_2^{(2)}-t_1^{(2)}\right ]
.\] 
Let $A=t_2^{(2)}-t_1^{(2)}$ and $B=t_3^{(2)}-t_2^{(2)}$. \\
Letting $N(t)$ be the number of buses that arrived in an interval of length $t$. Then \[
\Pr(N(t)=k)= \frac{e^{-\lambda t}(\lambda t)^k}{k!}
.\] 
We want the case when $k=0$ so \[
\Pr(N(t)=0) = e^{-\lambda t}
.\] 
This is equivalent to $\Pr(A>t)$ as this is the probability no buses arrived in the last $t$ minutes. So \[
\Pr(A>t)=e^{-\lambda t}
.\] 
So $A$ follows exponential distribution with rate $\lambda$ where $\lambda=1$ so $\E[A]=1$\\\\
The second term $\E[B]$ is the expected time since the \textit{previous} bus. Since the bus arrivals are a Poisson process with rate $\lambda=1$, the inter-arrival times are exponentially distributed. By the memoryless property of the exponential distribution, the time since the last event is also exponentially distributed with mean $1/\lambda = 1$.
Thus \[
E[A+B]=2
.\]  
Thus the two scenarios are different. The two quantities are not equal as a random point in time is more likely to fall in a longer interval than a shorter one, biasing the expected interval length upwards.

\newpage
\section*{Problem 3}
\subsection*{Part 1}
Consider dividing the steps into "blocks" of length $a$ so from $[1,a], [a+1,2a],...$. If Alice ever moves $a$ consecutive steps to the right then she will move a distance of $+a$ which guarantees she reaches the $a$ as otherwise she would've reached $0$ already. The probability of stepping right $a$ times in a row is \[
\epsilon = \left(  \frac{1}{2} \right)^a
.\] 
Regardless of where Alice starts in a block,  there is always at least $\epsilon$ probability she reaches $a$ as she can always touch $0$ before reaching $a$. For Alice to stay inside the interval forever, we must have that she fails to exit every single time so probability of not exiting at any block is at most $1-\epsilon.$ 
So for her to not exit by block $n$, it is at most $(1-\epsilon)^n$. So 
\[
\Pr( \text{Alice never hits $a$ or $0$})\leq \lim_{n\to \infty}(1-\epsilon)^n=0
.\] 
Consequently, \[
\Pr( \text{Alice eventually hits $a$ or $0$})=1-\Pr( \text{Alice never hits $a$ or $0$})\geq 1
.\]
So \[
\Pr( \text{Alice eventually hits $a$ or $0$})=1
.\] 
\subsection*{Part 2}
So we'll continue by induction on $i$.\\
(Base Case) When $i=1$ then there $p_2=\frac{1}{2}$ so it holds.\\
(Induction Step) Assume it holds for $i$ then for $i+1$, the probability she reaches $\frac{1}{2^i}$ before $0$ by our induction hypothesis was $2^i$ and from there is equal probability of reaching $2^{i+1}$ before $0$ as she has no preference for direction at any point.\\
So by law of total probability 
\begin{align*}
    p_{i+1}&=\Pr( \text{Reaches $2^{i+1}$ before $0 |$ At $2^i$})\cdot \Pr( \text{Reaches $2^{i}$ before $0$})\\&+\Pr( \text{Reaches $2^{i+1}$ before $0 |$ At $0$})\Pr( \text{Reaches $0$ before $2^i$})\\
    &=\frac{1}{2}p_i+0\cdot (1-p_i)\\
    &=\frac{1}{2^{i+1}}
\end{align*} 
So it holds by induction.
\subsection*{Part 3}
Define $E_i$ to be the event that reaches $2^i$ before $0$ then for Alice to never hit $0$, we must $E_i$ occur for all $i\in \N$. \[
\Pr(E_i)=\frac{1}{2^i} \tag{Part 2}
.\] 
If $E_k$ occurred then $E_{j}$ occurred for all $j\leq k.$ So
\[
\Pr(\text{Never reaching $0$})\leq \lim_{i\to \infty} \frac{1}{2^i} =0
.\]
Note: The sequence of events $E_i$ is indexed by integers $i \in \mathbb{N}$, so the limit is perfectly well-defined. \\
So \[
\Pr( \text{Reach } 0 )=1-\Pr(\text{Never reaching $0$})=1
.\] 
\subsection*{Part 4}
Let $T$ be the number of steps until Alice reaches $0$
If Alice is currently at $2^i$ then the minimum number of steps until she gets to $0$ is $2^i$ so 

\begin{align*}
    \E[T]&\geq \sum_{i=1}^\infty \Pr( \text{Alice reaches $2^i$ before $0$ and did not reach $2^{i+1}$})\cdot 2^{i+1} \tag{$2^i$ to get from and to each}\\&=\sum_{i=1}^\infty \left( p_i-p_{i+1}\right)2^{i+1}
    \\&=\sum_{i=1}^\infty{\frac{1}{2^{i+1}}}\cdot 2^{i+1}\\
    &=\sum_{i=1}^\infty 1
\end{align*}
The RHS diverges so $\E[T]$ diverages as well.
\end{document}