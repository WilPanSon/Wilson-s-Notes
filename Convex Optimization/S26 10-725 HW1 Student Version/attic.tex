\newpage
\section{Implications of Smoothness and Strong Convexity (20 points) THIS SHOULD GO TO HW2}

Let $f$ be convex and twice continuously differentiable.

\begin{enumerate}
\item \textbf{(10 pts)} Show that the following statements are equivalent. 
\begin{enumerate}
\item $\nabla f$ is Lipschitz with constant $\beta$;
\item $(\nabla f(x) - \nabla f(y))^T(x-y) \leq \beta \|x-y\|_2^2$ for all
  $x,y$; 
\item $\nabla^2 f(x) \preceq \beta I$ for all $x$;
\item $f(y) \leq f(x) + \nabla f(x)^T (y-x) + \frac{\beta}{2} \|y-x\|_2^2$
  for all $x,y$.
\end{enumerate}
Your solution should have 5 parts, where you prove (a) $\Rightarrow$ (b), (b)
$\Rightarrow$ (c), (c) $\Rightarrow$ (d), (d) $\Rightarrow$ (b), and (c)
$\Rightarrow$ (a). 

\begin{soln}
(a) $\Rightarrow$ (b). From the $L$-Lipschitz condition, we know
that for all $x, y$: 
$$ \|\nabla f(x) - \nabla f(y)\|_2\leq L\|x - y\|_2$$
and by the Cauchy-Schwarz inequality, we have:
$$
(\nabla f(x) - \nabla f(y))^T(x - y) 
\leq \|\nabla f(x) - \nabla f(y)\|_2\|x - y\|_2
\leq L\|x - y\|_2^2.
$$

(b) $\Rightarrow$ (c). Let $h$ be any unit vector, i.e., $\|h\|_2 = 1$. Consider
the directional derivative of $\nabla f(x)$ at $h$: 
$$
\nabla^2 f(x) h =
\lim_{t\rightarrow 0}\frac{\nabla f(x+th) - \nabla f(x)}{t}.
$$ 
Taking inner product of both sides in the above equation with $h$ leads to
\begin{align*}
h^T\nabla^2 f(x) h 
&= \lim_{t\rightarrow 0}\frac{h^T(\nabla f(x+th) - \nabla f(x))}{t} \\
&= \lim_{t\rightarrow 0}\frac{th^T(\nabla f(x+th) - \nabla f(x))}{t^2} \\
&\leq \lim_{t\rightarrow 0}\frac{L\|th\|_2^2}{t^2} \\
&= L.
\end{align*}
Now choosing $h$ to be the eigenvector corresponding to the largest eigenvalue 
of $\nabla^2 f(x)$, we get
$$
\lambda_1(\nabla^2 f(x))\leq L,
$$
which implies that $\nabla^2 f(x)\preceq LI$.

(c) $\Rightarrow$ (d). By Taylor expansion with Lagrange form of the remainder: 
\begin{align*}
f(y) &= f(x) + \nabla f(x)^T(y-x) + \frac{1}{2}(y-x)^T\nabla^2 f(\lambda x + 
(1-\lambda)y) (y-x) \\
&\leq f(x) + \nabla f(x)^T(y-x) + \frac{L}{2}\|y-x\|_2^2, 
\end{align*}
where $\lambda \in(0, 1)$ and the last inequality is due to $\nabla^2 f(x)
\preceq LI$, for all $x$.

(d) $\Rightarrow$ (b). By iv, we have that the following two inequalities hold 
simultaneously: 
$$
f(y) \leq f(x) + \nabla f(x)^T(y-x) + \frac{L}{2}\|y-x\|_2^2,
$$
and
$$
f(x) \leq f(y) + \nabla f(y)^T(x-y) + \frac{L}{2}\|y-x\|_2^2.
$$
Simply adding these two inequalities leads to ii.

(c) $\Rightarrow$ (a). Applying the integral form of the mean-value theorem, we
have for all $x, y$:
$$
\nabla f(y) - \nabla f(x) = \int_0^1 \nabla^2 f(x + t(y-x))(y-x)~dt.
$$
Taking the norm of both sides above, 
\begin{align*}
\|\nabla f(y) - \nabla f(x)\|_2 &= \|\int_0^1 \nabla^2 f(x + t(y-x))(y-x)~dt\|_2 \\
& \leq \int_0^1 \|\nabla^2 f(x + t(y-x))(y-x)\|_2~dt \\
& \leq \int_0^1 \|\nabla^2 f(x + t(y-x))\|_2\|(y-x)\|_2~dt \\
& \leq L\|y-x\|_2\int_0^1~dt \\
& = L\|y-x\|_2.
\end{align*}

\end{soln}

\bigskip
\noindent
\item \textbf{(10 pts)} Show that the following statements are equivalent. 
\begin{enumerate}
\item $f$ is strongly convex with constant $\alpha$;
\item $(\nabla f(x) - \nabla f(y))^T(x-y) \geq \alpha \|x-y\|_2^2$ for all
  $x,y$;
\item $\nabla^2 f(x) \succeq \alpha I$ for all $x$;
\item $f(y) \geq f(x) + \nabla f(x)^T (y-x) + \frac{\alpha}{2}
  \|y-x\|_2^2$ for all $x,y$.
\end{enumerate}
Your solution should have 4 parts, where you prove (a) $\Rightarrow$ (b), (b)
$\Rightarrow$ (c), (c) $\Rightarrow$ (d), and (d) $\Rightarrow$ (a).

\begin{soln}
     (a) $\Rightarrow$ (b). Since $f(x)$ is $m$-strongly convex, we have
for all $x, y$: 
$$
f(y) - \frac{m}{2}\|y\|_2^2 \geq f(x) - \frac{m}{2}\|x\|_2^2 + (\nabla f(x) -
mx)^T(y-x),
$$
and
$$
f(x) - \frac{m}{2}\|x\|_2^2 \geq f(y) - \frac{m}{2}\|y\|_2^2 + (\nabla f(y) -
my)^T(x-y).
$$
Adding the above two inequalities, after some algebra, we have
$$
m\|x -y\|_2^2 \leq (\nabla f(x) - \nabla f(y))^T(x - y).
$$

(b) $\Rightarrow$ (c). Let $h$ be any unit vector. Consider the directional
derivative of $\nabla f(x)$ at $h$: 
$$
\nabla^2 f(x) h = \lim_{t\rightarrow 0}\frac{\nabla f(x+th) - \nabla f(x)}{t}.
$$
Taking the inner product of both sides above with $h$ leads to
\begin{align*}
h^T\nabla^2 f(x) h 
&= \lim_{t\rightarrow 0}\frac{h^T(\nabla f(x+th) - \nabla f(x))}{t}\\ 
&= \lim_{t\rightarrow 0}\frac{th^T(\nabla f(x+th) - \nabla f(x))}{t^2} \\
&\geq \lim_{t\rightarrow 0}\frac{m\|th\|_2^2}{t^2} \\
&= m.
\end{align*}
Now choosing $h$ to be the eigenvector corresponding to the smallest eigenvalue
of $\nabla^2 f(x)$, we get
$$
\lambda_n(\nabla^2 f(x))\geq m,
$$
which implies that $\nabla^2 f(x)\succeq mI$.

(c) $\Rightarrow$ (d). Again, using a Taylor expansion with Lagrange form for
the remainder, we have
\begin{align*}
f(y) &= f(x) + \nabla f(x)^T(y-x) + \frac{1}{2}(y-x)^T\nabla^2 f(\lambda x +
       (1-\lambda)y) (y-x) \\
&\geq f(x) + \nabla f(x)^T(y-x) + \frac{m}{2}\|y-x\|_2^2,
\end{align*}
where $\lambda\in(0, 1)$ and the last inequality is due to $\nabla^2 f(x)\succeq
mI$, for all $x$. 

(d) $\Rightarrow$ (a). To show that $f(x)$ is $m$-strongly convex, we need to show
that for all $x, y$:
$$
f(y) - \frac{m}{2}\|y\|_2^2 \geq f(x) - \frac{m}{2}\|x\|_2^2 + (\nabla f(x) -
mx)^T(y-x).
$$
Starting from 
$$
f(y) \geq f(x) + \nabla f(x)^T(y-x) + \frac{m}{2}\|y-x\|_2^2,
$$
subtract $\frac{m}{2}\|y\|_2^2$ from the both sides of the above
inequality. After some algebraic manipulation we arrive at the desired
inequality.   

\end{soln}

\end{enumerate}
